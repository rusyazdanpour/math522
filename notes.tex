\documentclass[11pt,oneside]{amsart}

\title{Math 522 Course Notes}
\author{Samuel Coskey}
\author{Erik Holmes}

\usepackage[vscale=.8]{geometry}
\usepackage{mathpazo}
\usepackage{amssymb}
\usepackage{setspace}\onehalfspacing
\renewcommand{\labelitemi}{$\circ$}
\renewcommand{\labelenumi}{(\alph{enumi})}
\usepackage{etoolbox}\makeatletter\pretocmd{\@seccntformat}{\S}{}{}\pretocmd{\@subseccntformat}{\S}{}{}\makeatother

\newcommand{\set}[1]{\left\{\,#1\,\right\}}
\newcommand{\NN}{\mathbb N}
\newcommand{\QQ}{\mathbb Q}
\newcommand{\RR}{\mathbb R}

\swapnumbers
\newtheorem{thm}{Theorem}[section]
\newtheorem{lem}[thm]{Lemma}
\newtheorem{prop}[thm]{Proposition}
\theoremstyle{definition}
\newtheorem{defn}[thm]{Definition}
\theoremstyle{remark}
\newtheorem{rem}[thm]{Remark}
\newtheorem{example}[thm]{Example}

\begin{document}
\maketitle

%%%%%%%%%%%%%%%%%%%%%%%%%%%%%%%%%%%%%%%%%%%%%%%%%%
\setcounter{section}{-1}
\section{Introduction}
%%%%%%%%%%%%%%%%%%%%%%%%%%%%%%%%%%%%%%%%%%%%%%%%%%

This course is about set theory, and its use in the study of the real line. By way of motivation, consider the question of mesauring the \emph{size} of a given set of real numbers. The word ``size'' can mean many different things, depending on the context. To see what I mean, consider these three important examples.

\begin{itemize}
\item To a set theorist or combinatorist, the simplest notion of size is
\emph{cardinality}, which asks for instance whether a set is finite, countably infinite, or uncountable. Of course if the set is uncountable, a set theorist would further ask whether it is of the first uncountable cardinality ($\aleph_1$), the second uncountable cardinality ($\aleph_2$), etc.

\item To an analyst, one standard notion of size is \emph{measure}, which asks for instance whether a set has zero length or positive length.

\item To a topologist, a key notion of size is \emph{category}, which asks whether a set is meager, comeager, or neither meager nor comeager.
\end{itemize}

In this course we will ask: how do these three kinds of size compare with one another? In answering this question, we will discover numerous relationships between measure and category, with the diagram of these relationships revealing a beautiful structure.

Our investigation of mesaure and category will lead us to a number of \emph{independence results}, that is, statements that cannot be decided using the usual axioms of set theory. The most central example of such a statement is the \emph{continuum hypothesis} (CH), which asserts that the set of all real numbers has cardinality equal to the first uncountable cardinal number. Cantor initially posed this problem in 1878. In 1940 G\"odel showed the axioms of set theory can't disprove CH, and in 1965 Cohen finally showed that the axioms of set theory can't prove CH.

In his proof of the latter half of the independence of CH, Cohen developed a tool for building models of set theory called \emph{forcing}. Far from being an isolated application, since then forcing has become a standard tool for establishing independence results. We will develop the machinery of forcing and give a number of such applications.

With the tool of forcing in hand, we will return to the notions of measure and category. We will find that there are numerous cardinal numbers surrounding the zero length sets and the meager sets whose values, like the size of the set of all real numbers, are also independent of set theory.


%%%%%%%%%%%%%%%%%%%%%%%%%%%%%%%%%%%%%%%%%%%%%%%%%%
\section{The real number system}
%%%%%%%%%%%%%%%%%%%%%%%%%%%%%%%%%%%%%%%%%%%%%%%%%%

\begin{defn}
  The \emph{real number system} is a complete ordered field. That is, it is a set $\RR$ together with distinguished elements $0,1$, operations $+,\times$ and an ordering $<$ satisfying the following rules.
\begin{itemize}
\item It is a \emph{field}: it satsfies the associative, commutative, and distributive laws, $0$ is the additive identity and $1$ is the multiplicative identity;
\item It is an \emph{ordered} field: $<$ is a strict total order, $a<b$ implies $a+c<b+c$, and if $c$ is positive then $a<b$ implies $ac<bc$;
\item It is \emph{complete}: if $A\subset\RR$ has an upper bound, then it has a least upper bound (supremum).
\end{itemize}
\end{defn}

\begin{rem}
  It is not difficult to prove that any two complete ordered fields are isomorphic to each other, so it makes sense to define the real number system as the unique such object.
\end{rem}

\begin{rem}
  The above definition of the real number system is an abstract, axiomatic definition. That is, it doesn't tell you how real numbers behave, but it leaves it up to your imagination what real numbers really are. In fact, we have some choice about this. For example we can view the real numbers as decimal strings or binary strings. For our set-theoretic purposes, all that matters is that we give one construction of the real numbers that works.
\end{rem}

The first to give an explicit and rigorous construction of the real number system was Dedekind, who did so in the 19th century. Here is how his construction worked. First we will take the construction of the rational number system for granted. It is simply the set of fractions whose numerator is an integer and whose denominator is a nonzero natural number.

The key insight is to realize that each real number $r$ has a \emph{cut} in the rational numbers associated with it: the set of rational numbers that are strictly less than $r$. This association should be bijective: If the least upper bound property is true, then any such cut gives rise to a real number as the supremum of the cut. And if two real numbers are different, then they should give rise to different cuts.

So all we have to do to define real numbers was to define cuts. To avoid circularity, we should axiomatize cuts without referring to real numbers at all. A \emph{cut} in the rational numbers is a subset $r\subset\QQ$ satisfying three properties:
\begin{itemize}
\item nontriviality: $r\neq\emptyset$ and $r\neq\QQ$
\item downward closure: if $q\in r$ and $q'<q$ then $q'\in r$
\item no last element: if $q\in r$ then there is $q'>q$ such that $q'\in r$
\end{itemize}
To complete the construction one must say how the $+,\times$ operations and $<$ relation can be defined just in terms of cuts, and then prove that they satisfy the axioms of a complete ordered field. For example, one defines $r+r'$ to be the supremum of the set $\set{q+q'\mid q\in r,q'\in r'}$, and one defines $r<r'$ if and only if $r\subset r'$ and $r\neq r'$. We omit the details of verifying that these definitions work.

\begin{rem}
  It is interesting to compare Dedekind's construction with the modern decimal number construction (which we omit). The decimal number construction is practical for calculations, but has some oddities. For example some numbers have two decimal representations, like $.999\cdots$ and $1$. Also we made an arbitrary decision to use decimal expansions as opposed to any other base. Dedekind's construction is beautiful because it is completely uniform and does not involve any arbitrary decisions.
\end{rem}

\begin{defn}
  A set $A$ of real numbers is said to be \emph{countable} if its elements can be enumerated using natural number indices. In other words, $A$ is countable if there exists a sequence $(a_n)_{n\in\NN}$ such that $A=\set{a_n\mid n\in\NN}$.
\end{defn}

Note that this definition implies that finite sets are countable. If we wish to emphasize that a particular countable set is infinite, we will use the term \emph{countably infinite}. A set that is not countable is called \emph{uncountable}. We now give Cantor's famous proof of the fact that the set of all real numbers is uncountable. The argument goes by contradiction and uses a recursive construction to reach a contradiction one step at a time. Such arguments are often called ``diagonal'', for reasons we shall see later on.

\begin{thm}[Cantor]
  \label{thm:cantor}
  The set of all real numbers is uncountable.
\end{thm}

\begin{proof}
  Suppose towards a contradiction that the set of all real numbers is countable. Then there exists a sequence $(a_n)_{n\in\NN}$ such that $\RR=\set{a_n\mid n\in\NN}$. Our strategy will be to inductively construct a decreasing sequence of closed intervals $I_n=[l_n,r_r]$ such that for all $n$, $a_n\notin I_n$. Then if $r$ lies in the intersection of these intervals, $r$ cannot be equal to $a_n$ for any $n$, a contradiction.

The construction itself is straightforward. For the base case let $I_1=[l_1,r_1]$ be any interval which omits $a_1$. For the inductive step, if $I_{n-1}$ has been defined, let $I_n=[l_n,r_n]$ be any subinterval of $I_{n-1}$ which omits $a_n$.

We can now find a point in the intersection of the $I_n$ using the completeness property. First observe that since $I_n\subset I_{n-1}$ for all $n$, the set of left endpoints $\set{l_n\mid n\in\NN}$ is bounded above by any and all of the right endpoints $r_n$. By the completeness property, the set of left endpoints has a least upper bound $x$. Since $x$ is an upper bound for $\set{l_n\mid n\in\NN}$, we have $l_n\leq x$ for all $n$. Since $x$ is the least possible upper bound for $\set{l_n\mid n\in\NN}$, we have $x\leq r_n$ for all $n$. Therefore we have $x\in[l_n,r_n]=I_n$ for all $n$.

Since $x$ lies in $I_n$ for all $n$, and since we chose $I_n$ in such a way that it omits $a_n$, we know that $x\neq a_n$ for all $n$. This contradicts the hypothesis that $\RR=\set{a_n\mid n\in\NN}$, and concludes the proof.
\end{proof}

This argument appears in one form or another numerous times throughout elementary analysis and set theory. We shall next see it in the proof of the Baire category theorem.

%%%%%%%%%%%%%%%%%%%%%%%%%%%%%%%%%%%%%%%%%%%%%%%%%%
\section{Baire category theory}
%%%%%%%%%%%%%%%%%%%%%%%%%%%%%%%%%%%%%%%%%%%%%%%%%%

\begin{defn}
  A set of real numbers $A$ is said to be \emph{nowhere dense} if every positive-length interval $I$ contains a positive-length interval $J$ such that $J$ is disjoint from $A$.
\end{defn}

\begin{example}
  Any finite subset of $\RR$ is nowhere dense. The set $\set{1/n\mid n\in\NN}$ is nowhere dense. In fact, any \emph{discrete} set is nowhere dense (here, $A$ is discrete if every point of $A$ is isolated). The set of rational numbers whose numerator is $1$ and whose denominator is a power of $2$ is \emph{not} nowhere dense, because such a number can be found in every positive-length subinterval of $[0,1]$.
\end{example}

Note that nowhere dense sets need not be countable. The classical Cantor middle thirds set is an archetypal example of a nowhere dense set. Recall that the \emph{Cantor set} is defined as follows. Let $I_1=[0,1]$ and let $I_2=[0,1/3]\cup[2/3,1]$ be the set $I_1$ with its open middle third removed. Inductively let $I_n$ be constructed from $I_{n-1}$ by removing the open middle third from every interval of $I_{n-1}$. Finally the Cantor set $C$ is defined to be $C=\bigcap I_n$.

\begin{prop}
  The Cantor set $C$ is nowhere dense.
\end{prop}

\begin{proof}
  First compute that the sum of the lengths of all of those middle thirds removed in the construction of the cantor set is exactly $1$ (we leave this as an exercise). It follows from this that $C$ does not contain any positive-length intervals. Now if $I$ is any positive-length interval, then $I$ must contain a point $x\notin C$. Now note that $C$ is closed because it is an intersection of closed sets, and henc $\RR\smallsetminus C$ is open. Thus there is an open interval $J$ centered at $x$ such that $J\subset\RR\smallsetminus C$. Shrinking the radius of $J$ if necessary, we can suppose that $J\subset I$. This shows that $C$ is nowhere dense.
\end{proof}

The terminology ``nowhere dense'' may sound strange at first, but it is justified by the following fact.

\begin{prop}
  \label{prop:nwd-equiv}
  Let $A$ be a set of real numbers. The following are equivalent.
\begin{enumerate}
\item $A$ is nowhere dense;
\item the closure of $A$ has no interior;
\item for every nonempty open set $O$, we have that $A$ is not dense in $O$.
\end{enumerate}
\end{prop}

We leave the proof as an exercise.

\begin{thm}
  The class of nowhere dense sets is closed under the operations of subset, union, and closure.
\end{thm}

\begin{proof}
  Preservation to subsets is immediate from the definition.

  For preservation to unions, suppose that $A,B$ are nowhere dense and let $I$ be a positive-length interval. Since $A$ is nowhere dense, there is a positive-length interval $J\subset I\smallsetminus A$. Since $B$ is nowhere dense there is a further positive-length interval $J'\subset J\smallsetminus B$. It follows that
\[J'\subset (I\smallsetminus A)\smallsetminus B=I\smallsetminus (A\cup B)
\]
This establishes that $A\cup B$ is again nowhere dense.

Preservation to closures is immediate from condition (b) of Proposition~\ref{prop:nwd-equiv}, but we can also give a direct argument. Let $A$ be nowhere dense and let $I$ be a given positive-length interval. Since $A$ is nowhere dense there is a positive-length interval $J$ such that $J\subset\RR\smallsetminus A$. Removing endpoints if necessary, we may suppose that $J$ is open. It follows that $J\subset\RR\smallsetminus\bar{A}$ (recall that since $\bar{A}$ is the intersection of all closed sets containing $A$, we have that $\RR\smallsetminus\bar{A}$ is the union of all open sets disjoint from $A$). This shows that $\bar{A}$ is nowhere dense.
\end{proof}

\begin{rem}
  It is easy to see that the class of nowhere dense sets is \emph{not} preserved to infinite unions, even countably infinite ones. For example, the set of rational numbers is countable, and therefore it is a countable unions of singletons: $\QQ=\bigcup_{q\in\QQ}\{q\}$. Each singleton $\{q\}$ is clearly nowhere dense, so $\QQ$ is a countable union of nowhere dense sets, but $\QQ$ is dense.
\end{rem}

\begin{defn}
  A set of real numbers $A$ is called \emph{meager} if it is a union of countably many nowhere dense sets. A set $A$ is called \emph{comeager} if $\RR\smallsetminus A$ is meager.
\end{defn}

In principal it is possible that this notion is completely moot in the sense that maybe every set of real numbers is meager according to this definition. The Baire category theorem says that this is not the case. In fact, we will soon see that meager sets are rather paultry as the name implies.

\begin{thm}[Baire category theorem]
  The set of all real numbers $\RR$ is nonmeager. In fact, if $A$ is any set of real numbers whose complement is not dense, then $A$ is nonmeager.
\end{thm}

\begin{proof}
  The proof is very similar to Cantor's proof of Theorem~\ref{thm:cantor}. Suppose towards a contradiction that $\RR$ is meager. Then there exists a sequence $(A_n)_{n\in\NN}$ of nowhere dense sets such that $\bigcup A_n=\RR$. We will inductively construct a decreasing sequence of closed intervals $J_n$ such that for all $n$, $J_n\cap N_n=\emptyset$.

  The induction is again straightforward. Begin with an arbitrary positive-length interval $I$, and apply the definition of $A_1$ being nowhere dense to obtain $J_1$. Inductively apply the definition of $A_n$ being nowhere dense to $J_n$ to obtain $J_{n+1}$. (Each time we may shrink the newly obtained interval slightly to suppose that it is closed.)

  We can now argue just as in the proof of Theorem~\ref{thm:cantor} that there exists a point $x\in\bigcap I_n$. It follows that $x\notin\bigcup A_n$, which contradicts the assumption that $\RR=\bigcup A_n$, and completes the proof of the first statement.

  For the second statement, suppose that the complement of $A$ is not dense and that $A=\bigcup A_n$ is a countable union of nowhere dense sets. In the above argument the initial interval $I$ was chosen arbitrarily. In this argument since the complement of $A$ is not dense we may let $I$ be a positive-length subinterval of $A$. The rest of the argument proceeds identically, and gives a point $x\in A$ such that $x\notin\bigcup A_n$, a contradiction.
\end{proof}

\begin{rem}
  Baire category theory gets its name from the above theorem, and the fact that meager sets used to be called ``first category''. Nonmeager sets used to be called ``second category''.
\end{rem}

\begin{thm}
  \label{thm:meager-pres}
  The class of meager sets is closed under the operations of subset and countable union.
\end{thm}

We remark that the meager property is not necessarily preserved to the closure. In fact $\bar{A}$ is meager if and only if $A$ is nowhere dense.

As a consequence of the Baire category theorem, for any set $A$ it is possible to assign $A$ one of three distinct ``sizes'':
\begin{itemize}
\item $A$ is meager (small)
\item $A$ is comeager (large)
\item $A$ neither meager nor comeager (intermediate)
\end{itemize}
We should verify that no set of numbers can be both meager and comeager. Indeed, suppose that both $A$ and $\RR\smallsetminus A$ were meager. Then by Theorem~\ref{thm:meager-pres} their union $\RR$ would be meager, contradicting the Baire category theorem.

We should also verify that there exists sets of numbers that are neither meager nor comeager. Indeed, if $I$ is any bounded positive-length interval then by the second statement of the Baire category theorem $I$ is nonmeager, and since $\RR\smallsetminus I$ also contains a bounded interval, $I$ is also non-comeager.



%%%%%%%%%%%%%%%%%%%%%%%%%%%%%%%%%%%%%%%%%%%%%%%%%%
\section{Measure theory}
%%%%%%%%%%%%%%%%%%%%%%%%%%%%%%%%%%%%%%%%%%%%%%%%%%

%%%%%%%%%%%%%%%%%%%%%%%%%%%%%%%%%%%%%%%%%%%%%%%%%%
\section{Sets, functions, and cardinality}
%%%%%%%%%%%%%%%%%%%%%%%%%%%%%%%%%%%%%%%%%%%%%%%%%%

\end{document}
