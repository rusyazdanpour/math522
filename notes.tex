\documentclass[11pt,oneside]{amsart}

\title{Math 522 Course Notes}
\author{Samuel Coskey}
\author{Erik Holmes}

\usepackage[vscale=.8]{geometry}
\usepackage{mathpazo}
\usepackage{amssymb}
\usepackage{setspace}\onehalfspacing
\renewcommand{\labelitemi}{$\circ$}
\renewcommand{\labelenumi}{(\alph{enumi})}
\usepackage{etoolbox}\makeatletter\pretocmd{\@seccntformat}{\S}{}{}\pretocmd{\@subseccntformat}{\S}{}{}\makeatother
\usepackage{tikz}
\usetikzlibrary{decorations.fractals}

\newcommand{\set}[1]{\left\{\,#1\,\right\}}
\DeclareMathOperator{\cl}{cl}
\newcommand{\NN}{\mathbb N}
\newcommand{\QQ}{\mathbb Q}
\newcommand{\RR}{\mathbb R}

\theoremstyle{definition}
\newtheorem{exerc}{Exercise}[section]
\swapnumbers
\newtheorem{thm}{Theorem}[section]
\newtheorem{cor}[thm]{Corollary}
\newtheorem{lem}[thm]{Lemma}
\newtheorem{prop}[thm]{Proposition}
\theoremstyle{definition}
\newtheorem{defn}[thm]{Definition}
\theoremstyle{remark}
\newtheorem{rem}[thm]{Remark}
\newtheorem{example}[thm]{Example}

\begin{document}
\maketitle

%%%%%%%%%%%%%%%%%%%%%%%%%%%%%%%%%%%%%%%%%%%%%%%%%%
\setcounter{section}{-1}
\section{Introduction}
%%%%%%%%%%%%%%%%%%%%%%%%%%%%%%%%%%%%%%%%%%%%%%%%%%

This course is about set theory, and its use in the study of the real line. By way of motivation, consider the question of mesauring the \emph{size} of a given set of real numbers. The word ``size'' can mean many different things, depending on the context. To see what I mean, consider these three important examples.

\begin{itemize}
\item To a set theorist or combinatorist, the simplest notion of size is
\emph{cardinality}, which asks for instance whether a set is finite, countably infinite, or uncountable. Of course if the set is uncountable, a set theorist would further ask whether it is of the first uncountable cardinality ($\aleph_1$), the second uncountable cardinality ($\aleph_2$), etc.

\item To an analyst, one standard notion of size is \emph{measure}, which asks for instance whether a set has zero length or positive length.

\item To a topologist, a key notion of size is \emph{category}, which asks whether a set is meager, comeager, or neither meager nor comeager.
\end{itemize}

In this course we will ask: how do these three kinds of size compare with one another? In answering this question, we will discover numerous relationships between measure and category, with the diagram of these relationships revealing a beautiful structure.

Our investigation of mesaure and category will lead us to a number of statements that are \emph{independent} of the axioms of set theory. This means that such statements cannot be proved true or false using the usual axioms of set theory. The most central example of such a statement is the \emph{continuum hypothesis} (CH), which asserts that the set of all real numbers has cardinality equal to the first uncountable cardinal number. Cantor initially posed this problem in 1878. In 1940 G\"odel showed the axioms of set theory can't disprove CH, and in 1965 Cohen finally showed that the axioms of set theory can't prove CH.

In his proof of the latter half of the independence of CH, Cohen developed a tool for building models of set theory called \emph{forcing}. Far from being an isolated application, since then forcing has become a standard tool for establishing independence results. We will develop the machinery of forcing and give a number of such applications.

With the tool of forcing in hand, we will return to the notions of measure and category. We will find that there are numerous cardinal numbers surrounding the zero length sets and the meager sets whose values, like the size of the set of all real numbers, are also independent of the axioms of set theory.


%%%%%%%%%%%%%%%%%%%%%%%%%%%%%%%%%%%%%%%%%%%%%%%%%%
\section{The real number system}
%%%%%%%%%%%%%%%%%%%%%%%%%%%%%%%%%%%%%%%%%%%%%%%%%%

\begin{defn}
  The \emph{real number system} is a complete ordered field. That is, it is a set $\RR$ together with distinguished elements $0,1$, operations $+,\times$ and an ordering $<$ satisfying the following rules.
\begin{itemize}
\item It is a \emph{field}: it satsfies the associative, commutative, and distributive laws, $0$ is the additive identity and $1$ is the multiplicative identity;
\item It is an \emph{ordered} field: $<$ is a strict total order, $a<b$ implies $a+c<b+c$, and if $c$ is positive then $a<b$ implies $ac<bc$;
\item It is \emph{complete}: if $A\subset\RR$ has an upper bound, then it has a least upper bound (supremum).
\end{itemize}
\end{defn}

\begin{rem}
  It is not difficult to prove that any two complete ordered fields are isomorphic to each other, so it makes sense to define the real number system as the unique such object.
\end{rem}

\begin{rem}
  The above definition of the real number system is an abstract, axiomatic definition. That is, it doesn't tell you how real numbers behave, but it leaves it up to your imagination what real numbers really are. In fact, we have some choice about this. For example we can view the real numbers as decimal strings or binary strings. For our set-theoretic purposes, all that matters is that we give one construction of the real numbers that works.
\end{rem}

The first to give an explicit and rigorous construction of the real number system was Dedekind, who did so in the 19th century. Here is how his construction worked. First we will take the construction of the rational number system for granted. It is simply the set of fractions whose numerator is an integer and whose denominator is a nonzero natural number.

The key insight is to realize that each real number $r$ has a \emph{cut} in the rational numbers associated with it: the set of rational numbers that are strictly less than $r$. This association should be bijective: If the least upper bound property is true, then any such cut gives rise to a real number as the supremum of the cut. And if two real numbers are different, then they should give rise to different cuts.

So all we have to do to define real numbers was to define cuts. To avoid circularity, we should axiomatize cuts without referring to real numbers at all. A \emph{cut} in the rational numbers is a subset $r\subset\QQ$ satisfying three properties:
\begin{itemize}
\item nontriviality: $r\neq\emptyset$ and $r\neq\QQ$
\item downward closure: if $q\in r$ and $q'<q$ then $q'\in r$
\item no last element: if $q\in r$ then there is $q'>q$ such that $q'\in r$
\end{itemize}
To complete the construction one must say how the $+,\times$ operations and $<$ relation can be defined just in terms of cuts, and then prove that they satisfy the axioms of a complete ordered field. For example, one defines $r+r'$ to be the supremum of the set $\set{q+q'\mid q\in r,q'\in r'}$, and one defines $r<r'$ if and only if $r\subset r'$ and $r\neq r'$. We omit the details of verifying that these definitions work.

\begin{rem}
  It is interesting to compare Dedekind's construction with the modern decimal number construction (which we omit). The decimal number construction is practical for calculations, but has some oddities. For example some numbers have two decimal representations, like $.999\cdots$ and $1$. Also we made an arbitrary decision to use decimal expansions as opposed to any other base. Dedekind's construction is beautiful because it is completely uniform and does not involve any arbitrary decisions.
\end{rem}

\begin{defn}
  A set $A$ of real numbers is said to be \emph{countable} if its elements can be enumerated using natural number indices. In other words, $A$ is countable if there exists a sequence $(a_n)_{n\in\NN}$ such that $A=\set{a_n\mid n\in\NN}$.
\end{defn}

Note that this definition implies that finite sets are countable. If we wish to emphasize that a particular countable set is infinite, we will use the term \emph{countably infinite}. A set that is not countable is called \emph{uncountable}. We now give Cantor's famous proof of the fact that the set of all real numbers is uncountable. The argument goes by contradiction and uses a recursive construction to reach a contradiction one step at a time. Such arguments are often called ``diagonal'', for reasons we shall see later on.

\begin{thm}[Cantor]
  \label{thm:cantor}
  The set of all real numbers is uncountable.
\end{thm}

\begin{proof}
  Suppose towards a contradiction that the set of all real numbers is countable. Then there exists a sequence $(a_n)_{n\in\NN}$ such that $\RR=\set{a_n\mid n\in\NN}$. Our strategy will be to inductively construct a decreasing sequence of closed intervals $I_n=[l_n,r_r]$ such that for all $n$, $a_n\notin I_n$. Then if $r$ lies in the intersection of these intervals, $r$ cannot be equal to $a_n$ for any $n$, a contradiction.

The construction itself is straightforward. For the base case let $I_1=[l_1,r_1]$ be any interval which omits $a_1$. For the inductive step, if $I_{n-1}$ has been defined, let $I_n=[l_n,r_n]$ be any subinterval of $I_{n-1}$ which omits $a_n$.

We can now find a point in the intersection of the $I_n$ using the completeness property. First observe that since $I_n\subset I_{n-1}$ for all $n$, the set of left endpoints $\set{l_n\mid n\in\NN}$ is bounded above by any and all of the right endpoints $r_n$. By the completeness property, the set of left endpoints has a least upper bound $x$. Since $x$ is an upper bound for $\set{l_n\mid n\in\NN}$, we have $l_n\leq x$ for all $n$. Since $x$ is the least possible upper bound for $\set{l_n\mid n\in\NN}$, we have $x\leq r_n$ for all $n$. Therefore we have $x\in[l_n,r_n]=I_n$ for all $n$.

Since $x$ lies in $I_n$ for all $n$, and since we chose $I_n$ in such a way that it omits $a_n$, we know that $x\neq a_n$ for all $n$. This contradicts the hypothesis that $\RR=\set{a_n\mid n\in\NN}$, and concludes the proof.
\end{proof}

This argument appears in one form or another numerous times throughout elementary analysis and set theory. We shall next see it in the proof of the Baire category theorem.

\begin{exerc}
  With the definition of $r+r'$ for Dedekind cuts, show that addition is commutative and associative.
\end{exerc}

\begin{exerc}
  Define multiplication $r\times r'$ for Dedekind cuts, and show that it agrees with multiplication for rational numbers. Hint: consider four cases when $r,r'$ are negative or nonnegative.
\end{exerc}

\begin{exerc}
  Let $A$ be a set of real numbers containing a positive-length interval. Show that $A$ is uncountable.
\end{exerc}

%%%%%%%%%%%%%%%%%%%%%%%%%%%%%%%%%%%%%%%%%%%%%%%%%%
\section{Baire category theory}
%%%%%%%%%%%%%%%%%%%%%%%%%%%%%%%%%%%%%%%%%%%%%%%%%%

\begin{defn}
  A set of real numbers $A$ is said to be \emph{nowhere dense} if every positive-length interval $I$ contains a positive-length interval $J$ such that $J$ is disjoint from $A$.
\end{defn}

\begin{example}
  Any finite subset of $\RR$ is nowhere dense. The set $\set{1/n\mid n\in\NN}$ is nowhere dense. In fact, any \emph{discrete} set is nowhere dense (here, $A$ is discrete if every point of $A$ is isolated). The set of rational numbers whose numerator is $1$ and whose denominator is a power of $2$ is \emph{not} nowhere dense, because such a number can be found in every positive-length subinterval of $[0,1]$.
\end{example}

Note that nowhere dense sets need not be countable. The classical Cantor middle thirds set is an archetypal example of a nowhere dense set. Recall that the \emph{Cantor set} is defined as follows. Begin with the set $C_1=[0,1]$. Let $C_2=[0,1/3]\cup[2/3,1]$ be the set $C_1$ with its open middle third removed. Let $C_3=[1,1/9]\cup[2/9,1/3]\cup[2/3,7/9]\cup[8/9,1]$ be the set $C_2$ with the open middle third removed from each component of $C_2$. Refer to Figure~\ref{fig:cantor-set} for a picture of these first few sets. Recursively, let $C_{n+1}$ be constructed from $C_n$ by removing the open middle third from each component of $C_n$. Finally the Cantor set $C$ is defined to be $C=\bigcap C_n$. We will see in a later section that the Cantor set is uncountable. % add ref

\begin{figure}[h]
\begin{center}
  \begin{tikzpicture}
    \draw[|-|] (0,0) -- (10,0) node[anchor=west] {\ \ $C_1$};
    \draw[|-|] (0,-.5) -- (10/3,-.5);
    \draw[|-|] (20/3,-.5) -- (10,-.5) node[anchor=west] {\ \ $C_2$};
    \draw[|-|] (0,-1) -- (10/9,-1);
    \draw[|-|] (20/9,-1) -- (30/9,-1);
    \draw[|-|] (60/9,-1) -- (70/9,-1);
    \draw[|-|] (80/9,-1) -- (10,-1) node[anchor=west] {\ \ $C_3$};
    \node[anchor=west] at (10,-1.4) {\ \ \ $\vdots$};
    \draw[decoration=Cantor set,very thick]
    decorate{ decorate{ decorate{ decorate{ decorate{ decorate{
                (0,-2) -- (10,-2) }}}}}} node[anchor=west] {\ \ $C$};
  \end{tikzpicture}
\end{center}
\caption{The first few steps in the construction of the Cantor set.\label{fig:cantor-set}}
\end{figure}

\begin{prop}
  The Cantor set $C$ is nowhere dense.
\end{prop}

\begin{proof}
  First compute that the sum of the lengths of all of those middle thirds removed in the construction of the cantor set is exactly $1$ (Exercise~\ref{exerc:cantor}). It follows from this that $C$ does not contain any positive-length intervals. Now if $I$ is any positive-length interval, then $I$ must contain a point $x\notin C$. Now note that $C$ is closed because it is an intersection of closed sets, and henc $\RR\smallsetminus C$ is open. Thus there is an open interval $J$ centered at $x$ such that $J\subset\RR\smallsetminus C$. Shrinking the radius of $J$ if necessary, we can suppose that $J\subset I$. This shows that $C$ is nowhere dense.
\end{proof}

The terminology ``nowhere dense'' may sound strange at first, but it is justified by the following fact.

\begin{prop}
  \label{prop:nwd-equiv}
  Let $A$ be a set of real numbers. The following are equivalent.
\begin{enumerate}
\item $A$ is nowhere dense;
\item $\cl(A)$ (the topological closure of $A$) has no interior;
\item for every nonempty open set $O$, we have that $A$ is not dense in $O$.
\end{enumerate}
\end{prop}

The proof is requested in Exercise~\ref{exerc:nwd-equiv}.

\begin{thm}
  The class of nowhere dense sets is closed under the operations of subset, union, and closure.
\end{thm}

\begin{proof}
  Preservation to subsets is immediate from the definition.

  For preservation to unions, suppose that $A,B$ are nowhere dense and let $I$ be a positive-length interval. Since $A$ is nowhere dense, there is a positive-length interval $J\subset I\smallsetminus A$. Since $B$ is nowhere dense there is a further positive-length interval $J'\subset J\smallsetminus B$. It follows that
\[J'\subset (I\smallsetminus A)\smallsetminus B=I\smallsetminus (A\cup B)
\]
This establishes that $A\cup B$ is again nowhere dense.

Preservation to closures is immediate from condition (b) of Proposition~\ref{prop:nwd-equiv}, but we can also give a direct argument. Let $A$ be nowhere dense and let $I$ be a given positive-length interval. Since $A$ is nowhere dense there is a positive-length interval $J$ such that $J\subset\RR\smallsetminus A$. Removing endpoints if necessary, we may suppose that $J$ is open. It follows that $J\subset\RR\smallsetminus\cl(A)$ (recall that since $\cl(A)$ is the intersection of all closed sets containing $A$, we have that $\RR\smallsetminus\cl(A)$ is the union of all open sets disjoint from $A$). This shows that $\cl(A)$ is nowhere dense.
\end{proof}

\begin{rem}
  It is easy to see that the class of nowhere dense sets is \emph{not} preserved to infinite unions, even countably infinite ones. For example, the set of rational numbers is countable, and therefore it is a countable unions of singletons: $\QQ=\bigcup_{q\in\QQ}\{q\}$. Each singleton $\{q\}$ is clearly nowhere dense, so $\QQ$ is a countable union of nowhere dense sets, but $\QQ$ is dense.
\end{rem}

\begin{defn}
  A set of real numbers $A$ is called \emph{meager} if it is a union of countably many nowhere dense sets. A set $A$ is called \emph{comeager} if $\RR\smallsetminus A$ is meager.
\end{defn}

In principal it is possible that this notion is completely moot in the sense that maybe every set of real numbers is meager according to this definition. The Baire category theorem says that this is not the case. In fact, we will soon see that meager sets are rather paultry as the name implies.

\begin{thm}[Baire category theorem]
  The set of all real numbers $\RR$ is nonmeager. In fact, if $I$ is a positive-length interval then $I$ is nonmeager.
\end{thm}

\begin{proof}
  The proof is very similar to Cantor's proof of Theorem~\ref{thm:cantor}. Let $I$ be a positive-length interval and suppose towards a contradiction that $I$ is meager. Then there exists a sequence $(A_n)_{n\in\NN}$ of nowhere dense sets such that $\bigcup A_n=I$. We will inductively construct a decreasing sequence of closed subintervals $J_n\subset I$ such that for all $n$, $J_n\cap N_n=\emptyset$.

  The induction is again straightforward. First apply the definition of $A_1$ being nowhere dense to obtain $J_1$. Inductively apply the definition of $A_n$ being nowhere dense to $J_n$ to obtain $J_{n+1}$. (Each time we may shrink the newly obtained interval slightly to suppose that it is closed.)

  We can now argue just as in the proof of Theorem~\ref{thm:cantor} that there exists a point $x\in\bigcap J_n$. It follows that $x\notin\bigcup A_n$, which contradicts the assumption that $I=\bigcup A_n$, and completes the proof.
\end{proof}

\begin{rem}
  Baire category theory gets its name from the above theorem, and the fact that meager sets used to be called ``first category''. Nonmeager sets used to be called ``second category''.
\end{rem}

\begin{thm}
  \label{thm:meager-pres}
  The class of meager sets is closed under the operations of subset and countable union.
\end{thm}

We remark that the meager property is not necessarily preserved to the closure. In fact $\cl(A)$ is meager if and only if $A$ is nowhere dense.

As a consequence of the Baire category theorem, for any set $A$ it is possible to assign $A$ one of three distinct ``sizes'':
\begin{itemize}
\item $A$ is meager (small)
\item $A$ is comeager (large)
\item $A$ neither meager nor comeager (intermediate)
\end{itemize}
We should verify that no set of numbers can be both meager and comeager. Indeed, suppose that both $A$ and $\RR\smallsetminus A$ were meager. Then by Theorem~\ref{thm:meager-pres} their union $\RR$ would be meager, contradicting the Baire category theorem.

We should also verify that there exists sets of numbers that are neither meager nor comeager. Indeed, if $I$ is any bounded positive-length interval then by the Baire category theorem $I$ is nonmeager, and since $\RR\smallsetminus I$ also contains a bounded interval, $I$ is also non-comeager.

\begin{exerc}
  \label{exerc:cantor}
  Compute the sum of the lengths of all of the intervals removed from $[0,1]$ in the construction of the Cantor set. What if some other fraction is removed at each stage?
\end{exerc}

\begin{exerc}
  \label{exerc:nwd-equiv}
  Prove Proposition~\ref{prop:nwd-equiv}.
\end{exerc}

\begin{exerc}
  We have observed that unlike the nowhere dense property, the meager property is not necessarily preserved to the closure. Prove that if $\cl(A)$ is meager, then $A$ was nowhere dense to begin with.
\end{exerc}

%%%%%%%%%%%%%%%%%%%%%%%%%%%%%%%%%%%%%%%%%%%%%%%%%%
\section{Lebesgue measure theory}
%%%%%%%%%%%%%%%%%%%%%%%%%%%%%%%%%%%%%%%%%%%%%%%%%%

Classical measure theory aims to try to extend the length function on intervals to be defined on more complicated sets. For example, if a given set is a finite or even countable union of intervals, then it is appropriate to take the sum of the lengths of the components. But what about more unwieldy sets like the Cantor set? This was known as the \emph{measure problem}: to construct a measurement function $m$, defined on sets of real numbers and valued in $[0,\infty]$, satisfying:
\begin{enumerate}
\item $m$ agrees with length: if $I$ is an interval then $m(I)=\ell(I)$
\item translation invariance: $m(x+A)=m(A)$ for all sets $A$ and real numbers $x$
\item countably additivity: if $(A_n)_{n\in\NN}$ is a sequence pairwise disjoint sets then $m(\bigcup A_n)=\sum m(A_n)$
\end{enumerate}

Perhaps surprisingly, the conditions (a)--(c) are mutually contradictory and so no such measurement function $m$ exists! Here is Vitali's simple example of a contradiction arising from these requirements. Regard $\QQ$ as an additive subgroup of $\RR$ and consider the cosets of $\QQ$, that is, sets of the form $x+\QQ$. Let $A\subset[0,1]$ be a set of coset representatives, that is, $A$ contains exactly one element from each of the cosets. (It is always possible to choose a coset representative in $[0,1]$ because $\QQ$ is dense.)

Then the translations $q+A$ for $q\in\QQ$ form a countable sequence of pairwise disjoint sets that cover all of $\RR$. In fact, if we let $S=\QQ\cap[-1,1]$ then the translations $q+A$ for $q\in S$ already cover all of $[0,1]$. We can then infer from (a) and (c) that $m[\bigcup_{q\in S}(q+A)]$ lies between $1$ and $4$. But on the other hand, by (b) and (c) we have that 
\[m\left(\bigcup_{q\in S}(q+A)\right)=\sum_{q\in S}m(q+A)=\sum_{q\in S}m(A)
\]
This is a contradiction because an infinite sum of a single constant value $m(A)$ can be equal only to either $0$ or $\infty$, and so cannot lie between $1$ and $4$.

\begin{rem}
  We observe that the axiom of choice was explicitly used in Vitali's construction of the set $A$ above. In fact, it is known that the use of the axiom of choice is necessary to build a counterexample.
\end{rem}

The contradiction described above is typically resolved not by dropping one of the conditions (a)--(c), but rather by dropping the requirement that $m$ is defined on \emph{all} sets of real numbers. The justification for this decision is that sets like the $A$ constructed above should be regarded as pathological, and we don't usually need to measure them in applications.

Let us na\"ively begin to construct a measure $m$ that is at least defined on reasonable sets. First, condition (a) implies we should let $m(I)=\ell(I)$ for any interval $I$. Next, if $A=\bigcup I_n$ is a union of disjoint intervals $I_n$, then condition (c) implies we should let $m(A)=\sum\ell(I_n)$. Third, if $A=\bigcap A_n$ is an intersection of sets where $m$ is defined and finite, then it is natural to define $m(A)=\inf m(A_n)$. We now observe that all three of the above simple cases fall under the following formula:
\[m(A)=\inf\set{\sum\ell(I_n)\mid A\subset\bigcup I_n}
\]
The next result states that this rule for defining $m$ actually works for all sets that are reasonably constructible. Recall that a \emph{Borel set} is one that can be constructed by beginning with the intervals and then executing countably many countable unions or intersections.

\begin{thm}[Carath\'eodory's extension theorem]
  \label{thm:caratheodory}
  The measure $m$ defined above satisfies conditions (a) and (b), and additionally satisfies condition (c) when applied to Borel sets.
\end{thm}

\begin{proof}[Proof of conditions (a) and (b) only]
  For condition (a), let $I$ be an interval. It is clear that $m(I)\leq\ell(I)$, since $I$ itself is an interval covering $I$. For the other direction, we will show that $I\subset\bigcup I_n$ implies $\sum\ell(I_n)\geq\ell(I)$. Then taking the infemum over all such coverings this allows us to conclude that $m(I)\geq\ell(I)$.

  Let us first handle the case when $I=[a,b]$ is closed and bounded, and all of the $I_n$ are open intervals $(a_n,b_n)$. Then since $I$ is compact, $I$ is covered by just finitely many of the $I_n$, without loss of generality we may assume they are indexed $I_0,\ldots,I_N$. Now after further renaming or removing intervals from the list, we may suppose that $I_0$ covers the left endpoint of $I$, each $I_{n+1}$ covers the right endpoint of $I_n$, and $I_N$ covers the right endpoint of $I$. We can now compute that
\[\sum\ell(I_n)
\geq\sum_0^N(b_n-a_n)
\geq\sum_1^N(b_n-b_{n-1})-a
\geq b-a
= \ell(I)\;.
\]

  If $I$ is not necessarily closed and the $I_n$ are not necessarily open, then we look instead at $I'=\cl(I)$ and $I_n'=(I_n)^\circ$. Then the $I_n'$ cover all but a countable subset of $I'$, so for any $\epsilon$ we can find additional open intervals $J_n$ with $\ell(J_n)\leq\epsilon/2^n$ covering these missing points. Now the previous argument shows that $\sum\ell(I_n)+\sum\ell(J_n)\geq\ell(I)$. Using the geometric series formula we obtain $\sum\ell(I_n)+\epsilon\geq\ell(I)$. Letting $\epsilon\to0$ we have the desired result.

  Finally if $I$ is not bounded, we let $A_k$ be a sequence of bounded subintervals of $I$ such that $\ell(A_k)\to\infty$. The above result implies that $\sum\ell(I_n)\geq\ell(A_k)$, and letting $k\to\infty$ we once again achieve the desired result. This concludes the proof of condition (a).

  Condition (b) follows directly from the fact that the length function $\ell$ is translation invariant, and the definition of $m$ depends only on $\ell$.

  For the proof that $m$ satisfies condition (c), we refer the reader to any standard measure theory text.
\end{proof}

From now on we will ignore the full power of measure theory to assign a real number measure to any Borel set, and focus only on the specific value zero. Sets whose measure is zero are called null sets, and for convenience we extract the definition of null set from the above definition of arbitrary measure.

\begin{defn}
  \label{defn:null}
  A set of real numbers $A$ is \emph{null} if for all $\epsilon>0$ there exists a sequence of intervals $(I_n)_{n\in\NN}$ such that $A\subset\bigcup I_n$ and $\sum\ell(I_n)<\epsilon$.
\end{defn}

\begin{example}
  The Cantor set $C$ is null. Indeed, we have already computed that the sum of the lengths of all intervals removed from $[0,1]$ in the construction of $C$ is equal to $1$. Since $[0,1]$ has measure $1$, it follows from additivity that $C$ must have measure $0$.
\end{example}

The notion of null set bears many similarities with the notion of meager set. As was the case with meager sets, the notion of a null set allows us to assign to any given set $A$ one of three simple ``sizes'': null, conull, or non-null and non-conull. Moreover, null sets satisfy an analog of the Baire category theorem and the preservation properties of meager sets.

\begin{cor}
  \label{cor:interval-non-null}
  The set $\RR$ of all real numbers is not null. In fact, if $I$ is any positive-length interval then $I$ is not null.
\end{cor}

\begin{cor}
  \label{cor:null-pres}
  The class of null sets is closed under the operations of subset and countable union.
\end{cor}

Corollary~\ref{cor:interval-non-null} is immediate from condition (a) of Theorem~\ref{thm:caratheodory}. Corollary~\ref{cor:null-pres} is immediate from condition (c) of Theorem~\ref{thm:caratheodory}.

\begin{exerc}
  Prove that condition (c) of a measure implies \emph{monotonicity}: if $A\subset B$ then $m(A)\leq m(B)$.
\end{exerc}

\begin{exerc}
  Prove that condition (c) of a measure implies \emph{continuity from above}: if $A_n$ is a decreasing sequence of sets, $m(A_n)$ is finite, and $A=\bigcap A_n$, then $m(A)=\inf m(A_n)$.
\end{exerc}

\begin{exerc}
  Give a proof directly from Definition~\ref{defn:null} that the Cantor set is null.
\end{exerc}

\begin{exerc}
  Give a proof directly from Definition ~\ref{defn:null} that the class of null sets is closed under countable union.
\end{exerc}

%%%%%%%%%%%%%%%%%%%%%%%%%%%%%%%%%%%%%%%%%%%%%%%%%%
\section{Sets, functions, and cardinality}
%%%%%%%%%%%%%%%%%%%%%%%%%%%%%%%%%%%%%%%%%%%%%%%%%%

So far we have seen sets that are finite, countable, and uncountable. If a set $A$ is finite, then there is a natural number that tells us exactly how many elements $A$ has. If $A$ is countable, we understand that it has exactly as many elements as there are natural numbers. But if $A$ is uncountable, is that all that needs be said or is there some kind of number that tells us just how uncountable it is?

In this section we discuss the notion of ``cardinality'' of a set $A$, which replaces the notion of ``number of elements'' in the case when $A$ is infinite. Notationally, we write $|A|$ for the cardinality of $A$. As we shall see, when $A$ is finite $|A|$ will be a natural number. When $A$ is countable $|A|$ will take the value $\aleph_0$ (called aleph zero, or aleph nought). And when $A$ is uncountable $|A|$ will take one of the values $\aleph_1,\aleph_2$ and so forth.

Our first result confirms that there are many different uncountable cardinalities. Recall that if $A$ is a set, then the \emph{power set} of $A$, denoted $\mathcal P(A)$, is the set of all subsets of $A$.

\begin{thm}[Cantor]
  If $A$ is any set, then $|A|<\mathcal P(A)$.
\end{thm}

\begin{proof}

\end{proof}

\begin{defn}
  Let $A$ be a set (or class). A \emph{total ordering} of $A$ is a binary relation $\leq$ which is antisymmetric, transitive, and total. A \emph{well-ordering} of $A$ is a total ordering $\leq$ such that every subset $B\subset A$ has a $\leq$-least element $b_0$.
\end{defn}

The following result shows that the cardinalities are totally ordered.

\begin{thm}
  If $A$ and $B$ are sets, then either there is an injection from $A$ to $B$ or else there is an injection from $B$ to $A$.
\end{thm}

\begin{proof}

\end{proof}

The next result shows that the cardinalities (well, any set of cardinalities) is well-ordered.

\begin{thm}
  If $(A_\alpha)_{\alpha\in I}$ is any collection of sets indexed by a set $I$, then there exists $\alpha_0\in I$ such that $|A_{\alpha_0}|\leq|A_\alpha|$ for all $\alpha\in I$.
\end{thm}

The following result gives a simple recipe for showing two sets have the same cardinality.

\begin{thm}[Cantor--Schr\"oder--Bernstein]
  If there are injections $i\colon A\to B$ and $j\colon B\to A$ then there is a bijection $f\colon A\to B$.
\end{thm}

\begin{proof}

\end{proof}

We conclude by briefly diving further into set theory and giving a more rigorous definition of the $\aleph$ numbers.

%%%%%%%%%%%%%%%%%%%%%%%%%%%%%%%%%%%%%%%%%%%%%%%%%%
\section{The topology of Cantor and Baire space}
%%%%%%%%%%%%%%%%%%%%%%%%%%%%%%%%%%%%%%%%%%%%%%%%%%



\end{document}
