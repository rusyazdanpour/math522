\documentclass[11pt,oneside]{amsbook}

\title[Category, measure, and forcing]{Category, measure, and forcing\\Math 522 lecture notes\\Fall, 2014}
\author{Samuel Coskey}
\author{Erik Holmes}

\usepackage[vscale=.8,vmarginratio=4:3]{geometry}
\usepackage{mathpazo}
\usepackage{amssymb}
\usepackage{setspace}\onehalfspacing\raggedbottom
\renewcommand{\labelitemi}{$\circ$}
\renewcommand{\labelenumi}{(\alph{enumi})}
\usepackage{etoolbox}\makeatletter\pretocmd{\@seccntformat}{\S}{}{}\makeatother
\usepackage{tikz}\usetikzlibrary{positioning,decorations.fractals,decorations.pathmorphing}\usepackage{hyperref}

\newcommand{\set}[1]{\left\{\,#1\,\right\}}
\newcommand{\NN}{\mathbb N}
\newcommand{\PP}{\mathbb P}
\newcommand{\QQ}{\mathbb Q}
\newcommand{\RR}{\mathbb R}
\newcommand{\Null}{\mathcal N}
\newcommand{\Meager}{\mathcal M}
\newcommand{\Ksigma}{\mathcal K_\sigma}
\DeclareMathOperator{\add}{\mathsf{add}}
\DeclareMathOperator{\non}{\mathsf{non}}
\DeclareMathOperator{\cov}{\mathsf{cov}}
\DeclareMathOperator{\cof}{\mathsf{cof}}
\DeclareMathOperator{\Diff}{Diff}
\DeclareMathOperator{\dom}{dom}
\DeclareMathOperator{\cod}{cod}
\DeclareMathOperator{\rng}{rng}
\DeclareMathOperator{\cl}{cl}
\DeclareMathOperator{\Fn}{Fn}
\DeclareMathOperator{\Con}{Con}

\theoremstyle{definition}
\newtheorem{exerc}{Exercise}[section]
\swapnumbers
\theoremstyle{plain}
\newtheorem{thm}{Theorem}[section]
\newtheorem{cor}[thm]{Corollary}
\newtheorem{lem}[thm]{Lemma}
\newtheorem{prop}[thm]{Proposition}
\theoremstyle{definition}
\newtheorem{defn}[thm]{Definition}
\theoremstyle{remark}
\newtheorem{rem}[thm]{Remark}
\newtheorem{example}[thm]{Example}
\newtheorem*{notes}{Notes and further reading}

\begin{document}

\frontmatter

\maketitle

\tableofcontents

\mainmatter

\setcounter{page}{4}

%%%%%%%%%%%%%%%%%%%%%%%%%%%%%%%%%%%%%%%%%%%%%%%%%%
\chapter*{Introduction}
%%%%%%%%%%%%%%%%%%%%%%%%%%%%%%%%%%%%%%%%%%%%%%%%%%

This course is about set theory, and its use in the study of the real line. By way of motivation, consider the question of measuring the \emph{size} of a given set of real numbers. The word ``size'' can mean many different things, depending on the context. To see what I mean, consider these three important examples.

\begin{itemize}
\item To a set theorist or combinatorist, the simplest notion of size is
\emph{cardinality}, which asks for instance whether a set is finite, countably infinite, or uncountable. Of course if the set is uncountable, a set theorist would further hope to distinguish whether it is of the first uncountable cardinality ($\aleph_1$), the second uncountable cardinality ($\aleph_2$), etc.

\item To an analyst, one standard notion of size is \emph{measure}, which asks for instance whether a set has zero length or positive length.

\item To a topologist, a key notion of size is \emph{category}, which asks whether a set is meager (topologically negligable), comeager, or else neither meager nor comeager.
\end{itemize}

In this course we will ask the natural question: How do these three kinds of size compare with one another? In answering this question, we will discover numerous relationships between measure and category, with the diagram of these relationships revealing a beautiful structure.

Our investigation of measure and category will lead us to a number of statements that are \emph{independent} of the axioms of set theory. This means that such statements cannot be proved true or false using the usual axioms of set theory. The most central example of such a statement is the \emph{continuum hypothesis} (CH), which asserts that the set of all real numbers has cardinality equal to the first uncountable cardinal number. Cantor initially posed this problem in 1878. In 1940 G\"odel showed the axioms of set theory can't disprove CH, and in 1965 Cohen finally showed that the axioms of set theory can't prove CH.

In his proof of the latter half of the independence of CH, Cohen developed a tool for building models of set theory called \emph{forcing}. Far from being an isolated application, since then forcing has become a standard tool for establishing independence results. We will develop the machinery of forcing and give a number of such applications.

With the tool of forcing in hand, we will return to the notions of measure and category. We will find that there are numerous cardinal numbers surrounding the zero length sets and the meager sets whose values, like the size of the set of all real numbers, are also independent of the axioms of set theory.


\chapter*{Part I: Category and measure}

%%%%%%%%%%%%%%%%%%%%%%%%%%%%%%%%%%%%%%%%%%%%%%%%%%
\section{The real number system}
%%%%%%%%%%%%%%%%%%%%%%%%%%%%%%%%%%%%%%%%%%%%%%%%%%

\begin{defn}
  The \emph{real number system} is a complete ordered field. That is, it is a set $\RR$ together with distinguished elements $0,1$, operations $+,\times$ and an ordering $<$ satisfying the following rules.
\begin{itemize}
\item It is a \emph{field}: it satisfies the associative, commutative, and distributive laws, $0$ is the additive identity and $1$ is the multiplicative identity;
\item It is an \emph{ordered} field: $<$ is a strict total order, $a<b$ implies $a+c<b+c$, and if $c$ is positive then $a<b$ implies $ac<bc$;
\item It is \emph{complete}: if $A\subset\RR$ has an upper bound, then it has a least upper bound (supremum).
\end{itemize}
\end{defn}

\begin{rem}
  It is not difficult to prove that any two complete ordered fields are isomorphic to each other, so it makes sense to define the real number system as the unique such object.
\end{rem}

\begin{rem}
  The above definition of the real number system is an abstract, axiomatic definition. That is, it doesn't tell you how real numbers behave, but it leaves it up to your imagination what real numbers really are. In fact, we have some choice about this. For example we can view the real numbers as decimal strings or binary strings. For our set-theoretic purposes, all that matters is that we give one construction of the real numbers that works.
\end{rem}

The first to give an explicit and rigorous construction of the real number system was Dedekind, who did so in the 19th century. Here is how his construction worked. First we will take the construction of the rational number system for granted. It is simply the set of fractions whose numerator is an integer and whose denominator is a nonzero natural number.

The key insight is to realize that each real number $r$ has a \emph{cut} in the rational numbers associated with it: the set of rational numbers that are strictly less than $r$. This association should be bijective: If the least upper bound property is true, then any such cut gives rise to a real number as the supremum of the cut. And if two real numbers are different, then they should give rise to different cuts.

So all we have to do to define real numbers was to define cuts. To avoid circularity, we should axiomatize cuts without referring to real numbers at all. A \emph{cut} in the rational numbers is a subset $r\subset\QQ$ satisfying three properties:
\begin{itemize}
\item nontriviality: $r\neq\emptyset$ and $r\neq\QQ$
\item downward closure: if $q\in r$ and $q'<q$ then $q'\in r$
\item no last element: if $q\in r$ then there is $q'>q$ such that $q'\in r$
\end{itemize}
To complete the construction one must say how the $+,\times$ operations and $<$ relation can be defined just in terms of cuts, and then prove that they satisfy the axioms of a complete ordered field. For example, one defines $r+r'$ to be the supremum of the set $\set{q+q'\mid q\in r,q'\in r'}$, and one defines $r<r'$ if and only if $r\subset r'$ and $r\neq r'$. We omit the details of verifying that these definitions work.

\begin{rem}
  It is interesting to compare Dedekind's construction with the modern decimal number construction (which we omit). The decimal number construction is practical for calculations, but has some oddities. For example some numbers have two decimal representations, like $.999\cdots$ and $1$. Also we made an arbitrary decision to use decimal expansions as opposed to any other base. Dedekind's construction is beautiful because it is completely uniform and does not involve any arbitrary decisions.
\end{rem}

\begin{defn}
  A set $A$ of real numbers is said to be \emph{countable} if its elements can be enumerated using natural number indices. In other words, $A$ is countable if there exists a sequence $(a_n)_{n\in\NN}$ such that $A=\set{a_n\mid n\in\NN}$.
\end{defn}

Note that this definition implies that finite sets are countable. If we wish to emphasize that a particular countable set is infinite, we will use the term \emph{countably infinite}. A set that is not countable is called \emph{uncountable}. We now give Cantor's famous proof of the fact that the set of all real numbers is uncountable. The argument goes by contradiction and uses a recursive construction to reach a contradiction one step at a time. Such arguments are often called ``diagonal'', for reasons we shall see later on.

\begin{thm}[Cantor]
  \label{thm:cantor}
  The set of all real numbers is uncountable.
\end{thm}

\begin{proof}
  Suppose towards a contradiction that the set of all real numbers is countable. Then there exists a sequence $(a_n)_{n\in\NN}$ such that $\RR=\set{a_n\mid n\in\NN}$. Our strategy will be to inductively construct a decreasing sequence of closed intervals $I_n=[l_n,r_r]$ such that for all $n$, $a_n\notin I_n$. Then if $r$ lies in the intersection of these intervals, $r$ cannot be equal to $a_n$ for any $n$, a contradiction.

The construction itself is straightforward. For the base case let $I_1=[l_1,r_1]$ be any interval which omits $a_1$. For the inductive step, if $I_{n-1}$ has been defined, let $I_n=[l_n,r_n]$ be any subinterval of $I_{n-1}$ which omits $a_n$.

We can now find a point in the intersection of the $I_n$ using the completeness property. First observe that since $I_n\subset I_{n-1}$ for all $n$, the set of left endpoints $\set{l_n\mid n\in\NN}$ is bounded above by any and all of the right endpoints $r_n$. By the completeness property, the set of left endpoints has a least upper bound $x$. Since $x$ is an upper bound for $\set{l_n\mid n\in\NN}$, we have $l_n\leq x$ for all $n$. Since $x$ is the least possible upper bound for $\set{l_n\mid n\in\NN}$, we have $x\leq r_n$ for all $n$. Therefore we have $x\in[l_n,r_n]=I_n$ for all $n$.

Since $x$ lies in $I_n$ for all $n$, and since we chose $I_n$ in such a way that it omits $a_n$, we know that $x\neq a_n$ for all $n$. This contradicts the hypothesis that $\RR=\set{a_n\mid n\in\NN}$, and concludes the proof.
\end{proof}

This argument appears in one form or another numerous times throughout elementary analysis and set theory. We shall next see it in the proof of the Baire category theorem.

\begin{exerc}
  With the definition of $r+r'$ for Dedekind cuts, show that addition is commutative and associative.
\end{exerc}

\begin{exerc}
  Define multiplication $r\times r'$ for Dedekind cuts, and show that it agrees with multiplication for rational numbers. [Hint: consider four cases when $r,r'$ are negative or nonnegative.]
\end{exerc}

\begin{exerc}
  Let $A$ be a set of real numbers containing a positive-length interval. Show that $A$ is uncountable.
\end{exerc}

\begin{notes}
  Although the material in this section is standard and can be located in most any analysis book, an excellent introduction is \emph{Understanding Analysis} by Stephen Abbott. For the completeness axiom see Abbott, Section~1.3. For Cantor's theorem see Abbott, Section~1.4. For Dedekind cuts see Abbott, Section~8.4.
\end{notes}

%%%%%%%%%%%%%%%%%%%%%%%%%%%%%%%%%%%%%%%%%%%%%%%%%% 
\section{Baire category theory}
%%%%%%%%%%%%%%%%%%%%%%%%%%%%%%%%%%%%%%%%%%%%%%%%%%

\begin{defn}
  A set of real numbers $A$ is said to be \emph{nowhere dense} if every positive-length interval $I$ contains a positive-length interval $J$ such that $J$ is disjoint from $A$.
\end{defn}

\begin{example}
  Any finite subset of $\RR$ is nowhere dense. The set $\set{1/n\mid n\in\NN}$ is nowhere dense. In fact, any \emph{discrete} set is nowhere dense (here, $A$ is discrete if every point of $A$ is isolated). The set of rational numbers whose numerator is $1$ and whose denominator is a power of $2$ is \emph{not} nowhere dense, because such a number can be found in every positive-length subinterval of $[0,1]$.
\end{example}

Note that nowhere dense sets need not be countable. The classical Cantor middle thirds set is an archetypal example of a nowhere dense set. Recall that the \emph{Cantor set} is defined as follows. Begin with the set $C_1=[0,1]$. Let $C_2=[0,1/3]\cup[2/3,1]$ be the set $C_1$ with its open middle third removed. Let $C_3=[1,1/9]\cup[2/9,1/3]\cup[2/3,7/9]\cup[8/9,1]$ be the set $C_2$ with the open middle third removed from each component of $C_2$. Refer to Figure~\ref{fig:cantor-set} for a picture of these first few sets. Recursively, let $C_{n+1}$ be constructed from $C_n$ by removing the open middle third from each component of $C_n$. Finally the Cantor set $C$ is defined to be $C=\bigcap C_n$. We will see in a later section that the Cantor set is uncountable. % add ref

\begin{figure}[h]
\begin{center}
  \begin{tikzpicture}
    \draw[|-|] (0,0) -- (10,0) node[anchor=west] {\ \ $C_1$};
    \draw[|-|] (0,-.5) -- (10/3,-.5);
    \draw[|-|] (20/3,-.5) -- (10,-.5) node[anchor=west] {\ \ $C_2$};
    \draw[|-|] (0,-1) -- (10/9,-1);
    \draw[|-|] (20/9,-1) -- (30/9,-1);
    \draw[|-|] (60/9,-1) -- (70/9,-1);
    \draw[|-|] (80/9,-1) -- (10,-1) node[anchor=west] {\ \ $C_3$};
    \node[anchor=west] at (10,-1.4) {\ \ \ $\vdots$};
    \draw[decoration=Cantor set,very thick]
    decorate{ decorate{ decorate{ decorate{ decorate{ decorate{
                (0,-2) -- (10,-2) }}}}}} node[anchor=west] {\ \ $C$};
  \end{tikzpicture}
\end{center}
\caption{The first few steps in the construction of the Cantor set, and a rough image of the Cantor set itself.\label{fig:cantor-set}}
\end{figure}

\begin{prop}
  The Cantor set $C$ is nowhere dense.
\end{prop}

\begin{proof}
  First compute that the sum of the lengths of all of those middle thirds removed in the construction of the cantor set is exactly $1$ (Exercise~\ref{exerc:cantor}). It follows from this that $C$ does not contain any positive-length intervals. Now if $I$ is any positive-length interval, then $I$ must contain a point $x\notin C$. Now note that $C$ is closed because it is an intersection of closed sets, and hence $\RR\smallsetminus C$ is open. Thus there is an open interval $J$ centered at $x$ such that $J\subset\RR\smallsetminus C$. Shrinking the radius of $J$ if necessary, we can suppose that $J\subset I$. This shows that $C$ is nowhere dense.
\end{proof}

The terminology ``nowhere dense'' may sound strange at first, but it is justified by the following fact.

\begin{prop}
  \label{prop:nwd-equiv}
  Let $A$ be a set of real numbers. The following are equivalent.
\begin{enumerate}
\item $A$ is nowhere dense;
\item $\cl(A)$ (the topological closure of $A$) has no interior;
\item for every nonempty open set $O$, we have that $A$ is not dense in $O$.
\end{enumerate}
\end{prop}

The proof is requested in Exercise~\ref{exerc:nwd-equiv}.

\begin{thm}
  The class of nowhere dense sets is closed under the operations of subset, union, and closure.
\end{thm}

\begin{proof}
  Preservation to subsets is immediate from the definition.

  For preservation to unions, suppose that $A,B$ are nowhere dense and let $I$ be a positive-length interval. Since $A$ is nowhere dense, there is a positive-length interval $J\subset I\smallsetminus A$. Since $B$ is nowhere dense there is a further positive-length interval $J'\subset J\smallsetminus B$. It follows that
\[J'\subset (I\smallsetminus A)\smallsetminus B=I\smallsetminus (A\cup B)
\]
This establishes that $A\cup B$ is again nowhere dense.

Preservation to closures is immediate from condition (b) of Proposition~\ref{prop:nwd-equiv}, but we can also give a direct argument. Let $A$ be nowhere dense and let $I$ be a given positive-length interval. Since $A$ is nowhere dense there is a positive-length interval $J$ such that $J\subset\RR\smallsetminus A$. Removing endpoints if necessary, we may suppose that $J$ is open. It follows that $J\subset\RR\smallsetminus\cl(A)$ (recall that since $\cl(A)$ is the intersection of all closed sets containing $A$, we have that $\RR\smallsetminus\cl(A)$ is the union of all open sets disjoint from $A$). This shows that $\cl(A)$ is nowhere dense.
\end{proof}

\begin{rem}
  It is easy to see that the class of nowhere dense sets is \emph{not} preserved to infinite unions, even countably infinite ones. For example, the set of rational numbers is countable, and therefore it is a countable unions of singletons: $\QQ=\bigcup_{q\in\QQ}\{q\}$. Each singleton $\{q\}$ is clearly nowhere dense, so $\QQ$ is a countable union of nowhere dense sets, but $\QQ$ is dense.
\end{rem}

\begin{defn}
  A set of real numbers $A$ is called \emph{meager} if it is a union of countably many nowhere dense sets. A set $A$ is called \emph{comeager} if $\RR\smallsetminus A$ is meager.
\end{defn}

In principal it is possible that this notion is completely moot in the sense that maybe every set of real numbers is meager according to this definition. The Baire category theorem says that this is not the case. In fact, we will soon see that meager sets are rather paltry as the name implies.

\begin{thm}[Baire category theorem]
  \label{thm:baire}
  The set of all real numbers $\RR$ is nonmeager. In fact, if $I$ is a positive-length interval then $I$ is nonmeager.
\end{thm}

\begin{proof}
  The proof is very similar to Cantor's proof of Theorem~\ref{thm:cantor}. Let $I$ be a positive-length interval and suppose towards a contradiction that $I$ is meager. Then there exists a sequence $(A_n)_{n\in\NN}$ of nowhere dense sets such that $\bigcup A_n=I$. We will inductively construct a decreasing sequence of closed subintervals $J_n\subset I$ such that for all $n$, $J_n\cap N_n=\emptyset$.

  The induction is again straightforward. First apply the definition of $A_1$ being nowhere dense to obtain $J_1$. Inductively apply the definition of $A_n$ being nowhere dense to $J_n$ to obtain $J_{n+1}$. (Each time we may shrink the newly obtained interval slightly to suppose that it is closed.)

  We can now argue just as in the proof of Theorem~\ref{thm:cantor} that there exists a point $x\in\bigcap J_n$. It follows that $x\notin\bigcup A_n$, which contradicts the assumption that $I=\bigcup A_n$, and completes the proof.
\end{proof}

\begin{rem}
  Baire category theory gets its name from the above theorem, and the fact that meager sets used to be called ``first category''. Nonmeager sets used to be called ``second category''.
\end{rem}

\begin{thm}
  \label{thm:meager-pres}
  The class of meager sets is closed under the operations of subset and countable union.
\end{thm}

We remark that the meager property is not necessarily preserved to the closure. In fact $\cl(A)$ is meager if and only if $A$ is nowhere dense.

As a consequence of the Baire category theorem, for any set $A$ it is possible to assign $A$ one of three distinct ``sizes'':
\begin{itemize}
\item $A$ is meager (small)
\item $A$ is comeager (large)
\item $A$ neither meager nor comeager (intermediate)
\end{itemize}
We should verify that no set of numbers can be both meager and comeager. Indeed, suppose that both $A$ and $\RR\smallsetminus A$ were meager. Then by Theorem~\ref{thm:meager-pres} their union $\RR$ would be meager, contradicting the Baire category theorem.

We should also verify that there exists sets of numbers that are neither meager nor comeager. Indeed, if $I$ is any bounded positive-length interval then by the Baire category theorem $I$ is nonmeager, and since $\RR\smallsetminus I$ also contains a bounded interval, $I$ is also non-comeager.

\begin{exerc}
  \label{exerc:cantor}
  Compute the sum of the lengths of all of the intervals removed from $[0,1]$ in the construction of the Cantor set. What if some other fraction is removed at each stage?
\end{exerc}

\begin{exerc}
  \label{exerc:nwd-equiv}
  Prove Proposition~\ref{prop:nwd-equiv}.
\end{exerc}

\begin{exerc}
  We have observed that unlike the nowhere dense property, the meager property is not necessarily preserved to the closure. Prove that if $\cl(A)$ is meager, then $A$ was nowhere dense to begin with.
\end{exerc}

\begin{notes}
  For more on the Cantor set see Abbott, Section~3.1. For the Baire category theorem see Abbott, Section~3.5. For Baire category theory generally see Oxtoby, Chapter~1.
\end{notes}

%%%%%%%%%%%%%%%%%%%%%%%%%%%%%%%%%%%%%%%%%%%%%%%%%%
\section{Lebesgue measure theory}
%%%%%%%%%%%%%%%%%%%%%%%%%%%%%%%%%%%%%%%%%%%%%%%%%%

Classical measure theory aims to try to extend the length function on intervals to be defined on more complicated sets. For example, if a given set is a finite or even countable union of intervals, then it is appropriate to take the sum of the lengths of the components. But what about more unwieldy sets like the Cantor set? This was known as the \emph{measure problem}: to construct a measurement function $m$, defined on sets of real numbers and valued in $[0,\infty]$, satisfying:
\begin{enumerate}
\item $m$ agrees with length: if $I$ is an interval then $m(I)=\ell(I)$
\item translation invariance: $m(x+A)=m(A)$ for all sets $A$ and real numbers $x$
\item countably additivity: if $(A_n)_{n\in\NN}$ is a sequence pairwise disjoint sets then $m(\bigcup A_n)=\sum m(A_n)$
\end{enumerate}

Perhaps surprisingly, the conditions (a)--(c) are mutually contradictory and so no such measurement function $m$ exists! Here is Vitali's simple example of a contradiction arising from these requirements. Regard $\QQ$ as an additive subgroup of $\RR$ and consider the cosets of $\QQ$, that is, sets of the form $x+\QQ$. Let $A\subset[0,1]$ be a set of coset representatives, that is, $A$ contains exactly one element from each of the cosets. (It is always possible to choose a coset representative in $[0,1]$ because $\QQ$ is dense.)

Then the translations $q+A$ for $q\in\QQ$ form a countable sequence of pairwise disjoint sets that cover all of $\RR$. In fact, if we let $S=\QQ\cap[-1,1]$ then the translations $q+A$ for $q\in S$ already cover all of $[0,1]$. We can then infer from (a) and (c) that $m[\bigcup_{q\in S}(q+A)]$ lies between $1$ and $4$. But on the other hand, by (b) and (c) we have that 
\[m\left(\bigcup_{q\in S}(q+A)\right)=\sum_{q\in S}m(q+A)=\sum_{q\in S}m(A)
\]
This is a contradiction because an infinite sum of a single constant value $m(A)$ can be equal only to either $0$ or $\infty$, and so cannot lie between $1$ and $4$.

\begin{rem}
  We observe that the axiom of choice was explicitly used in Vitali's construction of the set $A$ above. In fact, it is known that the use of the axiom of choice is necessary to build a counterexample.
\end{rem}

The contradiction described above is typically resolved not by dropping one of the conditions (a)--(c), but rather by dropping the requirement that $m$ is defined on \emph{all} sets of real numbers. The justification for this decision is that sets like the $A$ constructed above should be regarded as pathological, and we don't usually need to measure them in applications.

Let us na\"ively begin to construct a measure $m$ that is at least defined on reasonable sets. First, condition (a) implies we should let $m(I)=\ell(I)$ for any interval $I$. Next, if $A=\bigcup I_n$ is a union of disjoint intervals $I_n$, then condition (c) implies we should let $m(A)=\sum\ell(I_n)$. Third, if $A=\bigcap A_n$ is an intersection of sets where $m$ is defined and finite, then it is natural to define $m(A)=\inf m(A_n)$. We now observe that all three of the above simple cases fall under the following formula:
\[m(A)=\inf\set{\sum\ell(I_n)\mid A\subset\bigcup I_n}
\]
The next result states that this rule for defining $m$ actually works for all sets that are reasonably constructible. Recall that a \emph{Borel set} is one that can be constructed by beginning with the intervals and then executing countably many countable unions or intersections.

\begin{thm}[Carath\'eodory's extension theorem]
  \label{thm:caratheodory}
  The measure $m$ defined above satisfies conditions (a) and (b), and additionally satisfies condition (c) when applied to Borel sets.
\end{thm}

\begin{proof}[Proof of conditions (a) and (b) only]
  For condition (a), let $I$ be an interval. It is clear that $m(I)\leq\ell(I)$, since $I$ itself is an interval covering $I$. For the other direction, we will show that $I\subset\bigcup I_n$ implies $\sum\ell(I_n)\geq\ell(I)$. Then taking the infemum over all such coverings this allows us to conclude that $m(I)\geq\ell(I)$.

  Let us first handle the case when $I=[a,b]$ is closed and bounded, and all of the $I_n$ are open intervals $(a_n,b_n)$. Then since $I$ is compact, $I$ is covered by just finitely many of the $I_n$, without loss of generality we may assume they are indexed $I_0,\ldots,I_N$. Now after further renaming or removing intervals from the list, we may suppose that $I_0$ covers the left endpoint of $I$, each $I_{n+1}$ covers the right endpoint of $I_n$, and $I_N$ covers the right endpoint of $I$. We can now compute that
\[\sum\ell(I_n)
\geq\sum_0^N(b_n-a_n)
\geq\sum_1^N(b_n-b_{n-1})-a
\geq b-a
= \ell(I)\;.
\]

  If $I$ is not necessarily closed and the $I_n$ are not necessarily open, then we look instead at $I'=\cl(I)$ and $I_n'=(I_n)^\circ$. Then the $I_n'$ cover all but a countable subset of $I'$, so for any $\epsilon$ we can find additional open intervals $J_n$ with $\ell(J_n)\leq\epsilon/2^n$ covering these missing points. Now the previous argument shows that $\sum\ell(I_n)+\sum\ell(J_n)\geq\ell(I)$. Using the geometric series formula we obtain $\sum\ell(I_n)+\epsilon\geq\ell(I)$. Letting $\epsilon\to0$ we have the desired result.

  Finally if $I$ is not bounded, we let $A_k$ be a sequence of bounded subintervals of $I$ such that $\ell(A_k)\to\infty$. The above result implies that $\sum\ell(I_n)\geq\ell(A_k)$, and letting $k\to\infty$ we once again achieve the desired result. This concludes the proof of condition (a).

  Condition (b) follows directly from the fact that the length function $\ell$ is translation invariant, and the definition of $m$ depends only on $\ell$.

  For the proof that $m$ satisfies condition (c), we refer the reader to any standard measure theory text.
\end{proof}

From now on we will ignore the full power of measure theory to assign a real number measure to any Borel set, and focus only on the specific value zero. Sets whose measure is zero are called null sets, and for convenience we extract the definition of null set from the above definition of arbitrary measure.

\begin{defn}
  \label{defn:null}
  A set of real numbers $A$ is \emph{null} if for all $\epsilon>0$ there exists a sequence of intervals $(I_n)_{n\in\NN}$ such that $A\subset\bigcup I_n$ and $\sum\ell(I_n)<\epsilon$.
\end{defn}

\begin{example}
  The Cantor set $C$ is null. Indeed, we have already computed that the sum of the lengths of all intervals removed from $[0,1]$ in the construction of $C$ is equal to $1$. Since $[0,1]$ has measure $1$, it follows from additivity that $C$ must have measure $0$.
\end{example}

The notion of null set bears many similarities with the notion of meager set. As was the case with meager sets, the notion of a null set allows us to assign to any given set $A$ one of three simple ``sizes'': null, conull, or nonnull and non-conull. Moreover, null sets satisfy an analog of the Baire category theorem and the preservation properties of meager sets.

\begin{cor}
  \label{cor:interval-nonnull}
  The set $\RR$ of all real numbers is not null. In fact, if $I$ is any positive-length interval then $I$ is not null.
\end{cor}

\begin{cor}
  \label{cor:null-pres}
  The class of null sets is closed under the operations of subset and countable union.
\end{cor}

Corollary~\ref{cor:interval-nonnull} is immediate from condition (a) of Theorem~\ref{thm:caratheodory}. Corollary~\ref{cor:null-pres} is immediate from condition (c) of Theorem~\ref{thm:caratheodory}.

\begin{exerc}
  Prove that condition (c) of a measure implies \emph{monotonicity}: if $A\subset B$ then $m(A)\leq m(B)$.
\end{exerc}

\begin{exerc}
  \label{exerc:continuity-above}
  Prove that condition (c) of a measure implies \emph{continuity from above}: if $A_n$ is a decreasing sequence of sets, $m(A_n)$ is finite, and $A=\bigcap A_n$, then $m(A)=\inf m(A_n)$.
\end{exerc}

\begin{exerc}
  Give a proof directly from Definition~\ref{defn:null} that the Cantor set is null.
\end{exerc}

\begin{exerc}
  Give a proof directly from Definition ~\ref{defn:null} that the class of null sets is closed under countable union.
\end{exerc}

\begin{notes}
  Our presentation of Vitali's construction follows Folland, Section~1.1. Our proof of Theorem~\ref{thm:caratheodory}, part (a) follows Oxtoby, Chapter~1. For a proof of part (c), see Folland, Section~1.4. 
\end{notes}


%%%%%%%%%%%%%%%%%%%%%%%%%%%%%%%%%%%%%%%%%%%%%%%%%%
\section{Sets, orderings, and cardinality}
%%%%%%%%%%%%%%%%%%%%%%%%%%%%%%%%%%%%%%%%%%%%%%%%%%

So far we have seen sets that are finite, countable, and uncountable. If a set $A$ is finite, then there is a natural number that tells us exactly how many elements $A$ has. If $A$ is countable, we understand that it has exactly as many elements as there are natural numbers. But if $A$ is uncountable, is that all that needs be said or is there some kind of number that tells us just how uncountable it is?

In this section we discuss the notion of ``cardinality'' of a set $A$, which replaces the notion of ``number of elements'' in the case when $A$ is infinite. Notationally, we write $|A|$ for the cardinality of $A$. As we shall see, when $A$ is finite $|A|$ will be a natural number. When $A$ is countable $|A|$ will take the value $\aleph_0$ (called aleph zero, or aleph nought). And when $A$ is uncountable $|A|$ will take one of the values $\aleph_1,\aleph_2$ and so forth.

In the next section we will describe exactly what the $\aleph$'s are. But first it turns out that we can give many of the most important facts about cardinality without ever defining the cardinal numbers precisely.

\begin{defn}
  \label{defn:cardinal-rel}
  Let $A$ and $B$ be sets. We say that $|A|=|B|$ if there is a bijective function $f\colon A\to B$, we say that $|A|\leq|B|$ if there is an injective function $i\colon A\to B$, and we say that $|A|<|B|$ if $|A|\leq|B|$ and also $|A|\neq|B|$.
\end{defn}

\begin{rem}
  In the above definition, we managed to define all of the comparisons between cardinals without ever defining the cardinal itself. For most practical purposes, this definition is sufficient.
\end{rem}

Our first result confirms that there are many different uncountable cardinalities. Recall that if $A$ is a set, then the \emph{power set} of $A$, denoted $\mathcal P(A)$, is the set of all subsets of $A$.

\begin{thm}[Cantor]
  If $A$ is any set, then $|A|<|\mathcal P(A)|$.
\end{thm}

\begin{proof}
  It is clear that $|A|\leq|\mathcal P(A)|$, since the map $i(a)=\{a\}$ is an injection from $A$ into $\mathcal P(A)$.

  To see that there is no bijection between $A$ and $\mathcal P(A)$, let $f\colon A\to\mathcal P(A)$ be any function. Then build the set $D$ of all elements $a\in A$ such that $a\notin f(a)$. We claim that $D$ is not in the range of $f$, and therefore that $f$ is not a bijection.

  To see this, suppose towards a contradiction that there exists $a_0\in A$ such that $D=f(a_0)$. Then by the definition of $D$, we have that $a_0\in D$ iff $a_0\notin f(a_0)$. And since $D=f(a_0)$ we can write this as $a_0\in D$ iff $a_0\notin D$, which is a clear contradiction.
\end{proof}

\begin{rem}
  The classical argument above is given the name ``diagonal'' because of the key formula in the proof: $a\notin f(a)$. The idea is that if you were to shade the set of pairs $(x,y)\in A^2$ such that that $x\in f(y)$, then the set $D$ would be built by taking the unshaded elements of the diagonal of the $A^2$ plane.
\end{rem}

\begin{defn}
  Let $A$ be a set (or class).
  \begin{itemize}
  \item A \emph{partial ordering} of $A$ is a binary relation $\leq$ satisfying: (reflexive) $a\leq a$; (antisymmetric) $a\leq b\leq a\implies a=b$; (transitive) $a\leq b\leq c\implies a\leq c$.
  \item A \emph{total ordering} of $A$ is a partial ordering such that for all $a,a'$ either $a\leq a'$ or $a'\leq a$.
  \item A \emph{well-ordering} of $A$ is a total ordering $\leq$ such that every subset $B\subset A$ has a $\leq$-least element $b_0$.
  \end{itemize}
\end{defn}

\begin{rem}
  The well-order property may seem obscure at first, but finding a least element is precisely what is needed in induction arguments. It is what allows us to say ``otherwise, let $x$ be the least counterexample''.
\end{rem}

In the next few results, we essentially show that the cardinals are well-ordered by $\leq$. Note that reflexivity holds because the identity function is an injection from $A$ into itself, and transitivity holds because the composition of two injections is an injection. The following result establishes antisymmetry.

\begin{thm}[Cantor--Schr\"oder--Bernstein]
  \label{thm:csb}
  If there are injections $i\colon A\to B$ and $j\colon B\to A$ then there is a bijection $f\colon A\to B$.
\end{thm}

\begin{proof}
  Replacing $B$ with $j(B)$, we may suppose that that $B\subset A$. Then we have
  \[A\supset B\supset i(A)\supset i(B)\supset i^2(A)\supset\cdots
  \]
  Now $A$ can be written as the union of the successive differences of these sets, together with the intersection of them all:
  \[A=(A\mathord{\smallsetminus}B)\cup(B\mathord{\smallsetminus}i(A))
  \cup(i(A)\mathord{\smallsetminus}i(B))\cup\cdots\cup \bigcap i^n(A)
  \]
  Meanwhile, $i$ gives bijections
  \[A\mathord{\smallsetminus}B\leftrightarrow i(A)\mathord{\smallsetminus}i(B)\leftrightarrow i^2(A)\mathord{\smallsetminus}i^2(B)\leftrightarrow\cdots
  \]
  It follows that the map
  \[f(a)=\begin{cases}i(a)&\text{if }a\in(A\mathord{\smallsetminus}B)\cup(i(A)\mathord{\smallsetminus}i(B))\cup(i^2(A)\mathord{\smallsetminus}i^2(B))\cup\cdots\\
    a&\text{otherwise}
  \end{cases}
  \]
  is a bijection from $A$ to $B$.
\end{proof}

\begin{rem}
  This result also gives a simple recipe for checking whether two sets have the same cardinality. For example, to show that there is a bijection between $[0,1]$ and $(0,1)$ it is much easier to construct two injections than a single bijection.
\end{rem}

The next result shows that the cardinalities are totally ordered. In the proof we will need \emph{Zorn's lemma}: If $P$ is a partially ordered set such that every totally ordered subset has an upper bound, then $P$ has at least one maximal element.  Zorn's lemma is used perennially in analysis, and it is a consequence of the axiom of choice.

\begin{thm}
  \label{thm:card-total}
  If $A,B$ are sets then either there is an injective function from $A$ to $B$ or an injective function from $B$ to $A$.
\end{thm}

\begin{proof}
  Consider the family $\mathcal F$ of all injective functions whose domain is a subset of $A$ and whose range is a subset of $B$. Then $\mathcal F$ is partially ordered by extension of functions, and it is easy to check that this ordering satisfies the hypothesis of Zorn's lemma. Thus there is a maximal element $f$, and since it is maximal either the domain of $f$ is all of $A$ or the range of $f$ is all of $B$. In the first case $f$ is an injection from $A$ to $B$, and in the second case $f^{-1}$ is an injection from $B$ to $A$.
\end{proof}

Finally we show that the cardinals are well-ordered. One technicality arises here that did not in the last four properties. To check that the cardinals are well-ordered, we should check that \emph{any} collection of sets has a minimal element, and that collection need not itself be a set. (Remember: the set of all sets isn't a set!)

\begin{thm}
  \label{thm:card-well}
  Let $\mathcal A$ be a class of sets. Then there exists $A\in\mathcal A$ that injects into all other sets $B\in\mathcal A$.
\end{thm}

\begin{proof}
  We argue very similarly to Theorem~\ref{thm:card-total}, but in order to apply Zorn's lemma we must first suppose that $\mathcal A$ really is a set. Fix any element $A\in\mathcal A$ and let
\[\mathcal F=\set{(f_B)_{B\in\mathcal A}\mid\text{there is $A_0\subset A$ such that for all $B$, $f_B$ is an injection from $A_0$ to $B$}}
\]
This time $\mathcal F$ is partially ordered by \emph{coordinatewise} extension of functions. By Zorn's lemma there is a maximal element $(f_B)_{B\in\mathcal A}$, and since it is maximal either the domain $A_0$ is all of $A$ or the range of one of the functions $f_B$ is all of $B$. In the first case $f_B$ is an injection from $A$ to $B$ for all $B$. In the second case, fix $B$ such that $f_B(A_0)=B$. Then for any $C\in\mathcal A$, the composition $f_C\circ f_B^{-1}$ is an injection from $B$ to $C$, so $B$ is as desired.

In the general case when $\mathcal A$ is a class, we can reduce to the set case as follows. Fix an element $D\in\mathcal A$ and let $\mathcal A'$ denote the collection of subsets of $D$ that are in bijection with some element of $\mathcal A$. Since $\mathcal A'$ is a set, we can find $A\in\mathcal A$ which injects into all elements of $\mathcal A'$. It follows that $A$ injects into all other elements of $\mathcal A$ too. Indeed if $B\in\mathcal A$ and $A$ does not inject into $B$, then by Theorem~\ref{thm:card-total}, $B$ injects into $A$. It follows that $B$ is in bijection with a subset of $D$ and hence $A$ injects into $B$ after all.
\end{proof}

Although we still can't be fully rigorous about the meaning of the symbols $\aleph_1,\aleph_2,\ldots$, the well-ordering property helps justify the use of these symbols. Essentially $\aleph_1$ is the least cardinality greater than $\aleph_0$, $\aleph_2$ is the least cardinality greater than $\aleph_1$, and so forth. In the next section, we will make this fully precise.

\begin{exerc}
  Show that there is a bijection between $\mathcal P(\NN)$ and $\RR$.
\end{exerc}

\begin{exerc}
  Which of the following categories satisfy the analog of the Cantor--Schr\"oder--Bernstein theorem? (That is, monomorphisms $A\to B\to A$ implies isomorphism $A\cong B$.) linear orders with order-preserving maps; groups with group homomorphisms; topological spaces with continuous maps; topological spaces with piecewise continuous maps.
\end{exerc}

% why not show the set of countable ordinals is an ordinal
\begin{exerc}
  We used the well-ordering principle together with Cantor's theorem to argue that there are uncountable ordinals. Here is another way: show that the set of isomorphism equivalence classes of well-orderings of $\omega$ is itself naturally well-ordered, and that this well-ordering is uncountable.
\end{exerc}

\begin{notes}
  Our proof of Theorem~\ref{thm:card-well} follows and extends the proof given in Chaim Samuel H\"onig, \emph{Proof of the well-ordering of cardinal numbers}. For a more detailed proof of Theorem~\ref{thm:csb}, see Kunen (Foundations), Section~I.10.
\end{notes}


%%%%%%%%%%%%%%%%%%%%%%%%%%%%%%%%%%%%%%%%%%%%%%%%%% 
\section{Ordinal numbers and cardinal numbers}
%%%%%%%%%%%%%%%%%%%%%%%%%%%%%%%%%%%%%%%%%%%%%%%%%%

Ordinal numbers play a central role in set theory, including both cardinality theory and forcing. While cardinal numbers are needed to measure the \emph{size} of an infinite set, ordinal numbers are needed to measure the \emph{length} of an infinite well-ordered set. The ordinals can be used to extend the notion of \emph{counting} into the infinite, and also to give a formal definition of the $\aleph$ cardinals described in the previous section.

The initial goal in defining the ordinals is to provide a collection of well-ordered sets such that for any given well-ordered set $A$ there is one and only one ordinal $\alpha$ such that $A$ is isomorphic to $\alpha$. For finite well-ordered sets, we can write such a definition explicitly:
\begin{align*}
0&=\emptyset\\
1&=\{0\}\\
2&=\{0,1\}\\
\vdots\\
n+1&=\{0,\ldots,n\}
\end{align*}
This construction can be summed up in one recurrence: $n+1=n\cup\{n\}$. The first infinite ordinal, called $\omega$ or $\omega_0$, is simply the union of all the finite ordinals. The process can then continue; the successor of $\omega$ is the ordinal $\omega+1=\omega\cup\{\omega\}$.

Intuitively, infinite ordinals are all constructed in this fashion. If $\alpha$ is any ordinal then its successor is $\alpha+1=\alpha\cup\{\alpha\}$, and after infinitely many such steps we take a union. Unfortunately this prescription is not rigorous because it is circular; we cannot make a definition like $\alpha=\bigcup_{\beta<\alpha}\beta$. Instead we have the following more technical characterization.

\begin{defn}
  \label{defn:ordinals}
  A set $\alpha$ is an \emph{ordinal} if it satisfies the properties:
  \begin{itemize}
  \item $\alpha$ is well-ordered by the $\in$ relation (or more precisely by the relation $\leq$ defined as $\in$-or-equal);
  \item $\alpha$ is transitive: if $\gamma\in\beta\in\alpha$ then $\gamma\in\alpha$.
  \end{itemize}
\end{defn}

\begin{rem}
The first condition ensures that the ordinals are in fact well-ordered; the order relation $\in$ is simply the most convenient one available in set theory. The second condition ensures that the ordinals have no ``gaps''; for instance the set $\{0,1,3,5,9\}$ is well-ordered but not an ordinal.
\end{rem}

The following fundamental facts about ordinals together imply that Definition~\ref{defn:ordinals} achieves our initial goals in defining the ordinals.

\begin{thm}\
  \label{thm:ordinals}
  \begin{itemize}
  \item The class of all ordinals is itself transitive and well-ordered by $\leq$.
  \item If $A$ is a well-ordered set then there exists a unique ordinal $\alpha$ such that $A$ is isomorphic to $\alpha$.\qed
  \end{itemize}
\end{thm}

An ordinal $\alpha$ is said to be \emph{successor} ordinal if it is of the form $\beta+1$ for some ordinal $\beta$. Otherwise $\alpha$ is said to be \emph{limit} ordinal. The first part of Theorem~\ref{thm:ordinals} allows us to show that successor and limit are appropriate names.

\begin{prop}
  If $\alpha=\beta+1$ then $\alpha$ is the least ordinal greater than $\beta$. If $\alpha$ is a limit ordinal then $\alpha$ is the union of the ordinals that came before it.
\end{prop}

\begin{proof}
  It is easy to check $\alpha=\beta\cup\{\beta\}$ is indeed an ordinal (it is transitive and well-ordered by $\in$). Since the ordinals are well-ordered we may let $\alpha'$ be the least ordinal greater than $\beta$; we want to show that $\alpha=\alpha'$. If this were not the case, then by totality we would either have $\alpha'\in\alpha$ or $\alpha\in\alpha'$. In the first case either $\alpha=\beta$ or $\alpha\in\beta$, which contradicts that $\alpha'$ is greater than $\beta$. The second case contradicts that $\alpha'$ is the least such.

  Next let $\alpha$ be a limit ordinal, and let $\alpha'$ be the union of all $\beta\in\alpha$. Since $\alpha$ is transitive we easily have that $\alpha'\subset\alpha$. On the other hand if $\gamma\in\alpha$ then by the previous paragraph $\gamma+1\leq\alpha$ and since $\alpha$ is limit we must have $\gamma+1\in\alpha$. Thus $\gamma\in\gamma+1\in\alpha$ and it follows that $\gamma\in\alpha'$.
\end{proof}

As we have hinted, although ordinals naturally measure the length of well-orderings, they can also be used to measure size $|A|$. Recall that the axiom of choice implies the \emph{well-ordering principle}, which states that any set $A$ admits a well-ordering $\leq$. Combining this with Theorem~\ref{thm:ordinals}, it follows that any set $A$ is in bijection with at least one ordinal. Different well-orderings of $A$ can lead to different ordinals, so we make the following definition.

\begin{defn}
  If $A$ is any set, then $|A|$ is the least ordinal $\alpha$ such that $A$ is in bijection with $\alpha$.
\end{defn}

Thus a cardinal is a special type of ordinal. Most ordinals will not be cardinals, since for instance if $A$ is in bijection with $\omega+7$ then clearly it is also in bijection with $\omega$. We give $\aleph$ names to the ordinals which are cardinals: The first infinite cardinal is $\aleph_0$, the first uncountable cardinal is $\aleph_1$, the next least cardinal is $\aleph_2$, and so forth.

The pattern continues transfinitely as well, with $\aleph_\alpha$ defined for every ordinal $\alpha$. Officially, these higher cardinals are defined using \emph{transfinite recursion}. Just as natural numbers are used to index traditional recursion, ordinals are used to index transfinite recursion. While ordinary recursion requires a special ``base'' case at $n=0$, transfinite recursion requires a ``limit'' case at each limit ordinal.

\begin{defn}
  The first infinite cardinal is $\aleph_0=\omega$. If $\aleph_\alpha$ is defined then $\aleph_{\alpha+1}$ is the least ordinal that is not in bijection with $\aleph_\alpha$. If $\lambda$ is a limit ordinal and $\aleph_\alpha$ has been defined for $\alpha<\lambda$, then $\aleph_\lambda=\bigcup_{\alpha<\lambda}\aleph_\alpha$.
\end{defn}

% additional topics
% discuss the value of the continuum |R|
% define cofinality
% konig's lemma limiting the value of |R|:
%   the continuum has uncountable cofinality

\begin{exerc}
  Show that if $\kappa$ is an infinite cardinal, then $\kappa$ is a limit ordinal.
\end{exerc}

\begin{exerc}
  Use Zorn's lemma to prove the well-ordering principle.
\end{exerc}

\begin{exerc}
  If $A$ is an infinite set, show that $|A|$ is equal to $\aleph_\alpha$ for some $\alpha$.
\end{exerc}

\begin{notes}
  This material on ordinals and cardinals can be found in any introductory set theory textbook. For ordinals, see for instance Section~1.7 of Devlin's \emph{The joy of sets}. For cardinals, see Section~3.6 of the same text.
\end{notes}


%%%%%%%%%%%%%%%%%%%%%%%%%%%%%%%%%%%%%%%%%%%%%%%%%%
\section{The topology of Cantor and Baire space}
%%%%%%%%%%%%%%%%%%%%%%%%%%%%%%%%%%%%%%%%%%%%%%%%%%

If $A$ is a countable set then $A^\omega$ denotes the space of all sequences with values in $A$, that is, functions from $\omega$ to $A$. We can endow $A^\omega$ with a topology by regarding $A$ as discrete, $A^\omega$ as a product of countably many copies of $A$, and using the \emph{product topology}. Officially a basic set in the product topology has the form
\[\set{x\in A^\omega\mid x(i_0)\in A_0,\ldots,x(i_{n-1})\in A_{n-1}}
\]
where $i_j\in\omega$ are distinct and $A_j\subset A$ are open. However in our case we can make two simplifications: since $\omega$ is countable we can replace $i_0,\ldots i_{n-1}$ with $0,\ldots,n-1$, and since $A$ is discrete we can suppose the $A_j$ are singletons. Putting this all together, we obtain a basis consisting of all
\[V_s=\set{x\in A^\omega\mid s\subset x}
\]
where $s$ is an element of $A^n$ for any $n$.

The two most important examples of sequence spaces are the \emph{Cantor space} $2^\omega$ and the \emph{Baire space} $\omega^\omega$.

\begin{prop}
  The Cantor space is homeomorphic to the Cantor middle thirds set. The Baire space is homeomorphic to the set of irrational real numbers.
\end{prop}

\begin{proof}
  We give the proof in the case of the Cantor space, and a brief hint in the case of the Baire space.

  The Cantor middle thirds set $C$ has a natural description in terms of ternary expansions. If $a\in[0,1]$ then $a$ lies in the Cantor set if and only if it has a ternary expansion that does not contain the digit $1$. Thus there is a simple bijection $2^\omega\to C$ given by replacing the $1$'s in $x$ with $2$'s:
\[f(x)=0.(2x(0))(2x(1))(2x(2))\cdots
\]
The map is continuous: if $f(x)\in(a,b)$ then there exists $n$ such that if we round $f(x)$ down at its $n$th digit then it is still $>a$ and if we round it up at its $n$th digit then it is still $<b$. It follows that if $s=x\restriction n$, then we have $f(V_s)>a$. Finally recall that a continuous bijection between compact spaces is always ahomeomorphism.

  For the Baire space, it is common to use the values of $x\in(\omega\smallsetminus\{0\})^\omega$ as the entries in a continued fraction:
\[f(x)=x(0)+\cfrac{1}{x(1)+\cfrac{1}{x(2)+\cdots}}
\]
With some elementary number theory, it is possible to verify that this map is a bijection onto the set of irrational numbers, and even a homeomorphism.
\end{proof}

This result also implies that $2^\omega$ admits a complete metric, because the Cantor set is a closed and hence complete subspace of $\RR$. In fact, any sequence space $A^\omega$ is completely metrizable. For an example of a complete metric, given $x,y\in A^\omega$ such that $x\neq y$, let $n$ be the least natural number such that $x(n)\neq y(n)$ and set $d(x,y)=1/n$. 

\begin{prop}
  The metric on $A^\omega$ defined above is complete.
\end{prop}

\begin{proof}
  Let $x_i$ be a Cauchy sequence in $A^\omega$. We can inductively construct an increasing sequence of indices $i_0,i_1,\ldots$ such that for all $n$ and $i\geq i_n$ we have $d(x_i,x_{i_n})<1/n$. In other words for $i\geq i_0$ the $x_i(0)$ all agree, for $i\geq i_1$ the $x_i(1)$ all agree, etc. Thus we may define an element $x$ by letting $x(i)=$ this agreed upon value. Now it is easy to check that $x_i\to x$.
\end{proof}

\begin{rem}
  While the metric on $A^\omega$ defined above is fairly natural, it is not canonical. For example, any reordering $\omega$ would give rise to a different compatible metric.
\end{rem}

We now discuss a variety of topological properties in the context of sequence space. The closed sets, nowhere dense sets, and compact sets all have special descriptions in sequence space.

To begin, let $A^{<\omega}=\bigcup A^n$ denote the set of finite sequences of elements of $A$. This set is partially ordered by the subset relation, but we emplay special terminology in this case. If $s\subset t$ we say that $s$ is an \emph{initial segment} of $t$ or alternatively that $t$ is an \emph{extension} of $s$. We use the same terminology if $s\in A^{<\omega}$, $x\in A^\omega$, and $s\subset x$: $s$ is a finite initial segment of $x$, or $x$ is an infinite extension of $s$.

A subset $T\subset A^{<\omega}$ is said to be a \emph{tree} if it is closed downward, that is, closed under initial segments. An element $x\in A^\omega$ is said to be a \emph{branch} through $T$ if all of its finite initial segments $x\restriction n$ lie in $T$. We denote by $[T]$ the subset of $A^\omega$ consisting of all branches through $T$.

\begin{prop}
  A subset $C\subset A^\omega$ is closed if and only if there exists a tree $T\subset A^{<\omega}$ such that $C=[T]$.
\end{prop}

\begin{proof}
  Given any set $B\subset A^\omega$, we let $T_B$ be the tree consisting of all $s$ such that $V_s\cap B\neq\emptyset$ (that is, all initial segments of elements of $B$). We will show that the set $[T_B]$ of all branches through the tree $T_B$ is precisely the \emph{closure} of $B$, which implies the desired result. For this, note that $x$ lies in the closure of $B$ if and only if every open neighborhood of of $x$ meets $B$. This is equivalent to the statement that for every initial segment $s\subset x$, $V_s\cap B\neq\emptyset$. Finally, this is equivalent to the statement that $x$ is a branch through $T_B$.
\end{proof}

In previous sections we have defined nowhere dense subsets of $\RR$ in terms of intervals. In fact we can define nowhere dense subsets of any topological space in terms of open sets: $S\subset X$ is \emph{nowhere dense} in $X$ if every open set has an open subset disjoint from $S$. This definition can even be made with basic open sets in place of open sets. Thus we have the following characterization.

\begin{prop}
  \label{prop:cantor-space-nwd}
  $S\subset A^{<\omega}$ is nowhere dense if and only if for every $s\in A^{<\omega}$ there exists $t$ such that $s\subset t$ and $V_t$ is disjoint from $S$.\qed
\end{prop}

Meager sets are now defined in the same way as before: $X\subset A^{<\omega}$ is \emph{meager} if it is the union of countably many nowhere dense sets. Our proof of the Baire category theorem for the real numbers naturally extends to arbitrary complete metric spaces.

\begin{thm}[Baire category theorem]
  If $X$ is a complete metric space then $X$ is not meager. Moreover if $O$ is a nonempty open subset of $X$ then $O$ is not meager.\qed
\end{thm}

In particular the notions of meager and comeager sets make perfect sense in both $2^\omega$ and $\omega^\omega$. The proof of this general Baire category theorem is nearly identical to the proof of Theorem~\ref{thm:baire} with closed intervals replaced by closures of basic open sets $\overline{O_n}$. The completeness of $X$ is used to verify there is a point $x\in\bigcap\overline{O_n}$.

\begin{exerc}
  Check that the topology on $A^\omega$ with basis consisting of all $V_s$ is really the same as the product topology.
\end{exerc}

\begin{exerc}
  Give an example of an open subset of $\omega^\omega$ which is not closed.
\end{exerc}

\begin{exerc}
  Show that the sequence space $A^\omega$ is homeomorphic to the cartesian product with itself $A^\omega\times A^\omega$.
\end{exerc}

\begin{notes}
  For introductory material on the combinatorics and topology of sequence space, see Kechris, Sections~2A and~2B. For another proof that the Baire space is homeomorphic to the irrationals, see Section~1 of Miller, \emph{Descriptive set theory and forcing}.
\end{notes}


%%%%%%%%%%%%%%%%%%%%%%%%%%%%%%%%%%%%%%%%%%%%%%%%%%
\section{$\sigma$-compactness in Baire space}
%%%%%%%%%%%%%%%%%%%%%%%%%%%%%%%%%%%%%%%%%%%%%%%%%%

As we have discussed in the introduction, the exact cardinal value of the size of the set of all real numbers is independent of the axioms of set theory. That is, if we write $\mathfrak c=|\RR|$ then we know $\mathfrak c=\aleph_\alpha$ for some $\alpha\geq1$, but we do not know which one. Here $\mathfrak c$ stands for \emph{continuum}. Since $\mathfrak c$ is also equal to the cardinality of Cantor space, it is also often denoted $2^{\aleph_0}$. In this section we discuss several cardinal values other than $\mathfrak c$ that arise from the special topology of the Baire space $\omega^\omega$.

We have measured the size of sets of real numbers using cardinality, category, and measure. We now introduce a fourth notion in the Baire space. A set $A\subset\omega^\omega$ is said to be \emph{$\sigma$-compact} if it is the union of countably many compact sets.

Like the meager and null sets, the class of $\sigma$-compact sets is closed under countable unions. While the class of $\sigma$-compact sets is not closed under subsets, if one wishes they can simply consider instead the class of subsets of $\sigma$-compact sets instead. In a moment we shall show that the whole Baire space $\omega^\omega$ is not $\sigma$-compact, so that $\sigma$-compactness really is a good notion of a small set.

\begin{rem}
  Among the spaces we've explored so far, the notion of $\sigma$-compactness only works in the Baire space. That's because both $2^\omega$ and $\RR$ both have the property that the whole space is $\sigma$-compact: $\RR$ is the union of closed bounded intervals, and $2^\omega$ is itself compact.
\end{rem}

We now introduce for the first time the process of associating \emph{cardinal characteristics} to a class of sets. First we denote by $\Ksigma$ the class of all sets $A\subset\omega^\omega$ such that $A$ is a subset of some $\sigma$-compact set.

\begin{defn}
  \begin{itemize}
  \item The cardinal $\non(\Ksigma)$ is the least cardinality of any set $A\subset\omega^\omega$ which is not $K_\sigma$.
  \item The cardinal $\cov(\Ksigma)$ is the least cardinality of any family $\mathcal F$ of subsets of $\omega^\omega$ whose union covers all of $\omega^\omega$.
  \end{itemize}
\end{defn}

Our next result will generalize Cantor's theorem by showing that the cardinals $\non(\Ksigma)$ and $\cov(\Ksigma)$ are uncountable lower bounds for the value of the continuum. First we need the following characterizations of compact sets in Baire space.

\begin{lem}
  \label{lem:baire-compact}
  A subset $A\subset\omega^\omega$ is compact if and only if it is a closed subset of some product of bounded intervals $\prod_{n\in\omega}\{0,\ldots,K_n\}$.
\end{lem}

\begin{proof}
  We will use the fact that a countable product of finite metric is compact (this follows from Tychonoff's theorem, but it's easy to check directly too). Since a closed subspace of a compact metric space is again compact, it follows that any closed subset of a product $\prod_{n\in\omega}\{0,\ldots,K_n\}$ is compact. For the converse suppose that $A\subset\omega^\omega$ is compact, and suppose towards a contradiction that $A$ is not contained in a product of bounded intervals. Then there exists some coordinate $n$ such that $A$ contains a sequence of elements $f_i$ with $f_i(n)\to\infty$. Then $f_i$ clearly does not have any convergent subsequence, contradicting that $A$ was compact.
\end{proof}

\begin{thm}
  \label{thm:k-sigma}
  We have $\aleph_1\leq\non(\Ksigma)\leq\mathfrak c$ and $\aleph_1\leq\cov(\Ksigma)\leq\mathfrak c$.
\end{thm}

\begin{proof}
  It is clear that $\aleph_1\leq\non(\Ksigma)$ because every countable set is $\Ksigma$, being the union of its points. To see that $\non(\Ksigma)\leq\mathfrak c$ it is enough to show that $\omega^\omega$ itself is not $\sigma$-compact. So suppose towards a contradiction that $\omega^\omega=\bigcup A_n$ where each $A_n$ is compact. Then by the Baire category theorem, at least one of the $A_n$ must have a nonempty interior, say $V_s\subset A_n$. But $V_s$ is not contained in a product of bounded intervals, so by Lemma~\ref{lem:baire-compact} this contradicts that $A_n$ is compact.

  To show that $\aleph_1\leq\cov(\Ksigma)$, suppose a countable family of $\sigma$-compact sets covered all of $\omega^\omega$. Then a countable family of compact sets would also cover all of $\omega^\omega$, again contradicting that $\omega^\omega$ is not $\sigma$-compact. Finally it is clear that $\cov(\Ksigma)\leq\mathfrak c$, since the whole space $\omega^\omega$ is the union of its points and $|\omega^\omega|=\mathfrak c$.
\end{proof}

In fact we'll see shortly that $\non(\Ksigma)\leq\cov(\Ksigma)$.

\begin{rem}
  \label{rem:ksigma-meager}
  It follows from the proof of Theorem~\ref{thm:k-sigma} that every $\Ksigma$ set in $\omega^\omega$ is meager. Indeed, compact sets are closed, and moreover since they are bounded they cannot contain any basic open sets $V_s$. It follows that compact sets are nowhere dense, and hence $\sigma$-compact sets are meager. This in turn implies the Baire category property for $\sigma$-compact sets: $\omega^\omega$ is not $\sigma$-compact and any basic open neighborhood $V_s$ is not $\sigma$-compact.
\end{rem}

We now work establish a key relationship between the cardinals $\non(\Ksigma)$ and $\cov(\Ksigma)$, and the combinatorics of the Baire space. To begin, we can partially order Baire space by $f\leq g$ iff for all $n$, we have $f(n)\leq g(n)$. We also have the more versatile \emph{eventual domination} order: $f\leq^*g$ if $f\leq g$ almost everywher, that is, there is some $N$ such that for all $n\geq N$ we have $f(n)\leq g(n)$. Of course eventual domination is not a partial order because $f\leq^*g\leq^*f$ only implies that $f$ and $g$ agree almost everywhere, which we denote by $f=^*g$.

\begin{defn}
  \begin{itemize}
  \item A subset $\mathcal F\subset\omega^\omega$ is said to be a \emph{dominating family} if for all $h\in\omega^\omega$ there exists $f\in\mathcal F$ such that $h\leq^*f$. The \emph{dominating number} $\mathfrak d$, is the smallest cardinality of any dominating family.
  \item A subset $\mathcal F\subset\omega^\omega$ is said to be an \emph{unbounded family} if there is no single $h\in\omega^\omega$ such that $f\leq^*h$ for all $f\in\mathcal F$. The \emph{unbounding number} $\mathfrak b$ is the smallest cardinality of any unbounded family.
  \end{itemize}
\end{defn}

The following result records the relationship between these two cardinal values.

\begin{prop}
  \label{prop:bd}
  We have $\aleph_1\leq\mathfrak{b}\leq\mathfrak{d}\leq\mathfrak{c}$.
\end{prop}
	
\begin{proof}
  We begin by showing that $\mathfrak{b}$ is uncountable. For this, it is enough to show that every countable family $\mathcal F\subset\omega^\omega$ is bounded by some $h\in\omega^\omega$. So let $\mathcal F\subset\omega^\omega$ be a countable family and enumerate its elements $f_0,f_1,\ldots$. To construct the bound $h$, we use a diagonalization process similar to the proof of Cantor's theorem. More specifically, we define $h(n)$ so that it dominates $f_0(n),\ldots,f_n(n)$:
\[h(n)=\max\set{f_i(n)\mid 0\leq i\leq n}
\]
Then for each $i$ we have $f_i(n)\leq h(n)$ for all $n\geq i$, and therefore $f_i\leq^*h$, as desired.

To show that $\mathfrak{b}\leq\mathfrak{d}$, it is enough to show that every dominating family is unbounded. Let us establish the contrapositive: if $\mathcal F$ is bounded, say by $h\in\omega^\omega$, then every $f\in\mathcal F$ satisfies $f\leq^*h$. Letting $h'(n)=h(n+1)$, it follows that every $f\in\mathcal F$ satisfies $h'\not\leq^*f$, and hence that $\mathcal F$ is not a dominating family.

Finally, $\mathfrak d\leq\mathfrak c$ since $\mathcal F=\omega^\omega$ is itself a dominating family, and $|\omega^\omega|=\mathfrak c$.
\end{proof}

We are now ready to state the connection between $\Ksigma$ sets and the eventual domination ordering.

\begin{thm}
  \label{thm:bd-vs-ksigma}
  We have $\mathfrak b=\non(\Ksigma)$ and $\mathfrak d=\cov(\Ksigma)$.
\end{thm}

\begin{proof}
  Let us say that a family $\mathcal F\subset\omega^\omega$ is $\leq$-bounded if there is an $h\in\omega^\omega$ such that $f\leq h$ for all $f\in\mathcal F$, and that $\mathcal F$ is $\leq^*$-bounded if there is an $h\in\omega^\omega$ such that $f\leq^*h$ for all $f\in\mathcal F$. Then it is clear from Lemma~\ref{lem:baire-compact} that a family $\mathcal F$ is $\leq$-bounded if and only if $\mathcal F$ is contained in a compact set. To establish the first equality in the theorem, it suffices to show that a family $\mathcal F$ is $\leq^*$-bounded if and only if $\mathcal F$ is contained in a $\sigma$-compact set.

  For this, suppose that $\mathcal F$ is $\leq^*$-bounded by $h\in\omega^\omega$. Let $h_i$ enumerate all finite modifications of $h$, so that for all $f\in\mathcal F$ there is some $i$ such that $f\leq h_i$. It follows that $\mathcal F$ is contained in the $\sigma$-compact set of functions that are $\leq$-bounded by some $h_i$. Conversely, suppose that $\mathcal F\subset\bigcup A_i$, where each $A_i$ is compact. Then for each $i$ there exists some $h_i\in\omega^\omega$ such that $A_i$ is $\leq$-bounded by $h_i$. Since $\aleph_1\leq\mathfrak b$, there exists $h$ such that $h_i\leq^*h$ for all $i$. It follows that $\mathcal F$ is $\leq^*$-bounded by $h$.

  The second equality is similar. If $\mathcal F$ is a dominating family then consider the collection of all $A_f=\set{g\mid g\leq f}$ for $f$ a finite modification of an element of $\mathcal F$. This latter collection is a family of compact sets that covers all of $\omega^\omega$. Conversely if $\mathcal F$ is a family of compact subsets of $\omega^\omega$ that covers all of $\omega^\omega$, then each $F\in\mathcal F$ is $\leq$-bounded by some $h_F\in\omega^\omega$. It follows that the collection of all $h_F$ is a dominating family.
\end{proof}

\begin{exerc}
  Prove directly that a countable product of finite metric spaces is compact.
\end{exerc}

\begin{exerc}
  Find an example of a meager subset of $\omega^\omega$ which is not $\Ksigma$.
\end{exerc}

\begin{exerc}
  Consider the space $\RR^\omega$ with the product topology. Is it $\sigma$-compact?
\end{exerc}

\begin{notes}
  The conection between $\mathcal K_\sigma$ and the domination relation $\leq^*$ is presented in Section~2 of Blass, \emph{Combinatorial cardinal characteristics of the continuum}.
\end{notes}


%%%%%%%%%%%%%%%%%%%%%%%%%%%%%%%%%%%%%%%%%%%%%%%%%%
\section{Ideals and their cardinals}
%%%%%%%%%%%%%%%%%%%%%%%%%%%%%%%%%%%%%%%%%%%%%%%%%%

Let $X$ be any one of the three spaces $\RR$, $2^\omega$, or $\omega^\omega$. An \emph{ideal} on $X$ is a subset $\mathcal I\subset\mathcal P(X)$ which is closed under subsets and (finite) unions. We think of an ideal as any collection of sets that captures some quality of smallness in subsets of $X$. Of course, the whole space $X$ should not be small, so we are only interested in \emph{proper} ideals, i.e.\ those do not contain $X$.

An ideal $\mathcal I$ is called a \emph{$\sigma$-ideal} if it is additionally closed under countable unions. In the past several sections, we have discussed three key notions of smallness and each one is a $\sigma$-ideal.
\begin{itemize}
\item $\Meager$ is the ideal of meager subsets of $\RR$.
\item $\Null$ is the ideal of Lebesgue null subsets of $\RR$.
\item $\Ksigma$ is the ideal of subsets of $\omega^\omega$ which are contained in a $\sigma$-compact subset of $\omega^\omega$.
\end{itemize}

As was the case with $\Ksigma$, we can associate cardinal characteristics to any ideal. We will consider four key characteristics.

\begin{defn} Let $\mathcal I$ be an ideal of subsets of a set $X$, such that $\mathcal I$ contains all singletons of $X$. 
  \begin{itemize}
  \item The \emph{additivity} of $\mathcal{I}$, $\add(\mathcal{I})$, is the smallest number of sets in $\mathcal{I}$ whose union is not in $\mathcal{I}$.
  \item The \emph{uniformity} of $\mathcal{I}$, $\non(\mathcal{I})$, is the smallest cardinality of any subset of $X$ that is not in $\mathcal{I}$.
  \item The \emph{covering number} of $\mathcal{I}$, $\cov(\mathcal{I})$, is the smallest number of sets in $\mathcal{I}$ whose union covers all of $X$.
  \item The \emph{cofinality} of $\mathcal{I}$, $\cof(\mathcal{I})$, is the smallest cardinality of any subset $\mathcal{B}\subset\mathcal{I}$ such that every element of $\mathcal{I}$ is a subset of an element of $\mathcal{B}$.
  \end{itemize}
\end{defn}

\begin{rem}
The cofinality of an ideal is the least number of sets you need to generate the ideal by closing under subsets. More precisely, a subset $\mathcal B\subset\mathcal I$ is called a \emph{basis} for $\mathcal I$ if every element of $I$ is a subset of some element of $\mathcal B$. Thus the cofinality of $\mathcal I$ is the least cardinality of a basis for $\mathcal I$.
\end{rem}

As was the case with the cardinals we introduced in the previous section, assuming $\mathcal I$ is reasonable, all four of the cardinals characteristics associated with $\mathcal I$ provides an uncountable lower bound for the value of the continuum.

\begin{lem}
  \label{lem:ideal-cards}
  Suppose that $\mathcal I$ is a proper $\sigma$-ideal, $\mathcal I$ contains the singletons, and that $\mathcal I$ has a basis consisting of Borel sets. Then each of the four cardinals above is uncountable and bounded above by $\mathfrak c$.
\end{lem}

\begin{proof}
  The additivity of $\mathcal I$ is uncountable simply because $\mathcal I$ is a $\sigma$-ideal. In the next lemma we will show that if $\mathcal I$ is a proper $\sigma$-ideal containing the singletons, then the additivity of $\mathcal I$ is a lower bound for all four cardinal characteristics of $\mathcal I$, and hence all four are uncountable.

  Next, it is easy to verify that for any proper $\sigma$-ideal $\mathcal I$ containing the singletons, the additivity, uniformity, and covering number of $\mathcal I$ are bounded above by $\mathfrak c$. If additionally $\mathcal I$ has a basis consisting of Borel sets, then since there are only $\mathfrak c$ many Borel sets, we have that the cofinality of $\mathcal I$ is bounded above by $\mathfrak c$ too.
\end{proof}

All three of the $\sigma$-ideals we have introduced have a basis of Borel sets. For $\Ksigma$ this is clear because $\sigma$-compact sets are Borel. For $\Meager$ any meager set is contained in a countable union of \emph{closed} nowhere dense sets, which are Borel. Finally for $\Null$, note that if $A$ is null then for all $n$ there is a Borel set $A_n$ such that $A\subset A_n$ and $m(A_n)<1/n$. It follows that $A$ is contained in the Borel null set $\bigcap A_n$.

\begin{rem}
  There can exist proper $\sigma$-ideals containing the singletons that don't have a basis of size $\leq\mathfrak c$. For example, it is consistent that the ideal $\mathcal{SN}$ of \emph{strongly null} sets has $\cof(\mathcal{SN})=\aleph_2$ while $\mathfrak c=\aleph_1$. [ref] % yorioka
\end{rem}

We next establish the basic relationships between the four cardinal characteristics associated with $\mathcal I$.

\begin{lem}
  \label{lem:diamond}
  Suppose that $\mathcal I$ is a proper $\sigma$-ideal and that $\mathcal I$ contains the singletons. Then we have the inequalities $\add(\mathcal I)\leq\non(\mathcal I)\leq\cof(\mathcal I)$, and also $\add(\mathcal I)\leq\cov(\mathcal I)\leq\cof(\mathcal I)$.
\end{lem}

\begin{proof}
  For the inequality $\add(\mathcal{I})\leq\non(\mathcal{I})$, we will show that for every set $A$ not in $\mathcal I$, we can find a family $\mathcal F$ of the same (or lower) cardinality such that $\bigcup\mathcal F\notin I$. This is easy: if $A\notin\mathcal I$, then the family $\mathcal F$ consisting of all $\{a\}$ such that $a\in A$ has the same cardinality as $A$ and satisfies $\bigcup\mathcal F\notin\mathcal I$.

  The proofs of each of the next three inequalities follows the same form: given a set indicated by the right-hand side, find a set of the same or lower cardinality indicated by the left-hand side. So to show $\non(\mathcal I)\leq\cof(\mathcal I)$, let $\mathcal F\subset \mathcal{I}$ be a basis for $\mathcal I$. For each set $B\in\mathcal F$ choose an element $x_B\notin B$ and let $A=\set{x_B\mid B\in\mathcal F}$. Then $A$ has the same or lower cardinality as $\mathcal F$, and we claim that $A\notin\mathcal{I}$. Indeed if $A\in\mathcal{I}$ then we would have $A\subseteq B$ for some $B\in\mathcal F$, contradicting that $x_B\notin B$.

  For the inequality $\add(\mathcal I)\leq\cov(\mathcal I)$, if $\mathcal F\subset\mathcal{I}$ satisfies $\bigcup\mathcal F=X$, then $\mathcal F$ itself satisfies $\bigcup\mathcal F\notin\mathcal I$.

  For the inequality $\cov(\mathcal I)\leq\cof(\mathcal I)$, again suppose that $\mathcal F$ is a basis for $\mathcal I$. Since $\mathcal I$ contains the singletons, it follows that $\mathcal F$ itself covers $X$.
\end{proof}

The conclusions of the two lemmas are summarized in Figure~\ref{fig:ideal}

\begin{figure}[h]
  \begin{tikzpicture}[->]
    \node (a1) {$\aleph_1$};
    \node (add) [right=of a1] {$\add(\mathcal I)$} edge (a1);
    \node (non) [above=of add] {$\non(\mathcal I)$} edge (add);
    \node (cov)	[right=of add] {$\cov(\mathcal I)$} edge (add);
    \node (cof) [above=of cov] {$\cof(\mathcal I)$}
      edge (non) edge (cov);
    \node [right=of cof] {$\mathfrak{c}$} edge (cof);
  \end{tikzpicture}
  \caption{The relationships between the cardinal characteristics of $\mathcal I$, assuming $\mathcal I$ satisfies the hypotheses of the two lemmas. In the figure, the arrow $\leftarrow$ represents the inequality $\leq$.\label{fig:ideal}}
\end{figure}

In the previous section we introduced $\Ksigma$, but we dealt only with the characteristics $\non(\Ksigma)$ and $\cov(\Ksigma)$. We conclude this section by giving the values of $\add(\Ksigma)$ and $\cof(\Ksigma)$ as well.

\begin{prop}
  We have $\add(\Ksigma)=\mathfrak b$ and $\cof(\Ksigma)=\mathfrak d$.
\end{prop}

\begin{proof}
  We have already shown that $\add(\Ksigma)\leq\non(\Ksigma)=\mathfrak b$. For the reverse inequality, first recall from Theorem~\ref{thm:bd-vs-ksigma} that a subset of $\omega^\omega$ is unbounded if and only if it is not $\Ksigma$. Now suppose that $\mathcal F$ is a family of $\Ksigma$ subsets of $\omega^\omega$ such that $\bigcup\mathcal F$ is not $\Ksigma$. For each $F\in\mathcal F$ let $g_F$ be a bound for $F$. Then $\{g_F\mid F\in\mathcal F\}$ is an unbounded family: Indeed, otherwise the set of things bounded by some $g_F$ would be bounded and hence so would $\bigcup\mathcal F$. This shows that $\add(\Ksigma)\geq\mathfrak b$.

The second equality is similar and left as Exercise~\ref{exerc:non-ksigma-d}.
\end{proof}

\begin{exerc}
  \label{exerc:non-ksigma-d}
  Show that $\cof(\Ksigma)=\mathfrak d$.
\end{exerc}

\begin{exerc}
  Let $\mathcal I$ be the ideal of all countable subsets of $\RR$. What are the values of the four cardinal characteristics of $\mathcal I$?
\end{exerc}


%%%%%%%%%%%%%%%%%%%%%%%%%%%%%%%%%%%%%%%%%%%%%%%%%%
\section{Asymmetry of category and measure: Cicho\'n's diagram}
%%%%%%%%%%%%%%%%%%%%%%%%%%%%%%%%%%%%%%%%%%%%%%%%%%

So far we have seen many similarities between the various size notions that we have discussed. The classes of meager sets and null sets are both ideals, both proper, both closed under countable unions, and so on. In this section we further investigate how far this comparison extends, and in the process we reveal a deep and intricate connection between measure and category.

To begin, it is natural to ask whether the meager and null sets are truly different notions. The following result shows that they are very different indeed: it is possible to be small according to either notion, and at the same time, large according to the other.

\begin{prop}
  \label{prop:null-comeager}
  There exists a null comeager set. There exists a meager conull set.  \end{prop}

\begin{proof}
  The key is that we can construct dense open sets of arbitrarily small measure. That is, for each $n$ we will construct a dense open set $A_n$ such that $m(A_n)<1/n$. Assuming we have done so, we let $A=\bigcap A_n$. Clearly the set $A$ is null. To see that it is comeager note that for all $n$, we have that $\RR\smallsetminus A_n$ is closed and nowhere dense. Thus $\RR\smallsetminus A=\bigcup\RR\smallsetminus A_n$ is meager.

  To construct the set $A_n$, let $q_0,q_1,\ldots$ be an enumeration of the rational numbers, and for each $i$ let $I_{n,i}$ be the open interval centered at $q_i$ of length $1/(n2^{i+1})$. We then let $A_n=\bigcup_iI_{n,i}$. Clearly $A_n$ is dense and open, and $m(A_n)\leq\sum_i1/(n2^{i+1})=1/n$.

  Finally, since $A$ is null and comeager, we clearly have that $\RR\smallsetminus A$ is meager and conull.
\end{proof}

Although this result shows that category and measure are very different in terms of which sets they deem small, it also provides a certain symmetry between the two notions. The next result breaks this symmetry, and in its place provides an intricate structure of relationships between category and measure. This result is what we came all this way to see.

\begin{thm}
  \label{thm:cichon}
  The cardinal characteristics defined in the previous sections exhibit the pattern of inequalities known as \emph{Cicho\'n's diagram}, which is depicted in Figure~\ref{fig:cichon}.
\end{thm}

\begin{figure}[h]
  \begin{tikzpicture}[->]
    \node (a1) {$\aleph_1$};
    \node (addn) [right=of a1] {$\add(\Null)$} edge (a1);
    \node (covn) [above=2 of addn] {$\cov(\Null)$} edge (addn);
    \node (addm) [right=of addn] {$\add(\Meager)$} edge (addn);
    \node (b) [above=.8 of addm] {$\mathfrak b$} edge (addm);
    \node (nonm) [above=2 of addm] {$\non(\Meager)$} edge (covn) edge (b);
    \node (covm) [right=of addm] {$\cov(\Meager)$} edge (addm);
    \node (d) [above=.8 of covm] {$\mathfrak d$} edge (b) edge (covm);
    \node (cofm) [above=2 of covm] {$\cof(\Meager)$} edge (nonm) edge (d);
    \node (nonn) [right=of covm] {$\non(\Null)$} edge (covm);
    \node (cofn) [above=2 of nonn]{$\cof(\Null)$} edge (cofm) edge (nonn);
    \node (c) [right=of cofn] {$\mathfrak c$} edge (cofn);
  \end{tikzpicture}
  \caption{Cicho\'n's diagram of cardinal characteristics. Here as usual the arrow $\leftarrow$ represents the inequality $\leq$.\label{fig:cichon}}
\end{figure}

\begin{rem}
  By Exercise~\ref{exerc:meager-bij}, the symbol $\Meager$ can stand for the meager sets on any one of the spaces $\RR$, $2^\omega$, or $\omega^\omega$. Similarly, in Exercise~\ref{exerc:null-bij} we give a natural measure on $2^\omega$, and again show that $\Null$ can stand for the null sets on either $\RR$ or $2^\omega$. (We omit the definition of an appropriate measure on $\omega^\omega$ [ref].) % blass
\end{rem}

\begin{rem}
  Theorem~\ref{thm:cichon} doesn't rule out the possibility that all or some of the cardinals in Cicho\'n's diagram are \emph{equal to each other}. And this is indeed possible: if CH is true then $\aleph_1=\mathfrak c$ and so all twelve of the cardinals are identical. So to truly break the symmetry between category and measure, we need some way to show that it is possible for these cardinals to be different. For that we will need the method of \emph{forcing}, which will be the focus of the second half of the course.
\end{rem}

For the rest of this section as well as the next few, we take up the proof of Theorem~\ref{thm:cichon}. We have already established seven out of the fifteen inequalities depicted: $\mathfrak b\leq\mathfrak d$ follows from Proposition~\ref{prop:bd}. The two bounds $\aleph_1\leq\add(\Null)$ and $\cof(\Null)\leq\mathfrak c$ follow from Lemma~\ref{lem:ideal-cards}. And four more of the inequalities are special cases of Lemma~\ref{lem:diamond}.

We conclude this section with proofs of four more fairly easy inequaleties. The remaining four turn out to be be slightly more difficult. In the next section we will present a handy but abstract tool that will reduce our workload to just two more inequalities. The final two will be proved in the following two sections.

\begin{lem}
  \label{lem:b-nonm}
  We have $\mathfrak b\leq\non(\Meager)$, and $\cov(\Meager)\leq\mathfrak d$.
\end{lem}

\begin{proof}
  For the first inequality, recall our observation in Remark~\ref{rem:ksigma-meager} that $\Ksigma\subset\Meager$, so any nonmeager subset of $\omega^\omega$ is certainly not $\Ksigma$. But we also know that every non-$\Ksigma$ set is unbounded. It follows that every nonmeager subset of $\omega^\omega$ is unbounded, which implies $\mathfrak b\leq\non(\Meager)$.

  The second inequality again uses the fact that $\Ksigma\subset\Meager$, and we leave it as Exercise~\ref{exerc:covm-d}.
\end{proof}

\begin{lem}
  \label{lem:covn-nonm}
  We have $\cov(\Null)\leq\non(\Meager)$, and $\cov(\Meager)\leq\non(\Meager)$.
\end{lem}

\begin{proof}
  For the first inequality, we must show that given a nonmeager set $X$, we can find a family $\mathcal F$ of null sets of smaller or equal size such that $\bigcup F=\RR$. For this, we let $A$ be the null comeager set constructed in Proposition~\ref{prop:null-comeager}, and consider the family of null sets $\mathcal F=\set{x+A\mid x\in X}$.

  To see that $\bigcup F=\RR$, suppose towards a contradiction that this is not the case. Then there exists $z\in\RR$ such that $z\notin x+A$ for all $x\in X$. It follows that $z-x\notin A$ for all $x\in X$, or in other words that $z-X$ is disjoint from $A$. But $A$ is comeager, so this would imply that $z-X$ is meager, which is a contradiction.

  The proof of the second inequality is identical but with the terms meager and null exchanged. This time let $X$ be nonmeager, and let $A$ be meager conull. Then the same argument shows that the family $\mathcal F$ defined above covers $\RR$, since otherwise there would be a translate $z-X$ of $X$ which is null.
\end{proof}

\begin{exerc}
  \label{exerc:meager-bij}
  There is a homeomorphism between comeager subsets of $\RR$ and $2^\omega$. There is a bijection between comeager subsets of $\RR$ and $\omega^\omega$.
\end{exerc}

\begin{exerc}
  \label{exerc:null-bij}
  For $V_s$ a basic open set of $2^\omega$, let $m(V_s)=2^{-|s|-1}$. Then $m$ extends to a measure on the Borel sets of $2^\omega$ (take this for granted). Show that there is a measure-preserving bijection between conull subsets of $[0,1]$ and $2^\omega$.
\end{exerc}

\begin{exerc}
  \label{exerc:covm-d}
  Show that $\cov(\Meager)\leq\mathfrak d$.
\end{exerc}

% See Bartoszynski


%%%%%%%%%%%%%%%%%%%%%%%%%%%%%%%%%%%%%%%%%%%%%%%%%%
\section{Duality}
%%%%%%%%%%%%%%%%%%%%%%%%%%%%%%%%%%%%%%%%%%%%%%%%%%

At this point one might have observed that many of the proofs of inequalities between cardinal characteristics have come in pairs. This phenomenon is easy to spot in Lemmas~\ref{lem:b-nonm} and~\ref{lem:covn-nonm}, but if one examines the proof it is even present in Lemma~\ref{lem:diamond}. It turns out that there is a category-theoretic duality lurking behind all of these pairings. (Here we mean \emph{category} as in objects and morphisms, not as in Baire's theorem.)

\begin{defn}
  Let $R$ be a relation and $R\subset A\times B$ (that is, the domain of $R$ is $A$ and the codomain of $R$ is $B$).
  \begin{itemize}
  \item A subfamily $\mathcal F\subset B$ is an \emph{$R$-dominating family} if for every $a\in A$ there exists $b\in\mathcal F$ such that $a\mathrel{R}b$.
  \item The \emph{dominating number} $\mathfrak d(R)$ of $R$ is the minimum cardinality of an $R$-dominating family.
  \end{itemize}
\end{defn}

It is clear that the dominating number $\mathfrak d$ is the dominating number of the relation $\leq^*$. But it is also easy to check that the unbounding number $\mathfrak b$ can be expressed as the dominating number of the relation $\not\geq^*$. Moreover, if $\mathcal I$ is an ideal then all four of the cardinal characteristics of $\mathcal I$ can be expressed as dominating numbers.

\begin{prop}
  Let $\mathcal I$ be an ideal on $X$. Then we have:
  \begin{itemize}
  \item $\cof(\mathcal I)$ is the dominating number of the subset relation $\subset$ on $\mathcal I$ (meaning $\mathcal I\times\mathcal I)$.
  \item $\cov(\mathcal I)$ is the dominating number of the $\in$ relation on $X\times\mathcal I$.
  \item $\non(\mathcal I)$ is the dominating number of the $\not\ni$ relation on $\mathcal I\times X$.
  \item $\add(\mathcal I)$ is the dominating number of the relation $\not\supset$ on $\mathcal I$.
  \end{itemize}
\end{prop}

\begin{proof}
  It is straightforward to see the characterizations of $\cof(\mathcal I)$ and $\cov(\mathcal I)$. The characterizations of $\non(\mathcal I)$ and $\add(\mathcal I)$ both use the downward closure property: $A$ lies in $\mathcal I$ if and only if $A$ is a subset of an element of $\mathcal I$.
\end{proof}

It is worth observing that the four cardinals come in two dual pairs. Here, if $R$ is a relation, then the \emph{dual} of $R$ is its negated transpose $\hat{R}$. That is, the dual of $R$ satisfies $a\mathrel{\hat{R}}b$ iff $\neg(b\mathrel Ra)$. Thus we see that $\add(\mathcal I)$ and $\cof(\mathcal I)$ arise from dual relations, as do $\non(\mathcal I)$ and $\cov(\mathcal I)$. Before we can exploit this duality, we need to describe the morphisms on the category of binary relations.

\begin{defn}
  Let $R$ and $S$ be relations, and to avoid excessive notation, write $\dom(R)$ and $\cod(R)$ for the demain and codomain of $R$, and similarly for $S$. A \emph{morphism} from $R$ to $S$ is a pair of maps
\[\begin{cases}\psi\colon\cod(R)\to\cod(S)\\\phi\colon\dom(S)\to\dom(R)\end{cases}
\]
such that the following holds
\[\phi(c)\mathrel{R}d\implies c\mathrel{S}\psi(d)
\]
for all $c\in\dom(S)$ and $d\in\cod(R)$.
\end{defn}

The definition in a sense implies that the following diagram ``commutes.'' Of course, the $R$ and $S$ edges symbolize relations, not functions, so this is only a partial analogy.
\begin{center}
\begin{tikzpicture}
  \node (domr) {$\dom(R)$};
  \node (codr) [above=of domr] {$\cod(R)$}
  edge[<-,dashed] node[left]{$R$} (domr);
  \node (doms) [right=of domr] {$\dom(S)$}
  edge[->] node[below]{$\phi$} (domr);
  \node (cods) [above=of doms] {$\cod(S)$}
  edge[<-] node[above]{$\psi$} (codr)
  edge[<-,dashed] node[right]{$S$} (doms);
\end{tikzpicture}
\end{center}
In any case, the definition of a morphism is custom-designed to give us the following two key properties.

\begin{prop}
  \label{prop:morphism}
  \begin{itemize}
  \item If $(\phi,\psi)$ is a morphism from $R$ to $S$, then $\mathfrak d(R)\geq\mathfrak d(S)$.
  \item If $(\phi,\psi)$ is a morphism from $R$ to $S$, then $(\psi,\phi)$ is a morphism from $\hat S$ to $\hat R$. Hence in this case we also have $\mathfrak d(\hat S)\geq\mathfrak d(\hat R)$.
  \end{itemize}
\end{prop}

\begin{proof}
  For the first statement, it is enough to show that if $\mathcal F$ is an $R$-dominating family, then the image $\psi(\mathcal F)$ is an $S$-dominating family. Indeed, the image $\psi(\mathcal F)$ has the same or smaller cardinality than $\mathcal F$. To check this we simply chase the diagram: since $\mathcal F$ is $R$-dominating, for any $d\in\dom(S)$ there exists $f\in\mathcal F$ such that $\phi(d)\mathrel{R}f$. Since $(\phi,\psi)$ is a morphism, it follows that $d\mathrel{S}\psi(f)$. This confirms that $\psi(\mathcal F)$ is an $S$-dominating family.

  For the second statement, just take the contrapositive of the morphism property, $\phi(c)\mathrel R d\implies c\mathrel S\psi(d)$, to arrive at the property required of a morphism from $\hat S$ to $\hat R$.
\end{proof}

For each of the inequalities between cardinal characteristics that we have seen, it is possible to extract a morphism hiding in the proof. 

\begin{example}
  In Lemma~\ref{lem:diamond}, where we showed that $\cov(\mathcal I)\leq\cof(\mathcal I)$, we could have used the morphism from $\subset$ relation on $\mathcal I$ to the $\in$ relation on $X\times\mathcal I$ given by the maps
  \[\begin{cases}\phi(x)=\{x\}\\\psi(B)=B\end{cases}
  \]
  Indeed, it is trivial to see that $\phi(x)\subset B\implies x\in\psi(B)$. The fact that $\psi$ can be taken to be the identity reflects the fact that if $\mathcal F$ is a basis for $\mathcal I$ then the very same $\mathcal F$ is also a covering family.

  Moreover, if we exchange the roles of $\phi$ and $\psi$ we obtain a morphism from $\not\ni$ to $\not\supset$. This morphism lies behind the inequality $\add(\mathcal I)\leq\non(\mathcal I)$. Indeed, this inequality was proved by sending any $A\notin\mathcal I$ to the family of all $\{a\}$ for $a\in A$.
\end{example}

\begin{example}
  In Lemma~\ref{lem:covn-nonm}, where we showed that $\cov(\Null)\leq\non(\Meager)$, we could have used the following morphism from the $\not\ni$ relation on $\Meager\times2^\omega$ to the $\in$ relation on $2^\omega\times\Null$. Again we let $A$ be a fixed null comeager set.
  \[\begin{cases}\phi(x)=x-A^c\\\psi(x)=x+A\end{cases}
  \]
  To check that it is a morphism, $\phi(x)\not\ni y$ implies $x-y\in A$, which in turn implies that $x\in y+A=\psi(y)$, as required. Recall that the function $\psi(x)=x+A$ featured prominently in our original proof: for each $X\notin\Meager$ we formed the family of translates $\{x+A\mid x\in X\}$.
  
  Once again, if we exchange the roles of $\phi$ and $\psi$ we obtain a morphism witnessing the inequality $\cov(\Meager)\leq\non(\Null)$. Recall that this proof was the same, but using translates of the comeager null set $A^c$. The minus sign in $x-A^c$ has no effect on the argument.
\end{example}

Beyond these two examples, we see that each arrow in Cicho\'n's diagram (Theorem~\ref{thm:cichon}) is potentially dual to the arrow that is positioned diametrically opposite to it. Assuming there is a morphism behind the proof of one of the inequalities, then the other inequality follows automatically from Proposition~\ref{prop:morphism}.

\begin{rem}
  True inequalities between cardinal characteristics are not always witnessed by morphisms. For example, it is true that $\cov(\Meager)\leq\non(\mathcal{SN})$, but it is consistent that the dual inequality fails. [ref]
\end{rem}

\begin{exerc}
  Find a morphism behind the proof of the inequality $\non(\mathcal I)\leq\cof(\mathcal I)$ (Lemma~\ref{lem:diamond}). Check that it is dual to a morphism behind the inequality $\add(\mathcal I)\leq\cov(\mathcal I)$.
\end{exerc}

\begin{exerc}
  Find a morphism behind the proof of the inequality $\mathfrak b\leq\non(\Meager)$ (Lemma~\ref{lem:b-nonm}). Check that it is dual to a morphism behind the inequality $\cov(\Meager)\leq\mathfrak d$.
\end{exerc}

\begin{notes}
  Much of the material of this section is borrowed from Blass, Section~4.
\end{notes}


%%%%%%%%%%%%%%%%%%%%%%%%%%%%%%%%%%%%%%%%%%%%%%%%%%
\section{Category and matching}
%%%%%%%%%%%%%%%%%%%%%%%%%%%%%%%%%%%%%%%%%%%%%%%%%%

In this section we will prove two of the four inequalities remaining in Theorem~\ref{thm:cichon}. We prove $\mathfrak d\leq\cof(\Meager)$ and $\add(\Meager)\leq\mathfrak b$ (actually thanks to duality only one proof is needed). Recall that $\mathfrak d$ is the dominating number of the $\leq^*$ relation on $\omega^\omega$, while $\cof(\Meager)$ is the dominating number of the $\subset$ relation on meager sets. Although the two relations don't seem directly comparable, we now introduce some technology to understand their relationship more clearly. An \emph{interval partition} is a partition $P=(I_n)$ of $\omega$ into finite nonempty intervals $I_n$. We always assume that the $I_n$ is adjacent to $I_{n+1}$, and hence also that they are enumerated in increasing order.

\begin{defn}
  If $x,y\in2^\omega$ and $P$ is an interval partition, we say $x$ \emph{strongly differs} from $y$ on $P$ if for all but finitely many $n$ we have $x\restriction I_n\neq y\restriction I_n$.
\end{defn}

If $x\in2^\omega$ is fixed and $P$ is an interval partition, we define the set $\Diff_P(x)$ to be the set of all $y\in2^\omega$ such that $y$ strongly differs from $x$ on $P$.

\begin{prop}
  \label{prop:meager-diff}
  A subset $A\subset2^\omega$ is meager if and only if there is some $x\in2^\omega$ and interval partition $P$ such that $A\subset\Diff_P(x)$.
\end{prop}

\begin{proof}
  First suppose that $A\subset\Diff_P(x)$, we will show $A$ is meager. For this, let
  \[A_N=\set{y\in2^\omega\mid(\forall n\geq N)\;y\restriction I_n\neq x\restriction I_n}
  \]
  Thus $A\subset\bigcup A_n$, and it is clear that $A_N$ contains no interior. Moreover $A_N$ is a countable intersection of boolean combinations of basic open sets, and hence $A_N$ is closed. Thus we have shown that $A$ is meager.

  Conversely suppose $A=\bigcup A_n$ where each $A_n$ is nowhere dense, and suppose without loss of generality that $A_n\subset A_{n+1}$ for all $n$. We will define an interval partition $P=(I_n)$ and an element $x\in2^\omega$ recursively. To begin, since $A_0$ isn't dense there exists $s\in2^{<\omega}$ such that $A_0\cap V_s=\emptyset$. We let $I_0=\dom(s)$ and $x\restriction I_0=s$.

  Now suppose that $I_i$ and $x\restriction I_i$ have been constructed for $i<n$ and let $s=x\restriction\bigcup_{i<n}I_i$ be the initial segment built so far. Since $A_n$ is nowhere dense, by Proposition~\ref{prop:cantor-space-nwd} we can find an extension $t$ of $s$ such that $A_n\cap V_t=\emptyset$. But we can do even better: we can find an extension $t$ of $s$ such that even if $y\in2^\omega$ only agrees with $t$ on $\dom(t\smallsetminus s)$, then we still have $y\notin A_n$.

  Such a $t$ is built in $k$ many steps: Let $s_0,\ldots,s_{k-1}$ enumerate all elements of $2^{<\omega}$ whose domain is exactly $\dom(s)$. Repeatedly using that $A_n$ is nowhere dense, we can find a sequence of successive extensions $s\subset t_0\subset\ldots\subset t_{k-1}=t$ such that $A_n\cap V_{s_i\cup(t_i\smallsetminus s)}=\emptyset$. We then let $I_n=\dom(t\smallsetminus s)$ and $x_n=t\restriction I_n$. This concludes the construction of $x$ and $P$.

  To see that $A\subset\Diff_P(x)$, observe that whenever $y\restriction I_n=x\restriction I_n$ we have $y\notin A_n$. In other words, whenever $y\in A_n$ we have that $y$ differs from $x$ on $I_n$. Since the $A_n$ are increasing, any $a\in A$ lies in all but finitely many of the $A_n$. Thus we can conclude that any $a\in A$ strongly differs from $x$ on $P$.
\end{proof}

The result implies that the collection of sets $\Diff_P(x)$ forms a basis for the meager ideal on Cantor space. This basis is particularly useful because it is easy to see when one set of the form $\Diff_P(x)$ is a subset of another.

\begin{prop}
  \label{prop:engulf}
  If $P=(I_n)$ and $Q=(J_m)$ are interval partitions, then we have $\Diff_P(x)\subset\Diff_Q(y)$ if and only if for all but finitely many $J_m$ there exists $I_n$ such that $I_n\subset J_m$ and $x\restriction I_n=y\restriction I_n$.
\end{prop}

The proof is elementary (assuming I have not made a typo) and we leave it for Exercise~\ref{exerc:engulf}. Proposition~\ref{prop:engulf} provides characterization of $\cof(\Meager)$ as the dominating number of a simple combinatorial relation on pairs $(P,x)$. The cardinals $\mathfrak b$ and $\mathfrak d$ can be characterized by a similar combinatorial relation just on interval partitions. First, if $P=(I_n)$ and $Q=(J_m)$ are interval partitions, we write $P\prec Q$ if for all but finitely many $J_m$ there exists $I_m$ such that $I_n\subset J_m$. 

\begin{prop}
  There is a morphism from $\prec$ to $\leq^*$, and there is a morphism from $\leq^*$ to $\prec$. Hence $\mathfrak d$ is equal to the dominating number of the $\prec$ relation on interval partitions, and $\mathfrak b$ is the dominating number of the $\not\succ$ relation on interval partitions.
\end{prop}

\begin{proof}
  We first construct a morphism from $\prec$ to $\leq^*$. Given any interval partition $P$ we define the element $\psi(P)\in\omega^\omega$ by $\psi(P)(n)=$ the right endpoint of the interval after the one containing $n$. Moreover for any function $f\in\omega^\omega$ define an interval partition $\phi(f)$ consisting of intervals $[a,b]$ such that $\max\{f(n)\mid n<a\}\leq b$.

  Now, if $\phi(f)\prec P=(I_m)$, we must verify that $f\leq^*\psi(P)$. Indeed, given any $n$ we may find $I_m$ such that $n\in I_m$. Then if $n$ is large enough, since $\phi(f)\prec P$ we know there exists an interval $J\in\phi(f)$ such that $J\subset I_{n+1}$. Then by definition of $\phi$ we have $f(n)\leq\max J$. And by definition of $\psi$ we have $\max J\leq\psi(P)(n)$. Hence $f(n)\leq\psi(P)(n)$, as desired.

  The morphism from $\prec$ to $\leq^*$ is the same, but with the roles of $\phi$ and $\psi$ reversed. We leave it to the reader to check that this is indeed the case.
\end{proof}

\begin{lem}
  We have $\add(\Meager)\leq\mathfrak b$ and $\mathfrak d\leq\cof(\Meager)$.
\end{lem}

\begin{proof}
  We prove the second inequality by finding a morphism from the $\subset$ relation on $\Meager$ to the $\prec$ relation on interval partitions. We therefore obtain the first inequality as a consequence of duality. For this it is enough to consider the $\subset$ relation just on the basis of meager sets of the form $A=\Diff_Q(y)$. We let:
  \[\begin{cases}\phi(P)=\Diff_P(0)\\\psi(\Diff_Q(y))=Q\end{cases}
  \]
  Then by Proposition~\ref{prop:engulf}, $\Diff_P(0)\subset\Diff_Q(y)$ in particular implies that $P\prec Q$, and hence $\phi(P)\subset A\implies P\prec\psi(A)$, as desired.
\end{proof}

We conclude this section with a further result that gives much more information about the relationship between $\add(\Meager)$ and the other cardinals.

\begin{thm}
  We have $\add(\Meager)=\min\{\mathfrak b,\cov(\Meager)\}$.
\end{thm}

\begin{proof}
  We have already shown that $\add(\Meager)\leq\mathfrak b$ and $\add(\Meager)\leq\cov(\Meager)$. Hence it remains only to show that $\add(\Meager)\geq\min\{\mathfrak b,\cov(\Meager)\}$. For this we must show that if $\mathcal F$ is a family of meager subsets of $\omega^\omega$ such that $|\mathcal F|<\min\{\mathfrak b,\cov(\Meager)\}$ then $\bigcup\mathcal F$ is meager. Without loss of generality, we can suppose that $\mathcal F$ consists of closed nowhere dense sets.

  We begin by finding a countable dense subset $Q\subset\omega^\omega$ disjoint from $\bigcup\mathcal F$. Indeed, since $|\mathcal F|<\cov(\Meager)$ we have that $\bigcup\mathcal F\neq\omega^\omega$. In fact for any basic open set $V_s$ we must have that $\bigcup\mathcal F$ does not cover $I$, since otherwise we could use Exercise~\ref{exerc:homeo} to find a countable family of translates of $\bigcup(\mathcal F)$ which covers all oof $\omega^\omega$. This shows $\bigcup\mathcal F$ is co-dense, and therefore we can find $Q$ as required.

  Now enumerate $Q=(q_n)_{n\in\NN}$. Then for each $F\in\mathcal F$ and $n\in\NN$, since $F$ is nowhere dense, we can find a finite initial segment $q_n\restriction g_F(n)$ such that $V_{q_n\restriction g_F(n)}$ is disjoint from $F$. Since $|\mathcal F|<\mathfrak b$, the family of functions $\set{g_F\mid F\in\mathcal F}$ is bounded by a single function $h\in\omega^\omega$. Finally we use this $h$ to define the set:
  \[R=\bigcap_N\bigcup_{n\geq N}V_{q_n\restriction h(n)}
  \]
  Then $R$ is a countable intersection of dense open sets, and hence it is comeager. Moreover since for all $F\in\mathcal F$ we have $g_F\leq^*h$, we have that $R$ is disjoint from $\bigcup\mathcal F$. We can thus conclude that $\bigcup\mathcal F$ is meager, as desired.
\end{proof}

\begin{rem}
  Although the last result does not apparently involve a morphism, it is true that an analogous argument shows the dual fact that $\add(\Meager)=\max\{\mathfrak d,\non(\Meager)\}$. This three-term duality is also part of an abstract morphism-like framework, which we omit. [ref]
\end{rem}

\begin{exerc}
  \label{exerc:engulf}
  Prove Proposition~\ref{prop:engulf}.
\end{exerc}

\begin{exerc}
  \label{exerc:homeo}
  Let $n\in\NN$ and $s,s'\in\omega^n$. Show that there exists a homeomorphism $\phi\colon\omega^\omega\to\omega^\omega$ such that $s\subset x$ iff $s'\subset\phi(x)$.
\end{exerc}

% blass sec4
% blass sec5
% bartoszynski sec 2.2


%%%%%%%%%%%%%%%%%%%%%%%%%%%%%%%%%%%%%%%%%%%%%%%%%%
\section{Bartoszy\'nski's theorem}
%%%%%%%%%%%%%%%%%%%%%%%%%%%%%%%%%%%%%%%%%%%%%%%%%%

In this section we complete the proof of the two inequalities remaining in Cicho'n's diagram: $\add(\Null)\leq\add(\Meager)$ and $\cof(\Meager)\leq\cof(\Null)$. By duality, the two inequalities are really one that was discovered independently by Bartoszy\'nski and Raissonier--Stern. The proof we give is mostly due to Bartoszy\'nski, and is markedly more difficult than our previous results. In fact I will need your help proving two of the preparatory facts (see the exercises).

We begin by introducing a new relation that sits between $\cof(\Null)$ and $\cof(\Meager)$. A \emph{slalom} is a sequence $S=(I_n)$ of subsets $I_n\subset\omega$ such that $|I_n|\leq2^n$. We write $SL$ for the set of slaloms. An element $f\in\omega^\omega$ is said to \emph{ski through} $S$, which we denote by $f\in^*S$, if for almost every $n$ we have $f(n)\in I_n$.

\begin{lem}
  \label{lem:null-slalom}
  There is a morphism from the $\subset$ relation on null subsets of $2^\omega$ to the $\in^*$ relation on $\omega^\omega\times SL$.
\end{lem}

Before the proof, we will need the following fact about null sets, which we leave for Exercise~\ref{exerc:self-supporting}.

\begin{prop}
  \label{prop:self-supporting}
  If $A\subset2^\omega$ is a null set, then there exists a closed subset $K\subset2^\omega\smallsetminus A$ with the property that whenever $V_s\cap K\neq\emptyset$ we have $V_s\cap K$ is nonnull.
\end{prop}

\begin{proof}[Proof of Lemma~\ref{lem:null-slalom}]
  We must define $\phi\colon\omega^\omega\to\Null$ and $\psi\colon\Null\to SL$ such that for all $f\in\omega^\omega$ and null sets $A\subset2^\omega$ we have $\phi(f)\subset A$ implies $f\in^*\psi(A)$.

  To define $\phi$, we start with a sequence of open subsets $E_{n,i}\subset2^\omega$ such that $m(E_{n,i})=2^{-n}$, and given $f\in\omega^\omega$ we let
  \[\phi(f)=\bigcap_N\bigcup_{n\geq N} E_{n,f(n)}\;.
  \]
  Then it is clear that $\phi(f)$ is a null set. For technical reasons later on, we will need to assume that the sets $E_{n,i}$ are mutually independent events (the measure of the intersection is the product of the measures). It is not difficult to write down such $E_{n,i}$ explicitly.

  Before defining $\psi$, given a null set $A\subset2^\omega$ let us preview what properties we will need of the slalom $\psi(A)$. So suppose that $f$ is given and $\phi(f)\subset A$. Let $K$ be chosen as in Proposition~\ref{prop:self-supporting}, so that we have $K\cap\bigcap_N\bigcup_{n\geq N}E_{n,f(n)}=\emptyset$. Since $K$ is complete and the sets $\bigcup_{n\geq N}E_{n,f(n)}$ are open, the Baire category theorem implies that at least one of them is non-dense in $K$. Thus there is a basic open set $V_s$ such that $V_s\cap K\neq\emptyset$ and eventually we have $V_s\cap K\cap E_{n,f(n)}=\emptyset$. We will define the slalom $\psi(A)=(I_n)$ so that for $n$ large enough, $I_n$ it includes all such $f(n)$. This will guarantee that $f\in^*\psi(A)$ as required.

  As a first approximation to this, whenever $V_s\cap K\neq\emptyset$ we begin by letting
  \[I_n(s)=\set{i\in\omega\mid V_s\cap K\cap E_{n,i}=\emptyset}
  \]
  and otherwise we simply let $I_n(s)=\emptyset$. We now have two issues to deal with:
  \begin{itemize}
  \item We need a single slalom $(I_n)$ that handles all $s$ at once.
  \item The $I_n(s)$ don't necessarily satisfy $|I_n(s)|\leq2^n$; we need $|I_n|\leq2^n$.
  \end{itemize}

  Tackling the second issue first, we will show that the sets $|I_n(s)|$ are not too large. More specifically we claim that the sum $\sum2^{-n}|I_n(s)|$ converges. Indeed, by the independence of the sets $E_{n,i}$ we have
  \[m(V_s\cap K)\leq\prod_{n\in\NN,i\in I_n(s)}m((E_{n,i})^c)
  =\prod_{n\in\NN}(1-2^{-n})^{|I_n(s)|}\;.
  \]
  By our choice of $K$, the left-hand side is positive. Hence the right-hand side is positive too and taking a logarithm we obtain
  \[\sum-\log(1-2^{-n})|I_n(s)|<\infty
  \]
  From the Taylor series expansion we always have $-\log(1-x)>x$, and the claim now follows.

  To deal with the first issue we construct a single slalom $I_n$ such that for all $s$ we eventually have $I_n(s)\subset I_n$. To do this fix an enumeration $s_1,s_2,\ldots$ of the elements of $2^{<\omega}$. By the convergence shown in the previous paragraph we can find $k(s_j)$ such that for all $n\geq k(s_j)$ we have $2^{-n}|I_n(s_j)|\leq1/2^j$. We then let $I_n=\bigcup\set{I_n(s_j)\mid n\geq k(s_j)}$. Then clearly
  \[|I_n|\leq\sum_{k(s_j)\leq n}|I_n(s_j)|\leq\sum_{j}2^n/2^j\leq2^n\;.
  \]
  Thus $\psi(A)=(I_n)$ is a slalom, and this completes the construction of $\psi$.

  To quickly recapitulate why this works, again suppose that $\phi(f)\subset A$. By the above reasoning we can find $V_s$ such that $V_s\cap K\neq\emptyset$ and for almost all $n$, $V_s\cap K\cap E_{n,f(n)}=\emptyset$. Thus for $n$ large enough we have both $f(n)\in I_n(s)$ and $I_n(s)\subset I_n$. It follows that $f\in^*\psi(A)$.
\end{proof}

\begin{rem}
  Although we don't need it, there is also a morphism the other way, that is, from $\in^*$ to the $\subset$ relation on $\Null$. Thus the ski-through relation on slaloms can be used to give a complete combinatorial characterization of both $\add(\Null)$ and $\cof(\Null)$.
\end{rem}

\begin{lem}
  \label{lem:slalom-meager}
  There is a morphism from the $\in^*$ relation on $\omega^\omega\times SL$ to the $\subset$ relation on meager subsets of $2^\omega$.
\end{lem}

This time we will need a topological fact before the proof, and once again we leave it for Exercise~\ref{exerc:pi-base-meager}.

\begin{prop}
  \label{prop:pi-base-meager}
  Given $n\in\NN$ and an open subset $O\subset2^\omega$, there is a countable family $\mathcal U_n$ of open subsets of $O$ with the properties:
  \begin{itemize}
  \item Whenever $U_1,\ldots,U_{2^n}\in\mathcal U_n$ we have $U_1\cap\cdots\cap U_{2^n}\neq\emptyset$.
  \item Every nowhere dense set is disjoint from some set in $\mathcal U_n$.
  \end{itemize}
\end{prop}

\begin{proof}[Proof of Lemma~\ref{lem:slalom-meager}]
  We must define $\phi\colon\Meager\to\omega^\omega$ and $\psi\colon SL\to\Meager$ such that whenever $\phi(A)\in^*S$ we have $A\subset\psi(S)$.

  To construct $\psi$, first let $s_1,s_2,\ldots$ be an enumeration of $2^{<\omega}$, and for each $n$ apply Proposition~\ref{prop:pi-base-meager} to $\mathcal O=V_{s_n}$ to obtain a family $\mathcal U_n=\{U_{n,i}\}$. Now if $S=(I_n)$ is a slalom we let
  \[\psi(S)=2^\omega\smallsetminus\bigcap_N\bigcup_{n\geq N}\bigcap_{i\in I_n}U_{n,i}\;.
  \]
  Then by the first property of the $\mathcal U_n$, for all $n$ we have that $\bigcap_{i\in I_n}U_{n,i}$ is a nonempty subset of $V_{s_n}$. It follows that each set $\bigcup_{n\geq N}\bigcap_{i\in I_n}U_{n,i}$ is dense and open, and hence that $\psi(S)$ is meager.

  To define $\phi$, let $A$ be a meager subset of $2^\omega$ and write $A$ as an increasing union of nowhere dense sets $A=\bigcup A_n$. By the second property of the $\mathcal U_n$, for all $n$ there exists $i$ such that $A_n\cap U_{n,i}=\emptyset$. We let $\phi(A)(i)=$ this value $i$.

  Finally let $A$ and $S=(I_n)$ be given and suppose that $\phi(A)\in^*S$. Then for $n$ large enough we have that $I_n$ contains an element $i$ with $A_n\cap U_{n,i}=\emptyset$. Thus for $N$ large enough we have that $A_n$ is disjoint from $\bigcup_{n\geq N}\bigcap_{i\in I_n}U_{n,i}$. It follows that $A_n\subset\psi(S)$, and hence $A\subset\psi(S)$ as desired.
\end{proof}

\begin{cor}
  We have $\add(\Null)\leq\add(\Meager)$ and $\cof(\Meager)\leq\cof(\Null)$.
\end{cor}

This concludes the proof of Bartoszy\'nski's theorem, as well as Theorem~\ref{thm:cichon}.

\begin{exerc}
  \label{exerc:self-supporting}
  Prove Proposition~\ref{prop:self-supporting}. [Hint: First show that there exists a closed subset $C\subset2^\omega\smallsetminus A$ which is nonnull. Let $D$ be the union of all basic open sets $V_s$ such that $m(V_s\cap C)=0$, and show that $m(D)=0$. Finally let $K=C\smallsetminus D$, and show that $K$ has the desired properties.]
\end{exerc}

\begin{exerc}
  \label{exerc:pi-base-meager}
  Prove Proposition~\ref{prop:pi-base-meager}. [Hint: Wlog $O=2^\omega$ itself, and wlog we can replace $2^n$ with $n$. Let $\mathcal U_1$ be an open basis, and show that $\mathcal U_1$ is as desired. If $\mathcal U_n$ has been defined, enumerate it as $V_1,V_2,\ldots$. For each $V_i$ consider all the clopen sets $V$ with the property:
  \begin{itemize}
  \item for all $J\subset i$, if $\bigcap_{j\in J}V_j\neq\emptyset$ then we have $V\cap\bigcap_{j\in\ J}V_j\neq\emptyset$.
  \end{itemize}
  Then let $\mathcal U_{n+1}$ be the collection of $V_i\cup V$ for all $i$ and $V$ with this property. Once again verify that any nowhere dense set is disjoint from an element of $\mathcal U_{n+1}$ and that the intersection of any $n+1$ elements of $\mathcal U_{n+1}$ is nonempty. It may help to start with the special case $n=2$.]
\end{exerc}

% fremlin, measure theory v5p1 522P
% bartoszynski, handbook

\chapter*{Part II: Le tour de forcing}

%%%%%%%%%%%%%%%%%%%%%%%%%%%%%%%%%%%%%%%%%%%%%%%%%%
\section{The idea of forcing}
%%%%%%%%%%%%%%%%%%%%%%%%%%%%%%%%%%%%%%%%%%%%%%%%%%

In this part, we will assume familiarity with the notions of statement, theory, and model. As a reminder, a \emph{formula} is simply a mathematical assertion that can be expressed using the ordinary logical symbols (and, or, not, quantifiers, variables) and non-logical symbols (in our case, $\in$, $\cup$, $\mathcal P$, etc). A \emph{sentence} is a formula in which every variable is quantified, and a \emph{theory} is a collection of sentences. If $T$ is a theory, then a \emph{model} of $T$ is a set, together with interpretations of the non-logical symbols of $T$, in which all the sentences in $T$ hold true.

We are interested in the theory \emph{ZFC}, which consists of the standard axioms for set theory, that govern how the $\in$ relation behaves. G\"odel's incompleteness theorem implies that for some sentences $\phi$, the models of ZFC do not all agree about whether $\phi$ is true or false. A \emph{theorem} is a sentence that holds true in every model of ZFC. A sentence is \emph{independent} of ZFC if it holds true in some models, and false in others. One example that we have mentioned already is CH, the sentence which says that $\mathfrak c=\aleph_1$.

Forcing is a method of adding new sets to a given model $V$ of ZFC, to obtain a larger model $W$ of ZFC. By carefully choosing \emph{how} we add the new sets, we may be able to \emph{force} a given statement to hold true in $W$. For example if $\mathcal F\subset\omega^\omega$, it is possible to add a new element $g\in\omega^\omega$ that is a bound for $\mathcal F$, even if $\mathcal F$ was unbounded to begin with. This is a powerful move, and when iterated it can have an effect on the value of the cardinal $\mathfrak b$.

But forcing cannot do just anything. For a very simple example, if $A\neq\emptyset$, then $A$ will remain nonempty after forcing. There is some element $a\in A$ and this element will remain in $A$ forever. More generally, we will see that a wide variety of statements are \emph{absolute}, meaning that their truth value cannot be changed by forcing. The Riemann hypothesis is an example of such a statement.

The concept of absoluteness highlights the distinction between ordinals and cardinals. Recall that an ordinal is a set which is transitive and well-ordered by the $\in$ relation. Both of these properties are absolute, and therefore forcing can never add or change ordinals. On the other hand, $\aleph_1$ is defined to be the least ordinal that cannot be mapped injectively into $\omega$. So $\aleph_1$ is equal to some ordinal $\alpha$, but this ordinal value can be raised by a forcing which adds a new injection $\alpha\to\omega$!

The situtation is even more fluid with an indirectly-defined cardinal such as $\mathfrak c$. Recall that $\mathfrak c$ is the unique cardinal number $\aleph_\alpha$ that is in bijection with $\mathcal P(\omega)$. This value can be raised by a forcing which adds at least $\aleph_{\alpha+1}$ many new subsets of $\omega$. The value can also be shrunk by a forcing which adds a new bijection between $\aleph_1$ and the old $\aleph_\alpha$ (while still preserving $\aleph_1$). Roughly speaking, this is how we will establish the independence of the continuum hypothesis.

For the rest of this section we introduce the basic combinatorial objects used in forcing. The main object is very simple: a \emph{forcing notion} is simply a partial order $\PP$ with a maximum element $\mathbb1$ (also written $\mathbb1_\PP$). When we force with $\PP$, we add a new subset of $\PP$ to the universe. We think of the elements of $\PP$ as imposing conditions on the new set to be added. If $p,q\in\PP$ we say that $p$ is \emph{stronger} than $q$ if $p\leq q$; lower elements hold more information.

When we force with $\PP$, the partial order $\PP$ can give fine control over the new subset that we add.

\begin{defn}
  A subset $G\subset\PP$ is a \emph{filter} if it satisfies the two properties:
  \begin{itemize}
  \item upwards closed: if $p\in G$ and $p\leq q$ then $q\in G$.
  \item downwards directed: if $p,q\in G$ then there exists $r\in G$ such that $r\leq p,q$.
  \end{itemize}
\end{defn}

Elements $p,q$ are said to be \emph{compatible} if there exists $r$ such that $r\leq p,q$. Thus a filter is simply a set of pairwise compatible elements, closed upwards for convenience.

\begin{example}
  Let $T\subset\omega^{<\omega}$ be a tree. Then $T$ can be made into a notion of forcing by turning it upside-down, that is, setting $s\leq t$ if and only if $s\supset t$. Since $T$ is a tree, this partial ordering has a maximal element $\mathbb1=$ the empty sequence.

  In an upside-down tree such as $T$, the only way for two elements $s,t$ to be compatible is to actually be \emph{comparable}, that is, $s\subset t$ or $t\subset s$. Thus if $G\subset T$ is a filter, then the elements of $G$ actually form a path through $T$.
\end{example}

We encourage the reader to note the distinction between the two words comparable and compatible: pairwise comparable sets form a chain while pairwise compatible sets form a filter. Compatibility is what is needed in direct limit constructions. In a tree, the two concepts end up the same.

\begin{defn}
  A subset $D\subset\PP$ is \emph{dense} if for all $p\in\PP$ there exists $q\in D$ such that $q\leq D$.
\end{defn}

The use of the word ``dense'' here may seem in conflict  with the use of the word in topology, but the two can be connected. If $\PP$ is a partial order then it has a topology with a basis consisting of the downward cones $V_p=\set{q\in\PP\mid q\leq p}$. A subset $D\subset\PP$ is dense if and only if it is dense in this topology.

We think of the dense subsets of $\PP$ as omnipresent---almost every element of $\PP$ lies below some element of $D$. When we force with $\PP$ we add a filter $G$ that is \emph{generic} in the sense that it meets these dense subsets of $\PP$. The more dense sets that we demand that $G$ meets, the stronger a condition this imposes on $G$.

For example, if just countably many dense sets are given, then we will see in the next section that there is always a filter that meets them all. On the other extreme, we could say that a filter is fully generic if it meets \emph{every} dense set of $\PP$, but the next result shows that this is too strong.

\begin{prop}
  Suppose that for every $p\in\PP$, there exist incompatible elements $q,r\leq p$. Then $\PP$ does not have a fully generic filter.
\end{prop}

\begin{proof}
  It is enough to show that if $G\subset\PP$ is a filter then its complement $D=\PP\smallsetminus G$ is dense. For this, let $p\in\PP$ be arbitrary and by the hypothesis on $\PP$ let $q,r$ be incompatible elements such that $q,r\leq p$. Since $G$ is a filter, it cannot contain both $q$ and $r$. Thus $D$ contains at least one of them, and this shows $D$ is dense.
\end{proof}

A partial order $\PP$ satisfying the hypothesis of the proposition is called \emph{atomless}. The condition ensures that forcing with $\PP$ is nontrivial. Once we acknowledge that fully generic filters do not exist in the given universe of set theory, we arrive at the following definition.

\begin{defn}
  If $V$ is a model of set theory and $\PP\in V$, we say that $G$ is \emph{$V$-generic} if for every $D\in V$ such that $D\subset\PP$ is dense, we have $G\cap D\neq\emptyset$.
\end{defn}

When we make forcing rigorous, we will show that under appropriate hypotheses, a $V$-generic filter $G$ can be found outside of $V$. We will then let $W=V[G]$ be the smallest model containing both $V$ and $G$, and call this the forcing extension with respect to the notion $\PP$.

\begin{example}
  Returning to the example of the forcing notion $\PP=\omega^{<\omega}$ with the upside-down ordering, we have already said that a filter $G\subset\omega^\omega$ is actually a path through $\omega^\omega$. If $G$ is $V$-generic, then much more can be said of this path. First of all, consider the sets
  \[D_n=\set{t\in\omega^{<\omega}\mid \dom(t)\geq n}\;.
  \]
  Clearly $D_n$ is dense, since any $s\in\omega^{<\omega}$ can be extended to an element with length at least $n$ (recall we are using the upside-down ordering). It follows that $G$ meets $D_n$ for all $n$, and hence that $G$ is an infinite path. Second, if $G$ is $V$-generic then we have already seen that $G\notin V$. Thus the union $g=\bigcup G$ is an element of $\omega^{<\omega}$ that wasn't there before, that is, $g$ is a new element of the Baire space! We can do the same with $2^{<\omega}$ and Cantor space.
\end{example}

This example brings us back to our discussion of forcing CH to be false in a forcing extension. We have just seen how to add one new element to $\omega^\omega$; if we were to repeat the process we can add many new elements to $\omega^\omega$.

% Marker for basic logic and models
% Kunen, IV.1 for forcing defs


%%%%%%%%%%%%%%%%%%%%%%%%%%%%%%%%%%%%%%%%%%%%%%%%%% 
\section{Martin's axiom}
%%%%%%%%%%%%%%%%%%%%%%%%%%%%%%%%%%%%%%%%%%%%%%%%%%

We will make the ideas of the previous section more rigorous shortly. But before going into the technical details of this, we will introduce the combinatorics of forcing using forcing itself as a black box. Specifically, we will appeal to a statement called Martin's axiom, which roughly speaking says that the current universe of set theory $V$ already contains many highly generic filters $G$. The consistency of MA is itself proved using forcing, but we shall take this for granted for now. This will allow us to explore the effects of forcing using a number of partial orders without ever leaving $V$.

There is unfortunately one major limitation on the variety of partial orders we will be able to explore using Martin's axiom. A subset $A\subset\PP$ is said to be an \emph{antichain} if for all $p,q\in A$ we have $p$ and $q$ are not compatible. Thus if $A$ is an antichain, a filter $G$ can only choose at most \emph{one} of the elements of $A$.

\begin{defn}
  A partial order $\PP$ is said to be \emph{ccc} (or satisfy the \emph{countable chain condition}) if every antichain $A\subset\PP$ is countable.
\end{defn}

In some sense the ccc imposes a limitation on the number of different filters that may be added when forcing with $\PP$. The partial order $\omega^{<\omega}$ considered in the previous section is obviously ccc because it is countable. However, there are many examples of ccc partial orders which are not countable.

\begin{example}
  \label{ex:cohen-ccc}
  Consider the partial order $\PP$ consisting of all nonempty open subsets of $\RR$, with the ordering $A\leq B$ iff $A\subset B$. Then $\PP$ is ccc: indeed, two sets are incompatible in this ordering if and only if they are disjoint. Thus an antichain is a family of pairwise disjoint open sets, and since $\RR$ is separable, such a family must be countable.
\end{example}

\begin{example}
  Consider the partial order $\PP=\omega_1^{<\omega}$ with the usual upside-down tree ordering. Then this $\PP$ is not ccc. Indeed, the set of sequences $s\in\PP$ of length $1$ is an uncountable antichain in $\PP$.
\end{example}

\begin{defn}
  \emph{Martin's axiom} (or just \emph{MA}) is the statement: For every ccc partial order $\PP$ and every family $\mathcal F$ of fewer than continuum many dense subsets of $\PP$ there exists an $\mathcal F$-generic filter $G\subset\PP$.
\end{defn}

We sometimes consider the refinements of MA defined as follows. The axiom MA$_\kappa$ is the same statement as MA, with $|\mathcal F|<\mathfrak c$ replaced by $|\mathcal F|=\kappa$.

\begin{prop}
  \label{prop:ma-aleph0}
  MA$_{\aleph_0}$ is true. Thus if CH holds, then MA is true.
\end{prop}

\begin{proof}
  Let $\PP$ be any partial order and let $D_0,D_1,\ldots$ be a countable family of dense subsets of $\PP$. Fix $p_0\in D_0$, and inductively let $p_{n+1}\leq p_n$ be an element of $D_{n+1}$. Then the set $\{p_n\}$ is downwards directed, and hence closing it upwards generates a filter $G$ which meets each of the $D_i$.
\end{proof}

In fact, in this proof we did not even need $\PP$ to be ccc. The next result states that MA is also consistent with a larger continuum. For now we take this fact for granted and use it to derive several applications, as an introduction to the combinatorics of forcing. 

\begin{thm}
  There is a model $V$ of set theory which satisfies both MA and $\neg$CH.
\end{thm}

We say that a set of axioms is \emph{consistent}, or consistent with ZFC, if there is a model of ZFC which satisfying those axioms. Thus the above result can be rephrased: MA+$\neg$CH is consistent.

The axiom MA packages a great deal of forcing into one universe $V$. When MA is true, it is as if somebody has already performed forcing with all of the ccc partial orders. In fact later we shall see that the consistency of MA can be established by repeatedly applying the forcing construction.

\begin{prop}
  In the definition of MA, we cannot in general drop the condition that the family of dense sets $\mathcal F$ have size less than continuum.
\end{prop}

\begin{proof}
  We again go to our favorite partial ordering $\omega^{<\omega}$ with the upside-down order. This time we consider the sets:
  \[D_f=\set{s\in\omega^{<\omega}\mid s\not\subset f}
  \]
  for all $f\in\omega^\omega$. It is easy to see that $D_f$ is dense: any sequence $s$ can be extended to a sequence $t$ that disagrees with $f$. Now if MA were to hold with respect to families of dense sets of size continuum, then there would be a filter $G$ that meets $D_f$ for \emph{every} $f\in\omega^\omega$. It follows from this that the function $g=\bigcup G$ is an element of $\omega^\omega$ which disagrees with every $f\in\omega^\omega$, a contradiction.
\end{proof}

\begin{prop}
  In the definition of MA, we cannot in general drop the condition that $\PP$ is ccc.
\end{prop}

\begin{proof}
  This time we let use the partial order $\PP=\omega_1^{<\omega}$ with the upside-down order. We have already seen that $\PP$ is not ccc. Now we consider the sets
  \[D_\alpha=\set{s\in\omega_1^{<\omega}\mid\alpha\in\rng(s)}
  \]
  for all $\alpha\in\omega_1$. Then $D_\alpha$ is dense: any $s\in\omega_1^{<\omega}$ can be extended to include $\alpha$ in its range. Now if MA were to hold with respect to this partial order, then there would be a filter $G\subset\omega_1^{<\omega}$ that meets $D_\alpha$ for all $\alpha$. It follows from this that there is an onto function $g=\bigcup G$ from $\omega$ to $\omega_1$, a contradiction.
\end{proof}

We close this section with a result showing the effect of MA on properties of the continuum. Recall that the Baire category theorem states that the set of all real numbers is not meager, that is, $\cov(\Meager)$ is uncountable. It turns out that MA implies $\cov(\Meager)$ is as large is possible.

\begin{thm}
  \label{thm:ma-covm}
  MA implies that the Baire category theorem holds for unions of length less than $\mathfrak c$, that is, $\cov(\Meager)=\mathfrak c$.
\end{thm}

\begin{proof}
  Suppose towards a contradiction that $\mathcal F$ is a family of nowhere dense subsets of $\RR$ of cardinality less than continuum, and that $\bigcup\mathcal F=\RR$. Let $\PP$ be the partial order consisting of all bounded, positive-length intervals in $\RR$, ordered by $I'\leq I$ iff $I'\subset I$. Then $\PP$ is easily seen to be ccc by the same argument as in Example~\ref{ex:cohen-ccc}. For each $A\in\mathcal F$, we consider the set
  \[D_A=\set{I\in\PP\mid \overline{I}\cap A=\emptyset}
  \]
  Then $D_A$ is dense for any $A$: since $A$ is nowhere dense, for any positive-length interval $I$ there is a positive length $J\subset I$ such that $J\cap A=\emptyset$. We may then shrink $J$ further to suppose that $\overline{J}\subset I$.

  Now by MA, we can find a filter $G\subset\PP$ such that $G$ meets $D_A$ for every $A\in\mathcal F$. Then the sets $\overline{I}$ for $I\in G$ have the finite intersection property, and it follows from the completeness of $\RR$ that there is a point $x\in\bigcap\set{\overline{I}\mid I\in G}$. Then if $A\in\mathcal F$ then since $G\cap D_A\neq\emptyset$ we can find a positive-length interval $I$ such that $I\in G$ and $\overline{I}\cap A=\emptyset$. That is, we have $x\in\overline{I}$ and $\overline{I}\cap A=\emptyset$, and so $x\notin A$. We conclude that $x$ is not an element of $\bigcup\mathcal F$, contradicting that $\bigcup\mathcal F=\RR$.
\end{proof}

% \begin{proof}
%   First recall from Proposition~\ref{prop:meager-diff} that a subset $A\subset2^\omega$ is meager if and only if it is contained in a set of the form $\Diff(x,P)$ for $x\in2^\omega$ and $P$ an interval partition. It follows that $\cov(\Meager)$ is the least cardinality of a family $\mathcal F$ of pairs $(y,P)$ such that for all $x\in2^\omega$ there exists $(y,P)\in\mathcal F$ such that $x$ strongly differs from $y$ on $P$.

%   Now suppose towards a contradiction that $\cov(\Meager)<\mathfrak c$, and fix a family $\mathcal F$ of pairs $(y,P)$ as above. We will apply MA to the partial order $\PP=2^{<\omega}$ to obtain a suitably generic filter $G$. Our goal will be to ensure that $G$ matches (that is, doesn't strongly differ from) all of the $(y,P)$ pairs, which will be a contradiction. To this end, consider the subsets
%   \[D_{y,P,n}=\set{s\in2^{<\omega}\mid \text{there is $I\in P$ such that $I>n$ and $s\restriction I=y\restriction I$}}
%   \]
%   Then for any $y$, $P$, and $n$ the set $D_{y,P,n}$ is dense: given any $s$ we can find an interval $I$ of $P$ beyond both $n$ and $\dom(s)$, and extend $s$ to a sequence $t$ which agrees with $y$ on $I$. 

%   Thus by MA, there exists a filter $G\subset\PP$ that meets the sets $D_{y,P,n}$ for all $(y,P)\in\mathcal F$, and we let $g=\bigcup G$ be the associated element of $2^\omega$. Given any $(y,P)\in\mathcal F$, since $G$ meets all dense sets $D_{y,P,n}$ it follows that $g$ does not strongly differ from $y$ on $P$. This contradicts our assumption about the family $\mathcal F$.
% \end{proof}

% kunen, III.2 for ccc
% kunen, III.3 for MA
% jech, Sec16 for covm
% blass, forcing section for a generalization of "solving" forcing


%%%%%%%%%%%%%%%%%%%%%%%%%%%%%%%%%%%%%%%%%%%%%%%%%% 
\section{Martin's axiom and Cicho\'n's diagram}
%%%%%%%%%%%%%%%%%%%%%%%%%%%%%%%%%%%%%%%%%%%%%%%%%%

We have seen the effect of MA on properties of the continuum. We now explore the effect of MA on further cardinal characteristics found in Cicho\'n's diagram. We begin by showing that MA makes $\mathfrak d$ large. Although this already follows from Theorem~\ref{thm:ma-covm} together with all we have proved about Cicho\'n's diagram, here we give a direct proof.

\begin{thm}
  \label{thm:ma-d}
  Assuming MA, we have $\mathfrak d=\mathfrak c$.
\end{thm}

\begin{proof}
  Suppose towards a contradiction that $\mathfrak d<\mathfrak c$, and fix a dominating family $\mathcal F\subset\omega^\omega$. Let $\PP$ be the partial order $\omega^{<\omega}$ with the upside-down ordering. For each $f\in\mathcal F$ we consider the family sets
  \[D_{f,n}=\set{s\in\omega^{<\omega}\mid\text{there is $n'>n$ such that $s(n)>f(n)$}}
  \]
  Observe that $D_{f,n}$ is dense for all $f$ and $n$: if $s\in\omega^{<\omega}$ we can easily find $n'>n,|s|$ and find $t\supset s$ such that $t(n')>f(n)$. Thus by MA we can find a filter $G$ which meets $D_{f,n}$ for all $f\in\mathcal F$ and $n\in\NN$. Letting $g=\bigcup G$, it is clear that $g$ is not dominated by $f$ for any $f\in\mathcal F$, contradicting that $\mathcal F$ is a dominating family.
\end{proof}

The content of the above proof is that if $g$ is fully generic for $\omega^{<\omega}$, then in $V[g]$ none of the elements in the Baire space of $V$ dominate $g$. It falls short, however, of showing that $g$ actually dominates every element in the Baire space of $V$. ith a slightly more specialized partial order, this can be achieved as well.

Given a family $\mathcal F\subset\omega^\omega$, the \emph{Hechler forcing} with respect to $\mathcal F$ is the partial order $\mathbb H_{\mathcal F}$ consisting of elements $(s,F)$ where $s\in\omega^{<\omega}$ and $F\subset\mathcal F$ is finite. We partially order $\mathbb H_{\mathcal F}$ by $(s',F')\leq(s,F)$ if the conditions hold:
\begin{itemize}
\item $s'\supset s$,
\item $F'\supset F$, and
\item $s'(n)>f(n)$ whenever $f\in F$ and $n\in\dom(s'\smallsetminus s)$.
\end{itemize}
If $(s,F)$ is an element of a Hechler forcing, then as usual the left coordinate $s$, called the \emph{stem}, is a finite approximation for a new element of Baire space. The new coordinate $F$ can be thought of as a set of \emph{promises}, which ensures that stronger conditions will dominate the elements of $F$.

Hechler forcing is always ccc, because incompatible conditions cannot have the same stem $s$, and there are only countably many possible stems $s$.

\begin{thm}
  Assuming MA, we have $\mathfrak b=\mathfrak c$.
\end{thm}

\begin{proof}
  Suppose towards a contradiction that $\mathcal F$ is an unbounded family of size $<\mathfrak c$, and let $\PP$ be the Hechler forcing $\mathbb H_{\mathcal F}$. We now consider two families of dense sets. First, for each $n$ we consider the set
  \[D_n=\set{(s,F)\in\mathbb H_{\mathcal F}\mid n\in\dom(s)}
  \]
  To see that $D_n$ is dense, let $(s,F)$ be a given element of $\mathbb H_{\mathcal F}$. If $n\notin\dom(s)$ then we extend the stem $s$ to a sequence $t$ defined by $t(i)=\max_{f\in F}f(i)+1$ for all $i\notin\dom(s)$ such that $i\leq n$. Then clearly $(t,F)\leq(s,F)$ and $(t,F)\in D_n$.

  Next, for each $f\in\mathcal F$ we consider the set
  \[D_f=\set{(s,F)\in\mathbb H_{\mathcal F}\mid f\in F}
  \]
  Then $D_f$ is again dense, since given $(s,F)$ we have $(s,F\cup\{f\})\leq(s,F)$ (Observe that the promises $F$ are trivially satisfied since we have not lengthened the stem!)

  Now by MA we cand find a filter $G$ that meets $D_n$ for all $n\in\NN$ and  $D_f$ for all $f\in\mathcal F$. We then let
  \[g=\bigcup\set{s\mid(\exists F)\;(s,F)\in G}
  \]
  be the element of the Baire space determined by $G$. Given $f\in\mathcal F$ and $n\in\NN$ we first find an element $(s,F)\in G\cap D_f$, and then letting $n'=\max(n,\dom(s))+1$ we find an element $(s',F')\in G\cap D_{n'}$. Since $G$ is a filter we can find $(s'',F'')\in G$ which is stronger than both. Then by the definition of the ordering on $\mathbb H_{\mathcal F}$, there is some $n''>n$ such that $s''(n'')>f(n'')$ and hence $g(n'')>f(n'')$. Since $f$ and $n$ were arbitrary, it follows that $f\leq^*g$ for all $f\in\mathcal F$, contradicting that $\mathcal F$ is unbounded.
\end{proof}

We close this section by finding the effect of MA on the cardinal $\add(\Null)$. As we will see, MA once again implies that this cardinal has its maximum value of $\mathfrak c$. By the inequalities shown in Theorem~\ref{thm:cichon} and especially Bartoszy\'nski's theorem, this implies that assuming MA, \emph{all} the cardinal characteristics in Cicho\'n\'s diagram have the value $\mathfrak c$. At face value, this obviates the last three results which showed that individual characteristics are equal to $\mathfrak c$. But those arguments were more direct, and moreover introduced us to several forcings which will remain important outside of the MA context.

\begin{thm}
  \label{thm:ma-addn}
  Assuming MA, we have $\add(\Null)=\mathfrak c$. In particular, all the cardinal characteristics in Cicho\'n's diagram (except possibly for $\aleph_1$) have the same value.
\end{thm}

For this proof, we once again introduce a new and special notion of forcing. The \emph{amoeba} partial ordering $\mathbb A_\epsilon$ with parameter $\epsilon>0$ consists of all open subsets $O\subset\RR$ such that $m(O)<\epsilon$. The ordering on $\mathbb A_\epsilon$ is defined by $O\leq O'$ iff $O\supset O'$. Thus the elements of $\mathbb A_\epsilon$ are inner approximations to an open set of measure $\leq\epsilon$.

\begin{lem}
  For any $\epsilon$, the amoeba partial ordering $\mathbb A_\epsilon$ is ccc.
\end{lem}

\begin{proof}
  Given an uncountable family $\mathcal F\subset\mathbb A_\epsilon$ we must show that at least two elements $O,O'\in\mathcal F$ are compatible, that is, $m(O\cup O')<\epsilon$. We first give ourselves a little wiggle room as follows: since $\mathcal F$ is uncountable and there are only countably many values of $n\in\NN$, by the pigeon-hole principle we can find a fixed $n$ and an uncountable subset $\mathcal F_0\subset\mathcal F$ such that for all $O\in\mathcal F_0$ we have $m(O)<\epsilon-1/n$.

  Now for each $O\in\mathcal F_0$ we can find an open set $B_O\subset O$ which is a finite union of intervals with rational endpoints and such that $m(O\smallsetminus B_O)<1/2n$. Once again, since there are only countably many possible sets $B_O$, we can find a fixed set $B$ and an uncountable $\mathcal F_{00}\subset\mathcal F$ such that $B=B_O$ for every $O\in\mathcal F_{00}$.

  We claim that any two elements $O,O'\mathcal F_{00}$ are compatible. Indeed, we compute:
  \begin{align*}
    m(O\cup O')&=m(B_O\cup(O\smallsetminus B_O)\cup B_{O'}\cup(O'\smallsetminus B_{O'}))\\
    &=m(B\cup(O\smallsetminus B_O)\cup(O'\smallsetminus B_{O'}))\\
    &\leq m(B)+2(1/2n)\\
    &<(\epsilon-1/n)+1/n\\
    &=\epsilon
  \end{align*}
  Thus we can conclude that $\mathcal F$ is not an antichain, as desired.
\end{proof}

We are now prepared to prove Theorem~\ref{thm:ma-addn}.

\begin{proof}[Proof of Theorem~\ref{thm:ma-addn}]
  Given a family $\mathcal F$ of fewer than continuum many null sets, we wish to show that $\bigcup\mathcal F$ is again null. For this, it is sufficient to show that for any $\epsilon>0$ we have $\bigcup\mathcal F$ is contained in an open set of measure $\leq\epsilon$.

  For this, let $\epsilon$ be given and let $\mathbb A_\epsilon$ be the corresponding amoeba forcing. For each $A\in\mathcal F$ we consider the set
  \[D_A=\set{O\in\mathbb A_\epsilon\mid A\subset O}
  \]
  To see that $D_A$ is dense, let $O$ be a given open set with $m(O)<\epsilon$. Since $A$ is null we can find an open set $U$ such that $A\subset U$ and $m(U)<\epsilon-m(O)$. Then $m(O\cup U)<\epsilon$ and $A\subset O\cup U$, and so $O\cup U$ is stronger than $O$ and $O\cup U\in D_A$. (Kunen says: The amoeba $O$ reached out a pseudopod $U$ to engulf the morsel of food $A$.)

  Now by MA there is a filter $G\subset\mathbb A_\epsilon$ such that $G$ meets $D_A$ for all $A\in\mathcal F$. Letting $U=\bigcup G$ be the open set determined by $G$, we clearly have that $\bigcup\mathcal F\subset U$. To conclude, we claim that $m(U)\leq\epsilon$. For this, since $\mathbb R$ is second countable we can find countably many elements $O_1,O_2,\cdots\in G$ such that $U=\bigcup O_n$. Since measure is continuous from below (similar to Exercise~\ref{exerc:continuity-above}) for all $\delta>0$ there exists some $N$ such that $m(O_1\cup\cdots\cup O_N)\geq m(U)-\delta$. Since $G$ is downwards-directed we inductively have $O_1\cup\cdots\cup O_n\in G$, which implies that $m(O_1\cup\cdots\cup O_N)<\epsilon$. It follows that $m(U)\leq\epsilon+\delta$, and since $\delta$ was arbitrary we have $m(U)\leq\epsilon$, as desired.
\end{proof}

\begin{exerc}
  Suppose that $\mathcal F$ is an \emph{almost disjoint} family of subsets of $\omega$: for all $A,A'\in\mathcal F$ we have that $A\cap A'$ is finite. Show that if $|\mathcal F|<\mathfrak c$ then there exists a single set $B$ such that $A\cap B$ is finite for all $A\in\mathcal F$. [Hint: consider the forcing $\PP$ consisting of pairs $(s,F)$ where $s$ is a finite subset of $\omega$, $F$ is a finite subset of $\mathcal F$, and $(s',F')\leq(s,F)$ iff $s'\supset s$, $F'\supset F$, and whenever $A\in\mathcal F$ we have $A\cap s'\subset s$.]
\end{exerc}

% blass for "solving" posets
% jech, Sec16 for hechler
% kunen, III.3 for amoeba


%%%%%%%%%%%%%%%%%%%%%%%%%%%%%%%%%%%%%%%%%%%%%%%%%% 
\section{The machinery of forcing}
%%%%%%%%%%%%%%%%%%%%%%%%%%%%%%%%%%%%%%%%%%%%%%%%%%

So far we have seen forcing only in the context of appeals to Martin's axiom, which gives us highly generic filters $G$ in a single model $V$ of set theory. When we actually do forcing, we wish to find a fully generic filter $G$, even though we know this must lie outside of $V$. But what does it mean that a set $G$ lies outside of the universe of sets? In fact, to find ``new'' sets $G$ we will really assume that $V$ is a countable, transitive model of ZFC. In other words $V$ is a countable set lying in some larger universe $W$, $V$ it is closed downwards under $\in$, and the structure $(V,\in)$ satisfies all of the axioms of set theory.

The mild assumption that there is such a $V$ turns out to be very powerful. First of all, if $V$ is countable then given a partial order $\PP\in V$ we can easily find a filter $G\subset\PP$ that meets all the dense subsets of $\PP$ that actually lie in $V$. After all there are just countably many sets in $V$ and therefore just countably many dense subsets of $\mathbb P$ in $V$. Hence we can use the argument of Proposition~\ref{prop:ma-aleph0} to diagonalize and obtain a $V$-generic filter $G$.

Second of all, if $V$ is transitive then we can use the machinery of absoluteness to reason about the basic properties of $V$ and its forcing extensions $V[G]$. Recall that a formula $\phi$ in the language of set theory is $\Delta_0$ if all of its quantifiers are \emph{bounded}, that is, of the form $\exists x\in y$ or else $\forall x\in y$. We will use the well-known fact that if $\phi$ is a $\Delta_0$ formula with parameters in a transitive set $A$, then $\phi$ is true if and only if $\phi$ holds in $(A,\in)$.

So just how mild is the assumption that there is a countable transitive model of ZFC? Unfortunatley it is not a totally nontrivial one. The countability isn't the problem; recall that by the L\"owenheim--Skolem theorem if there is an infinite model of a theory $T$ then there is a countable one. Rather, the problem is assuming there is a model of ZFC at all. Indeed this is equivalent to assuming that ZFC is \emph{consistent}, that is, ZFC does not entail any contradictions. But we know from G\"odel's incompleteness theorem that it is not possible to establish the consistency of ZFC using ZFC alone.

It is possible to work around this technicality by supposing only that $V$ satisfies a large but finite subsystem of ZFC. By the reflection theorem, it is possible to obtain transitive set models of such a subsystem, and it is not difficult to use such a model to get a countable transitive one. However, for simplicity we will simply assume for the rest of this section that $V$ is a countable, transitive model of all of ZFC.

The first problem we have to deal with when introducing forcing is that if one is working in $V$, then since $G$ is not available it is difficult to refer to elements of $V[G]$ (and hence difficult to prove facts about them). We need the following construction.

\begin{defn}
  If $\PP$ is a partial order, then a set $\tau$ is called a \emph{$\PP$-name} for an element of $V[G]$ if $\tau$ conists of elements $(\sigma,p)$ where $\sigma$ is a \emph{$\PP$}-name and $p\in\PP$.
\end{defn}

Like sets themselves, the definition is recursive. To explain it briefly, we can think of an ordinary set $x$ as a wellfounded tree with $x$ at the root, the elements of $x$ on the first level, the elements of elements of $x$ on the second level, and so on. A $\PP$-name is similar, except the nodes of the tree are also $\PP$-names, and each node is labelled with an element of $\PP$.

If $\tau$ is a $\PP$-name and we do gain access to a generic filter $G$, then we can use $G$ to resolve the name $\tau$ into an honest set as follows. Thinking of $\tau$ as a tree, we include only the nodes labelled with an element $p\in G$.

\begin{defn}
  If $\PP$ is a partial order in $V$, $\tau$ is a $\PP$-name, and $G\subset\PP$ is a filter, then we define $\tau_G$ recursively by
  \[\tau_G=\set{\sigma_G\mid (\exists p)\;(\sigma,p)\in\tau\text{ and }p\in G}
  \]
  We then define the forcing extension of $V$ corresponding to $G$ by
  \[V[G]=\set{\tau_G\mid\tau\in V\text{ is a $\PP$-name}}
  \]
\end{defn}

The definitions of $\PP$-name and of $\tau_G$ are heavily recursive, but one should expect this from such a general set-theoretic construction. In fact, it is amazing that such a short-winded construction actually works. But it will take us some time to prove everything we need.

\begin{lem}
  If $\PP$ is a partial order and $G$ is a filter, then we have $V\subset V[G]$ and $G\in V[G]$. Moreover, $M[G]$ is the smallest transitive model with these two properties.
\end{lem}

\begin{proof}
  To prove the statements, we need only write down explicit $\PP$-names for the objects in question. First recall that we are always assuming $\PP$ has a maximum element $\mathbb 1$. Thus if $x\in V$ we can write the recursive definition
  \[\check x=\set{(\check y,\mathbb 1)\mid y\in x}
  \]
  The idea here is to take the tree of $x$ and label each node with a $\mathbb 1$. To see that $\check x_G=x$, suppose inductively that $\check y_G=y$ for all $y\in x$. Then since $\mathbb 1$ always lies in $G$, the definition implies that $\check x_G=\set{\check y_G\mid y\in x}=\set{y\mid y\in x}=x$.

  To see that $G\in V[G]$ we construct a single name
  \[\Gamma=\set{(\check p,p)\mid p\in\PP}
  \]
  Then by the previous paragraph, we have that $\Gamma_G=\set{p\mid p\in G}=G$, as desired.

  Finally, if $W$ is a model of ZFC which contains $V$ as a submodel and $G$ as an element, we must check that $W$ contains $\tau_G$ for all names $\tau$. For this, $W$ can simply evaluate $\tau_G$ as in its definition, and $W$ will be correct in this evaluation because the definition of $\tau_G$ is $\Delta_0$ and hence absolute to transitive models. [ref]
\end{proof}

Names of the form $\check x$ for $x\in V$ are called \emph{check names}. For convenience we will sometimes identify $x$ with its check name. The name $\Gamma$ is called the \emph{canonical name for the generic}, and we observe that it does not depend on $G$ at all. Still, when a filter $G$ is given we may again identify $G$ with this name.

Now that we can name elements of the forcing extension $V[G]$, we need a way to decide whether a given sentence holds in $V[G]$. But the truth value of a given setence may depend on the filter $G$, just as the set value $\tau_G$ of a name depends on $G$.

\begin{defn}
  Let $\PP$ be a partial order.
  \begin{itemize}
  \item The \emph{forcing language} for $\PP$ consists of the usual logical symbols, $\in$, and the $\PP$-names as constant symbols.
  \item If $\phi$ is a sentence of the forcing language for $\PP$, we say $\phi$ holds in $V[G]$ if it holds with its names $\tau$ replaced by their values $\tau_G$.
  \item If $p\in\PP$ and $\phi$ is a sentence of the forcing language for $\PP$, then we write $p\Vdash\phi$, read $p$ \emph{forces} $\phi$, if $p\in G$ implies $\phi$ holds in $V[G]$.
  \end{itemize}
\end{defn}

The following three results are the fundamental tools used in the development of the theory of forcing. Essentially all further facts about forcing will be derived from these properties of the forcing machinery.

\begin{thm}[The forcing theorems]\
\begin{enumerate}
\item The relation $\Vdash$ is definable in $V$.
\item A sentence $\phi$ is true in $M[G]$ if and only if there exists $p\in G$ such that $p\Vdash\phi$.
\item If $V$ is a model of ZFC then $V[G]$ is a model of ZFC.
\end{enumerate}
\end{thm}


%%%%%%%%%%%%%%%%%%%%%%%%%%%%%%%%%%%%%%%%%%%%%%%%%% 
\section{Preservation of cardinals}
%%%%%%%%%%%%%%%%%%%%%%%%%%%%%%%%%%%%%%%%%%%%%%%%%%

\begin{thm}
  If $\PP$ is ccc, then $\PP$ preserves cardinals.
\end{thm}

\begin{lem}
  Let $\PP$ be a ccc partial order, and $G\subset\PP$ a $V$-generic fliter. Given any function $f\in V[G]$, there exists a function $F\in V$ such that for all $a\in\dom(f)$ we have $F(a)$ is countable and $f(a)\in F(a)$.
\end{lem}

\begin{proof}
  
\end{proof}


% Kunen, sec 4.3

%%%%%%%%%%%%%%%%%%%%%%%%%%%%%%%%%%%%%%%%%%%%%%%%%% 
\section{Cohen and random forcing}
%%%%%%%%%%%%%%%%%%%%%%%%%%%%%%%%%%%%%%%%%%%%%%%%%%

\begin{defn}
  The partial order for \emph{Cohen forcing} $\mathbb C$ consists of all equivalence classes of nonmeager subsets of $2^\omega$, where sets $A,B$ are said to be equivalent if they differ by a meager set. The elements of $\mathbb C$ are ordered by $[A]\leq[B]$ iff $A\smallsetminus B$ is meager.
\end{defn}

\begin{lem}
  The Cohen partial order $\mathbb C$ is ccc.
\end{lem}

\begin{thm}
  The Cohen forcing partial order $\mathbb C$ is forcing equivalent to the partial order $2^{<\omega}$.
\end{thm}

\begin{defn}
  The partial order for \emph{random forcing} $\mathbb B$ consists of all equivalence classes of nonnull subsets of $2^\omega$, where sets $A,B$ are equivalent if they differ by a null set. The elements of $\mathbb B$ are ordered by $[A]\leq[B]$ iff $A\smallsetminus B$ is null.
\end{defn}

\begin{lem}
  The random partial order $\mathbb B$ is ccc.
\end{lem}

\begin{proof}
  In $\mathbb B$, two elements $A,B$ are incompatible if and only if their intersection is null. Thus it suffices to show that any family $\mathcal F$ of pairwise disjoint subsets of $2^\omega$ is countable. Indeed, since $m(2^\omega)=1$, for each $n$ there can be just finitely many elements of $\mathcal F$ of measure $\geq1/n$. It follows that there can be just countably many elements of $\mathcal F$.
\end{proof}

The random partial order does not have such an elementary characterization as the Cohen partial order.


%%%%%%%%%%%%%%%%%%%%%%%%%%%%%%%%%%%%%%%%%%%%%%%%%% 
\section{Multiple Cohen and random forcing}
%%%%%%%%%%%%%%%%%%%%%%%%%%%%%%%%%%%%%%%%%%%%%%%%%%

\begin{defn}
  For any cardinals $\kappa,\lambda$, we define the set $\Fn(\kappa,\lambda)$ to be the family of all functions $p\colon F\to\lambda$, where $F$ is a finite subset of $\kappa$. We partially order $\Fn(\kappa,\lambda)$ by function extension: $p\leq q$ if and only if $p\supset q$.
\end{defn}

Just as the forcing $\omega^{<\omega}$ uses finite approximations to add a new function $\omega\to\omega$, the forcing $\Fn(\kappa,\lambda)$ uses finite approximations to add a new function $\kappa\to\lambda$.

\begin{prop}
  \label{prop:cohen-ccc}
  The partial order $\Fn(\kappa,\omega)$ is ccc.
\end{prop}

A family $\mathcal F$ of sets is called a \emph{delta system} if there exists a set $A$ such that for every $F,F'\in\mathcal G_0$ we have $F\cap F'=A$. The set $A$ is called the \emph{root} of the delta system.

\begin{lem}[Delta system lemma]
  \label{lem:delta-system}
  Let $\kappa$ be any cardinal, and let $\mathcal F$ be an uncountable family of finite subsets of $\kappa$. Then there exists an uncountable subset $\mathcal F_0\subset F$ which forms a delta system.
\end{lem}

\begin{proof}
  Since there are only countably many possible values of $|F|$ for $F\in\mathcal F$, by the pigeon-hole principle we can replace $\mathcal F$ with an uncountable subset of it such that $|F|=$ some fixed value $n$ for all $F\in\mathcal G$.

  Now we use induction on $n$. If $n=1$, then $\mathcal G$ forms a delta system automatically. Next suppose that the claim is true for families of sets of size $<n$. First consider the case that there is some $\alpha\in\kappa$ which lies in uncountably many of the $F\in\mathcal G$. Then by the inductive hypothesis we can find a delta system $\mathcal H_0$ contained in $\set{F\smallsetminus\{\alpha\}\mid F\in\mathcal G}$. It follows that the family $\mathcal G_0=\set{F_0\cup\{\alpha\}\mid F_0\in\mathcal H_0}$ is a delta system too.

  Second consider the case that every $\alpha\in\kappa$, is contained in just countably many of the $F\in\mathcal G$. Then we can inductively find an uncountable subfamily $\mathcal G_0\subset\mathcal G$ that is pairwise disjoint, and hence a delta system with trivial root. Indeed, if countably many pairwise disjoint sets $F_0,F_1,\ldots$ have been found so far, then all the ordinals in $\bigcup F_n$ are still only contained in countably many of the $F\in\mathcal G$. This means that we can find an $F$ disjoint from them all, and completes the proof of the claim.
\end{proof}

\begin{proof}[Proof of Proposition~\ref{prop:cohen-ccc}]
  It suffices to show that any uncountable $\mathcal G\subset\Fn(\kappa,\omega)$ contains two compatible elements. In fact we will show that it contains uncountably many pairwise compatible elements. For this we first consider the corresponding family of domains $\mathcal F=\set{\dom(f)\mid f\in\mathcal G}$. Then $\mathcal F$ is an uncountable family of finite subsets of $\kappa$. Thus by Lemma~\ref{lem:delta-system}, we can find an uncountable subset $\mathcal F_0\subset\mathcal F$ which forms a delta system with some root $A$.

  Now, let $\mathcal G_0\subset\mathcal G$ be the subset of $\mathcal G$ consisting of functions $g$ such that $\dom(g)\in\mathcal F_0$. These functions may disagree on $A$, but they can never disagree elsewhere. Since there are only countably many possibilities for $g\restriction A$, by the pigeon-hole principle we can replace $\mathcal G_0$ with an uncountable subset of it such that $g\restriction A=$ some fixed function $g_0$ for all $g$. It follows from this that every pair of elements of $\mathcal G_0$ is compatible.
\end{proof}

\begin{defn}
  For any cardinal $\kappa$, we define the set $\mathbb B_\kappa$ to be the collection of all nonnull subsets of $2^\kappa$, where two such sets are identified if they differ only on a null set. The elements of $\mathbb B_\kappa$ are ordered by $A\leq B$ iff $A\subset B$.
\end{defn}

\begin{lem}
  The partial order $\mathbb B_\kappa$ is ccc.
\end{lem}

\end{document}
