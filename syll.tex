\documentclass[12pt,oneside]{amsart}
\title{Math 522 Syllabus}

\usepackage[vscale=.8]{geometry}
\usepackage{setspace}\onehalfspacing
\renewcommand{\labelitemi}{$\circ$}

\begin{document}
\maketitle
\thispagestyle{empty}

\subsection*{Course information}
\begin{description}
\item[Meeting times] W,F from 9:00 to 10:15am
\item[Meeting place] M 124
\item[Text] Kunen, \emph{Set Theory} (Studies in Logic edition)
\item[My email] \texttt{scoskey@boisestate}
\item[My office] MG 237-A
\item[Office hours] \underline{\hspace{1in}}
\end{description}

\subsection*{Course topics}

This course is about sets of real numbers, and more specifically about mesauring the \emph{size} of sets of real numbers. But what does size mean? There at least three possibilities: cardinality, measure, and category. \emph{Cardinality} asks whether a set is countable or uncountable, \emph{measure} asks whether a set has zero length or nonzero length, and \emph{category} asks whether a set is meager or nonmeager. But how do these notions of size compare with one another?

Our investigation of mesaure and category will lead us to a number of \emph{independence results}, that is, statements that cannot be decided using the usual axioms of set theory. The most central of these statements is the \emph{continuum hypothesis}, which states that the set of all real numbers has cardinality equal to the first uncountable cardinal number. In order to prove this statement is indeed independent of set theory, we will study a technique called set-theoretic \emph{forcing}.

Once we have developed the machinery of forcing, we will return to the notions of measure and category. We will find that there are numerous cardinal numbers concerning null and meagure sets whose values, like the size of the set of all real numbers, are also independent of set theory. For example, one can prove that the smallest nonmeager set has cardinality independent of set theory.

\subsection*{Homework}
Homework problems will be assigned regularly. To receive a passing grade, you must complete all problems. To receive an A or B, your solutions should be correct. You may correct and resubmit your homework problems as many times as you wish.

\subsection*{Attendance and participation}
Attendance is required, with a few exceptions of course allowed. You should also participate in classroom discussions. Finally you may be asked to present homework solutions or other class material.


\end{document}
